\documentclass[10pt,handout,twocolumn,twoside,wordchoicegiven]{xourse}

%\usepackage{todonotes}

\newcommand{\todo}{}

\usepackage{esint} % for \oiint
\graphicspath{
{./}
{functionsOfSeveralVariables/}
{normalVectors/}
{lagrangeMultipliers/}
{vectorFields/}
{greensTheorem/}
{shapeOfThingsToCome/}
}


\usepackage{tkz-euclide}
\tikzset{>=stealth} %% cool arrow head
\tikzset{shorten <>/.style={ shorten >=#1, shorten <=#1 } } %% allows shorter vectors

\usetikzlibrary{backgrounds} %% for boxes around graphs
\usetikzlibrary{shapes,positioning}  %% Clouds and stars
\usetikzlibrary{matrix} %% for matrix
\usepgfplotslibrary{polar} %% for polar plots
\usetkzobj{all}
\usepackage[makeroom]{cancel} %% for strike outs
%\usepackage{mathtools} %% for pretty underbrace % Breaks Ximera
\usepackage{multicol}
\usepackage{pgffor} %% required for integral for loops


%% http://tex.stackexchange.com/questions/66490/drawing-a-tikz-arc-specifying-the-center
%% Draws beach ball
\tikzset{pics/carc/.style args={#1:#2:#3}{code={\draw[pic actions] (#1:#3) arc(#1:#2:#3);}}}



\usepackage{array}
\setlength{\extrarowheight}{+.1cm}   
\newdimen\digitwidth
\settowidth\digitwidth{9}
\def\divrule#1#2{
\noalign{\moveright#1\digitwidth
\vbox{\hrule width#2\digitwidth}}}





\newcommand{\RR}{\mathbb R}
\newcommand{\R}{\mathbb R}
\newcommand{\N}{\mathbb N}
\newcommand{\Z}{\mathbb Z}

%\newcommand{\sage}{\textsf{SageMath}}


%\renewcommand{\d}{\,d\!}
\renewcommand{\d}{\mathop{}\!d}
\newcommand{\dd}[2][]{\frac{\d #1}{\d #2}}
\newcommand{\pp}[2][]{\frac{\partial #1}{\partial #2}}
\renewcommand{\l}{\ell}
\newcommand{\ddx}{\frac{d}{\d x}}

\newcommand{\zeroOverZero}{\ensuremath{\boldsymbol{\tfrac{0}{0}}}}
\newcommand{\inftyOverInfty}{\ensuremath{\boldsymbol{\tfrac{\infty}{\infty}}}}
\newcommand{\zeroOverInfty}{\ensuremath{\boldsymbol{\tfrac{0}{\infty}}}}
\newcommand{\zeroTimesInfty}{\ensuremath{\small\boldsymbol{0\cdot \infty}}}
\newcommand{\inftyMinusInfty}{\ensuremath{\small\boldsymbol{\infty - \infty}}}
\newcommand{\oneToInfty}{\ensuremath{\boldsymbol{1^\infty}}}
\newcommand{\zeroToZero}{\ensuremath{\boldsymbol{0^0}}}
\newcommand{\inftyToZero}{\ensuremath{\boldsymbol{\infty^0}}}



\newcommand{\numOverZero}{\ensuremath{\boldsymbol{\tfrac{\#}{0}}}}
\newcommand{\dfn}{\textbf}
%\newcommand{\unit}{\,\mathrm}
\newcommand{\unit}{\mathop{}\!\mathrm}
\newcommand{\eval}[1]{\bigg[ #1 \bigg]}
\newcommand{\seq}[1]{\left( #1 \right)}
\renewcommand{\epsilon}{\varepsilon}
\renewcommand{\phi}{\varphi}


\renewcommand{\iff}{\Leftrightarrow}

\DeclareMathOperator{\arccot}{arccot}
\DeclareMathOperator{\arcsec}{arcsec}
\DeclareMathOperator{\arccsc}{arccsc}
\DeclareMathOperator{\si}{Si}
\DeclareMathOperator{\proj}{\vec{proj}}
\DeclareMathOperator{\scal}{scal}
\DeclareMathOperator{\sign}{sign}


%% \newcommand{\tightoverset}[2]{% for arrow vec
%%   \mathop{#2}\limits^{\vbox to -.5ex{\kern-0.75ex\hbox{$#1$}\vss}}}
\newcommand{\arrowvec}{\overrightarrow}
%\renewcommand{\vec}[1]{\arrowvec{\mathbf{#1}}}
\renewcommand{\vec}{\mathbf}
\newcommand{\veci}{{\boldsymbol{\hat{\imath}}}}
\newcommand{\vecj}{{\boldsymbol{\hat{\jmath}}}}
\newcommand{\veck}{{\boldsymbol{\hat{k}}}}
\newcommand{\vecl}{\boldsymbol{\l}}
\newcommand{\uvec}[1]{\mathbf{\hat{#1}}}
\newcommand{\utan}{\mathbf{\hat{t}}}
\newcommand{\unormal}{\mathbf{\hat{n}}}
\newcommand{\ubinormal}{\mathbf{\hat{b}}}

\newcommand{\dotp}{\bullet}
\newcommand{\cross}{\boldsymbol\times}
\newcommand{\grad}{\boldsymbol\nabla}
\newcommand{\divergence}{\grad\dotp}
\newcommand{\curl}{\grad\cross}
%\DeclareMathOperator{\divergence}{divergence}
%\DeclareMathOperator{\curl}[1]{\grad\cross #1}
\newcommand{\lto}{\mathop{\longrightarrow\,}\limits}

\renewcommand{\bar}{\overline}

\colorlet{textColor}{black} 
\colorlet{background}{white}
\colorlet{penColor}{blue!50!black} % Color of a curve in a plot
\colorlet{penColor2}{red!50!black}% Color of a curve in a plot
\colorlet{penColor3}{red!50!blue} % Color of a curve in a plot
\colorlet{penColor4}{green!50!black} % Color of a curve in a plot
\colorlet{penColor5}{orange!80!black} % Color of a curve in a plot
\colorlet{penColor6}{yellow!70!black} % Color of a curve in a plot
\colorlet{fill1}{penColor!20} % Color of fill in a plot
\colorlet{fill2}{penColor2!20} % Color of fill in a plot
\colorlet{fillp}{fill1} % Color of positive area
\colorlet{filln}{penColor2!20} % Color of negative area
\colorlet{fill3}{penColor3!20} % Fill
\colorlet{fill4}{penColor4!20} % Fill
\colorlet{fill5}{penColor5!20} % Fill
\colorlet{gridColor}{gray!50} % Color of grid in a plot

\newcommand{\surfaceColor}{violet}
\newcommand{\surfaceColorTwo}{redyellow}
\newcommand{\sliceColor}{greenyellow}




\pgfmathdeclarefunction{gauss}{2}{% gives gaussian
  \pgfmathparse{1/(#2*sqrt(2*pi))*exp(-((x-#1)^2)/(2*#2^2))}%
}


%%%%%%%%%%%%%
%% Vectors
%%%%%%%%%%%%%

%% Simple horiz vectors
\renewcommand{\vector}[1]{\left\langle #1\right\rangle}


%% %% Complex Horiz Vectors with angle brackets
%% \makeatletter
%% \renewcommand{\vector}[2][ , ]{\left\langle%
%%   \def\nextitem{\def\nextitem{#1}}%
%%   \@for \el:=#2\do{\nextitem\el}\right\rangle%
%% }
%% \makeatother

%% %% Vertical Vectors
%% \def\vector#1{\begin{bmatrix}\vecListA#1,,\end{bmatrix}}
%% \def\vecListA#1,{\if,#1,\else #1\cr \expandafter \vecListA \fi}

%%%%%%%%%%%%%
%% End of vectors
%%%%%%%%%%%%%

%\newcommand{\fullwidth}{}
%\newcommand{\normalwidth}{}



%% makes a snazzy t-chart for evaluating functions
%\newenvironment{tchart}{\rowcolors{2}{}{background!90!textColor}\array}{\endarray}

%%This is to help with formatting on future title pages.
\newenvironment{sectionOutcomes}{}{} 



%% Flowchart stuff
%\tikzstyle{startstop} = [rectangle, rounded corners, minimum width=3cm, minimum height=1cm,text centered, draw=black]
%\tikzstyle{question} = [rectangle, minimum width=3cm, minimum height=1cm, text centered, draw=black]
%\tikzstyle{decision} = [trapezium, trapezium left angle=70, trapezium right angle=110, minimum width=3cm, minimum height=1cm, text centered, draw=black]
%\tikzstyle{question} = [rectangle, rounded corners, minimum width=3cm, minimum height=1cm,text centered, draw=black]
%\tikzstyle{process} = [rectangle, minimum width=3cm, minimum height=1cm, text centered, draw=black]
%\tikzstyle{decision} = [trapezium, trapezium left angle=70, trapezium right angle=110, minimum width=3cm, minimum height=1cm, text centered, draw=black]

%% \iftikzexport
%% \newcommand\chapterstyle{}
%% \newcommand\sectionstyle{}
%% \else
%% \usepackage{lulu1}
%% \fi
%\usepackage{lulu1}

\title{Math 160 Calculus 1 for Colorado State University: Algebra and Trigonometry Review (Optional)}

\begin{document}
\maketitle

\setcounter{tocdepth}{2}

\part{Functions}

%% What is a Function?
\chapterstyle
\activity{understandingFunctions/titlePage.tex}
\sectionstyle
\activity{PreRequisiteXards/U1Functions/1.1WhatIsAFunction/1.1PreQuiz/1.1PreQuiz.tex}   \activity{PreRequisiteXards/U1Functions/1.1WhatIsAFunction/1.1PreQuizReflection/1.1PreQuizReflection.tex}
\activity{prerequisiteVideos/functionNotation.tex}
\activity{prerequisiteVideos/functionsAndTheirGraphs.tex}
\activity{understandingFunctions/breakGround.tex}
\activity{understandingFunctions/digInForEachInputExactlyOneOutput.tex}

%% Piece-Wise Functions
%\chapterstyle
%\activity{PreRequisiteXards/U1Functions/1.2PieceWiseFunctions/titlePage.tex}
%\sectionstyle
%NXA: What are piece-wise Functions, 1 or 2 examples.  Evaluating, domain/range. (See U2 for how to graph) Note: Part of this is in Ben's videos...except can't do the graphing yet...so the video xard is better left to U2

%% Composition of Functions
\chapterstyle
\activity{PreRequisiteXards/U1Functions/1.3CompositionOfFunctions/titlePage.tex}
\sectionstyle
\activity{prerequisiteVideos/compositionOfFunction.tex}
\activity{understandingFunctions/digInCompositionOfFunctions.tex}

%% Inverses of Functions
%\chapterstyle
%\activity{PreRequisiteXards/U1Functions/1.4InversesOfFunctions/titlePage.tex}
%\sectionstyle
%\activity{understandingFunctions/digInInversesOfFunctions.tex}

\part{Review of Common Functions}

%% Polynomials
\chapterstyle
\activity{PreRequisiteXards/U2CommonFunctions/2.1Polynomials/2.1TitlePage/2.1TitlePage.tex}
\sectionstyle
\activity{PreRequisiteXards/U2CommonFunctions/2.1Polynomials/2.1PreQuiz/2.1PreQuiz.tex}
\activity{PreRequisiteXards/U2CommonFunctions/2.1Polynomials/2.1PreQuizReflection/2.1PreQuizReflection.tex}
\activity{PreRequisiteXards/U2CommonFunctions/2.1Polynomials/digInLinearFunctions/digInLinearFunctions.tex}
\activity{prerequisiteVideos/polyGraphBehavior.tex}
\activity{reviewOfFamousFunctions/breakGround.tex}
\activity{reviewOfFamousFunctions/digInPolynomialFunctions.tex}

%% Rational Functions
\chapterstyle
\activity{PreRequisiteXards/U2CommonFunctions/2.2RationalFunctions/titlePage.tex}
\sectionstyle
%NXA: 1/x-like functions
\activity{reviewOfFamousFunctions/digInRationalFunctions.tex}

%% Root Functions
%\chapterstyle
%\activity{PreRequisiteXards/U2CommonFunctions/2.3RootFunctions/titlePage.tex}
%\sectionstyle
%NXA: Even Root Functions (Mainly Square root)
%NXA: Odd Root Functions (Mainly cube root)

%% Piece-Wise Functions
\chapterstyle
\activity{PreRequisiteXards/U2CommonFunctions/2.4PieceWiseFunctions/titlePage.tex}
\activity{prerequisiteVideos/PieceWiseFunctions.tex}

%% The Absolute Value Function
\chapterstyle
\activity{PreRequisiteXards/U2CommonFunctions/2.5TheAbsoluteValueFunction/titlePage.tex}
\sectionstyle
%NXA: Absolute Value Fxns (include one without a corner!). BenVidz.

%% Function Transformations 
%\chapterstyle
%\activity{PreRequisiteXards/U2CommonFunctions/2.6FunctionTransformations/titlePage.tex}
%\sectionstyle
%NXA: Using function transformations to get a rough idea of ANY graph. (vertical shifts, horizontal shifts, reflection over x-axis, vertical compression/stretching, horizontal compression/stretch.)

%% Graphing Other Functions
%\chapterstyle
%activity{PreRequisiteXards/U2CommonFunctions/2.7GraphingOtherFunctions/titlePage.tex}
%\sectionstyle
%NXA: When it doubt, use a t-table!  Circle, semi-circle and a few other graphs that didn't fall into a previous category that we can understand after plotting some points.

\part{Fractions}

%% Arithmetic and Algebra of Fractions
\chapterstyle
\activity{PreRequisiteXards/U3Fractions/3.1ArithmeticAndAlgebraOfFractions/3.1TitlePage/3.1TitlePage.tex}
\sectionstyle
\activity{PreRequisiteXards/U3Fractions/3.1ArithmeticAndAlgebraOfFractions/3.1PreQuiz/3.1PreQuiz.tex}
\activity{PreRequisiteXards/U3Fractions/3.1ArithmeticAndAlgebraOfFractions/3.1PreQuizReflection/3.1PreQuizReflection.tex}
\activity{PreRequisiteXards/U3Fractions/3.1ArithmeticAndAlgebraOfFractions/digInFractions/digInFractions.tex}
%NXA: Adding, Subtracting
%NXA: Multiplying, Dividing
%NXA: Simplifying
%NXA: Fractions with algebra (+, -, x, dividing, simplifying)

%\part{Exponents and Roots}

%% Exponents and Roots
%\chapterstyle
%\activity{PreRequisiteXards/U4ExponentsAndRoots/4.1ExponentsAndRoots/titlePage.tex}
%\sectionstyle
%NXA: What are Exponents (Focus at first on whole number exp's)?
%NXA: Properties/Patterns of Exponents (Focus examples on how these patterns work for most any exponent: fractional, negative, 0, etc.; use table to explain ^0 and ^(-) patterns.)
%NXA: Fractional Exponents: How to re-write as a root raised to a whole number exponent. 
%NXA: Using the properties of exponents: Simplifying roots, rationalizing the denominator (including using the conjugate)

\part{Solving Equations and Linear Inequalities}

%% Solving Equations
%\chapterstyle
%\activity{PreRequisiteXards/U5SolvingEquationsAndInequalities/5.1SolvingEquations/titlePage.tex}
%\sectionstyle
%Solving Equations Xard (Linear, whole exponents, roots, fractional exponents.)

%%Solving Linear Inequalities
\chapterstyle
\activity{PreRequisiteXards/U5SolvingEquationsAndInequalities/5.2SolvingLinearInequalities/titlePage.tex}
\sectionstyle
\activity{prerequisiteVideos/algebraInequalities.tex}

\part{Multiplying and Factoring Polynomials}

\sectionstyle
\activity{PreRequisiteXards/U6MultiplyingAndFactoringPolynomials/6.1MultiplyingPolynomials/6.1PreQuiz/6.1PreQuiz.tex}
\activity{PreRequisiteXards/U6MultiplyingAndFactoringPolynomials/6.1MultiplyingPolynomials/6.1PreQuizReflection/6.1PreQuizReflection.tex}

%% Multiplying Polynomials
\chapterstyle
\activity{PreRequisiteXards/U6MultiplyingAndFactoringPolynomials/6.1MultiplyingPolynomials/6.1TitlePage/6.1TitlePage.tex}
\sectionstyle
\activity{prerequisiteVideos/distribAndMultPolys.tex}

%% Factoring Polynomials
\chapterstyle
\activity{PreRequisiteXards/U6MultiplyingAndFactoringPolynomials/6.2FactoringPolynomials/titlePage.tex}
\sectionstyle
\activity{prerequisiteVideos/fundamentalsOfFactoring.tex}

%% Simplifying Rational Functions
\chapterstyle
\activity{PreRequisiteXards/U6MultiplyingAndFactoringPolynomials/6.3SimplifyingRationalFunctions/titlePage.tex}
\sectionstyle
\activity{understandingFunctions/breakGround.tex}
\activity{PreRequisiteXards/U6MultiplyingAndFactoringPolynomials/6.3SimplifyingRationalFunctions/breakGround.tex}
\activity{PreRequisiteXards/U6MultiplyingAndFactoringPolynomials/6.3SimplifyingRationalFunctions/digInSimplifyingRationalFunctions.tex}

%%Common Mistakes
%\chapterstyle
%\activity{PreRequisiteXards/U6MultiplyingAndFactoringPolynomials/6.4CommonMistakes/titlePage.tex}
%\sectionstyle
% NXA Common misconceptions with rational exponents (Exponents, square roots, and rational functions)

\part{Trigonometry}

%% Trig Functions
\chapterstyle
\activity{PreRequisiteXards/U7Trigonometry/7.1TrigFunctions/7.1TitlePage/7.1TitlePage.tex}
\sectionstyle
\activity{PreRequisiteXards/U7Trigonometry/7.1TrigFunctions/7.1PreQuiz/7.1PreQuiz.tex}
\activity{PreRequisiteXards/U7Trigonometry/7.1TrigFunctions/7.1PreQuizReflection/7.1PreQuizReflection.tex}
\activity{prerequisiteVideos/trigUnitCircle.tex}
\activity{reviewOfFamousFunctions/digInTrigonometricFunctions.tex}
% Add dynamic unit circle xard to aid transition from unit circle to graphs of trig. fxns? (https://www.geogebra.org/m/RC8JeqBN)

%% Solving Trig Equations
\chapterstyle
\activity{PreRequisiteXards/U7Trigonometry/7.2SolvingTrigEquations/7.2TitlePage/7.2TitlePage.tex}
\sectionstyle
\activity{PreRequisiteXards/U7Trigonometry/7.2SolvingTrigEquations/7.2PreQuiz/7.2PreQuiz.tex}
\activity{PreRequisiteXards/U7Trigonometry/7.2SolvingTrigEquations/7.2PreQuizReflection/7.2PreQuizReflection.tex}
\activity{prerequisiteVideos/trigSolvingEquations.tex}

\part{Geometry of Common 2D and 3D Shapes}

%% 2D Shapes
\chapterstyle
\activity{PreRequisiteXards/U8GeometryOfCommon2DAnd3DShapes/8.12DShapes/8.1TitlePage/8.1TitlePage.tex}
\sectionstyle
\activity{PreRequisiteXards/U8GeometryOfCommon2DAnd3DShapes/8.12DShapes/8.1PreQuiz/8.1PreQuiz.tex}
\activity{PreRequisiteXards/U8GeometryOfCommon2DAnd3DShapes/8.12DShapes/8.1PreQuizReflection/8.1PreQuizReflection.tex}
\activity{prerequisiteVideos/trigGeometricMeasures.tex}
% Area of a circle, re-write circumference, too.
% Area and perimeter of rectangles (squares) and triangles. [Basically can be done in one fell swoop building from area of a rectangle: square is just a special case where l=w, for a triangle, draw a rectangle around it and realize the triangle takes up half the area.]

%% 3D Shapes
%\chapterstyle
%\activity{PreRequisiteXards/U8GeometryOfCommon2DAnd3DShapes/8.23DShapes/titlePage.tex}
%\sectionstyle
% NXA: Right rectangular prism / Right circular cylinder
% NXA: Cones and spheres

%\part{Materials We Aren't Using in Math 160}

%\chapterstyle
%\activity{understandingFunctions/digInInversesOfFunctions.tex}
%\activity{reviewOfFamousFunctions/digInExponentialAndLogarithmeticFunctions.tex}
%\activity{prerequisiteVideos/introLogarithms.tex}

%\iftikzexport\else\printindex\fi
\end{document}

