\documentclass[handout,nooutcomes]{ximera}
%handout
%wordchoicegiven
%space
%nooutcomes
\title{Math 160 Lab 3}
\author{Jeremy Buss} %Used Bart Snapp and Jim Fowler's mooculus textbook, and Ben Sencindiver's Linear Approx. Lab as a guide
%\usepackage{todonotes}

\newcommand{\todo}{}

\usepackage{esint} % for \oiint
\graphicspath{
{./}
{functionsOfSeveralVariables/}
{normalVectors/}
{lagrangeMultipliers/}
{vectorFields/}
{greensTheorem/}
{shapeOfThingsToCome/}
}


\usepackage{tkz-euclide}
\tikzset{>=stealth} %% cool arrow head
\tikzset{shorten <>/.style={ shorten >=#1, shorten <=#1 } } %% allows shorter vectors

\usetikzlibrary{backgrounds} %% for boxes around graphs
\usetikzlibrary{shapes,positioning}  %% Clouds and stars
\usetikzlibrary{matrix} %% for matrix
\usepgfplotslibrary{polar} %% for polar plots
\usetkzobj{all}
\usepackage[makeroom]{cancel} %% for strike outs
%\usepackage{mathtools} %% for pretty underbrace % Breaks Ximera
\usepackage{multicol}
\usepackage{pgffor} %% required for integral for loops


%% http://tex.stackexchange.com/questions/66490/drawing-a-tikz-arc-specifying-the-center
%% Draws beach ball
\tikzset{pics/carc/.style args={#1:#2:#3}{code={\draw[pic actions] (#1:#3) arc(#1:#2:#3);}}}



\usepackage{array}
\setlength{\extrarowheight}{+.1cm}   
\newdimen\digitwidth
\settowidth\digitwidth{9}
\def\divrule#1#2{
\noalign{\moveright#1\digitwidth
\vbox{\hrule width#2\digitwidth}}}





\newcommand{\RR}{\mathbb R}
\newcommand{\R}{\mathbb R}
\newcommand{\N}{\mathbb N}
\newcommand{\Z}{\mathbb Z}

%\newcommand{\sage}{\textsf{SageMath}}


%\renewcommand{\d}{\,d\!}
\renewcommand{\d}{\mathop{}\!d}
\newcommand{\dd}[2][]{\frac{\d #1}{\d #2}}
\newcommand{\pp}[2][]{\frac{\partial #1}{\partial #2}}
\renewcommand{\l}{\ell}
\newcommand{\ddx}{\frac{d}{\d x}}

\newcommand{\zeroOverZero}{\ensuremath{\boldsymbol{\tfrac{0}{0}}}}
\newcommand{\inftyOverInfty}{\ensuremath{\boldsymbol{\tfrac{\infty}{\infty}}}}
\newcommand{\zeroOverInfty}{\ensuremath{\boldsymbol{\tfrac{0}{\infty}}}}
\newcommand{\zeroTimesInfty}{\ensuremath{\small\boldsymbol{0\cdot \infty}}}
\newcommand{\inftyMinusInfty}{\ensuremath{\small\boldsymbol{\infty - \infty}}}
\newcommand{\oneToInfty}{\ensuremath{\boldsymbol{1^\infty}}}
\newcommand{\zeroToZero}{\ensuremath{\boldsymbol{0^0}}}
\newcommand{\inftyToZero}{\ensuremath{\boldsymbol{\infty^0}}}



\newcommand{\numOverZero}{\ensuremath{\boldsymbol{\tfrac{\#}{0}}}}
\newcommand{\dfn}{\textbf}
%\newcommand{\unit}{\,\mathrm}
\newcommand{\unit}{\mathop{}\!\mathrm}
\newcommand{\eval}[1]{\bigg[ #1 \bigg]}
\newcommand{\seq}[1]{\left( #1 \right)}
\renewcommand{\epsilon}{\varepsilon}
\renewcommand{\phi}{\varphi}


\renewcommand{\iff}{\Leftrightarrow}

\DeclareMathOperator{\arccot}{arccot}
\DeclareMathOperator{\arcsec}{arcsec}
\DeclareMathOperator{\arccsc}{arccsc}
\DeclareMathOperator{\si}{Si}
\DeclareMathOperator{\proj}{\vec{proj}}
\DeclareMathOperator{\scal}{scal}
\DeclareMathOperator{\sign}{sign}


%% \newcommand{\tightoverset}[2]{% for arrow vec
%%   \mathop{#2}\limits^{\vbox to -.5ex{\kern-0.75ex\hbox{$#1$}\vss}}}
\newcommand{\arrowvec}{\overrightarrow}
%\renewcommand{\vec}[1]{\arrowvec{\mathbf{#1}}}
\renewcommand{\vec}{\mathbf}
\newcommand{\veci}{{\boldsymbol{\hat{\imath}}}}
\newcommand{\vecj}{{\boldsymbol{\hat{\jmath}}}}
\newcommand{\veck}{{\boldsymbol{\hat{k}}}}
\newcommand{\vecl}{\boldsymbol{\l}}
\newcommand{\uvec}[1]{\mathbf{\hat{#1}}}
\newcommand{\utan}{\mathbf{\hat{t}}}
\newcommand{\unormal}{\mathbf{\hat{n}}}
\newcommand{\ubinormal}{\mathbf{\hat{b}}}

\newcommand{\dotp}{\bullet}
\newcommand{\cross}{\boldsymbol\times}
\newcommand{\grad}{\boldsymbol\nabla}
\newcommand{\divergence}{\grad\dotp}
\newcommand{\curl}{\grad\cross}
%\DeclareMathOperator{\divergence}{divergence}
%\DeclareMathOperator{\curl}[1]{\grad\cross #1}
\newcommand{\lto}{\mathop{\longrightarrow\,}\limits}

\renewcommand{\bar}{\overline}

\colorlet{textColor}{black} 
\colorlet{background}{white}
\colorlet{penColor}{blue!50!black} % Color of a curve in a plot
\colorlet{penColor2}{red!50!black}% Color of a curve in a plot
\colorlet{penColor3}{red!50!blue} % Color of a curve in a plot
\colorlet{penColor4}{green!50!black} % Color of a curve in a plot
\colorlet{penColor5}{orange!80!black} % Color of a curve in a plot
\colorlet{penColor6}{yellow!70!black} % Color of a curve in a plot
\colorlet{fill1}{penColor!20} % Color of fill in a plot
\colorlet{fill2}{penColor2!20} % Color of fill in a plot
\colorlet{fillp}{fill1} % Color of positive area
\colorlet{filln}{penColor2!20} % Color of negative area
\colorlet{fill3}{penColor3!20} % Fill
\colorlet{fill4}{penColor4!20} % Fill
\colorlet{fill5}{penColor5!20} % Fill
\colorlet{gridColor}{gray!50} % Color of grid in a plot

\newcommand{\surfaceColor}{violet}
\newcommand{\surfaceColorTwo}{redyellow}
\newcommand{\sliceColor}{greenyellow}




\pgfmathdeclarefunction{gauss}{2}{% gives gaussian
  \pgfmathparse{1/(#2*sqrt(2*pi))*exp(-((x-#1)^2)/(2*#2^2))}%
}


%%%%%%%%%%%%%
%% Vectors
%%%%%%%%%%%%%

%% Simple horiz vectors
\renewcommand{\vector}[1]{\left\langle #1\right\rangle}


%% %% Complex Horiz Vectors with angle brackets
%% \makeatletter
%% \renewcommand{\vector}[2][ , ]{\left\langle%
%%   \def\nextitem{\def\nextitem{#1}}%
%%   \@for \el:=#2\do{\nextitem\el}\right\rangle%
%% }
%% \makeatother

%% %% Vertical Vectors
%% \def\vector#1{\begin{bmatrix}\vecListA#1,,\end{bmatrix}}
%% \def\vecListA#1,{\if,#1,\else #1\cr \expandafter \vecListA \fi}

%%%%%%%%%%%%%
%% End of vectors
%%%%%%%%%%%%%

%\newcommand{\fullwidth}{}
%\newcommand{\normalwidth}{}



%% makes a snazzy t-chart for evaluating functions
%\newenvironment{tchart}{\rowcolors{2}{}{background!90!textColor}\array}{\endarray}

%%This is to help with formatting on future title pages.
\newenvironment{sectionOutcomes}{}{} 



%% Flowchart stuff
%\tikzstyle{startstop} = [rectangle, rounded corners, minimum width=3cm, minimum height=1cm,text centered, draw=black]
%\tikzstyle{question} = [rectangle, minimum width=3cm, minimum height=1cm, text centered, draw=black]
%\tikzstyle{decision} = [trapezium, trapezium left angle=70, trapezium right angle=110, minimum width=3cm, minimum height=1cm, text centered, draw=black]
%\tikzstyle{question} = [rectangle, rounded corners, minimum width=3cm, minimum height=1cm,text centered, draw=black]
%\tikzstyle{process} = [rectangle, minimum width=3cm, minimum height=1cm, text centered, draw=black]
%\tikzstyle{decision} = [trapezium, trapezium left angle=70, trapezium right angle=110, minimum width=3cm, minimum height=1cm, text centered, draw=black]

\outcome{Apply the notion of extreme values to find an optimal solution to a problem.}
\outcome{Explore a novel application of optimization techniques.}
\outcome{Interpret the results of an optimal solution in context.}

\begin{document}

\section{Calculus 1 Lab 3 \\ An Optimal Shape}

%% Have to edit the date here each semester.
\begin{abstract}
This is Lab 3 for Math 160 - Due Friday, October 27, 2017 at 5:00PM MST.
This lab will guide you through an optimization problem whose solution is relevant in many fields of study.\\

Unless stated otherwise, input answers in \underline{exact form} in this lab.
\end{abstract}

\maketitle

\hspace{2cm}In class we explored a problem that involved finding the optimal dimensions for a rectangular pen that maximizes its area. In this lab we will examine what happens if we optimize the area of a regular polygonal pen with n sides. Our goal is finding the optimal number of sides to use when building a pen.\\

\medskip
%%%%%% Area of an n-sided regular polygon
\hspace{2cm}To optimize the area of an n-sided regular polygon we will need a function that represents the area of the polygon. One way to do this is to split the polygon into smaller, more manageable pieces. Since we are working with a regular polygon (A polygon whose sides and angles are all the same size), a natural choice is to cut it into triangles from the center as in the figure below.

%\includegraphics[scale=0.25]{Octagon.png}

%\begin{comment}
%%Fancy nice picture of a triangle
\begin{center}
\usetikzlibrary{calc,patterns,angles,quotes}
\begin{tikzpicture}

\coordinate (origin) at (0,0);
    \coordinate (a) at (4.62,1.91);
    \coordinate (b) at (1.91,4.62);
    \coordinate (c) at (-1.91,4.62);
    \coordinate (d) at (-4.62,1.91);
    \coordinate (e) at (-4.62,-1.91);
    \coordinate (f) at (-1.91,-4.62);
    \coordinate (g) at (1.91,-4.62);
    \coordinate (h) at (4.62,-1.91);
    \coordinate (midpoint) at (4.62,0);

    %draw innards
    \draw[thick, gray] (origin) -- (midpoint);
    \draw[thick, gray] (origin) -- (e);
    \draw[thick, gray] (c) -- (g);
    \draw[thick, gray] (d) -- (origin);
    \draw[thick, gray] (f) -- (b);
    \draw[thick, blue] (a) -- (origin);
    \draw[thick, blue] (h) -- (origin);
    
    %% draw octagon
    \draw[thick] (a) -- (b);
    \draw[thick] (b) -- (c);
    \draw[thick] (c) -- (d);
    \draw[thick] (d) -- (e);
    \draw[thick] (e) -- (f);
    \draw[thick] (f) -- (g);
    \draw[thick] (g) -- (h);
    \draw[thick,blue] (h) -- (a);
    \node[above=1pt of {(2.31,.96)}] {$r$};
    
    
    \pic [draw, "$\theta$", angle eccentricity=1.5] {angle = midpoint--origin--a};
  \end{tikzpicture}
\end{center}
%\end{comment}

\begin{problem}
What is the area of the blue triangle in terms of $\theta$? $\answer[given]{r^2\sin(\theta)\cos(\theta)}$.

How large is the angle $\theta$ in the figure above? $\answer[given]{\pi/8}$. (Remember to use radians instead of degrees!)

How large is the angle $\theta$ for a regular n-sided polygon? $\answer[given]{\pi/n}$.

What is the area of the blue triangle in terms of $n$ and $r$? $\answer[given]{r^2\sin(\pi/n)\cos(\pi/n)}$.

Then what is the total area of an n-sided regular polygon in terms of $n$ and $r$?\\
Area $=\answer[given]{(n)(r^2)\sin(\pi/n)\cos(\pi/n)}$.

\end{problem}

\bigskip

\hspace{2cm}Recall when finding the optimal dimensions for a rectangular pen we found an area formula in terms of two variables. We wished to find the derivative, so we needed to rewrite the formula in terms of only one variable. We used the fact that the length of fencing was limited in order to write one variable in terms of the other. Let's try the same strategy here.\\
\hspace{2cm}Let $P$ be a constant representing the length of fence that we can use to build the pen. Refer back to the triangle diagram to compute the perimeter of the polygon in terms of $n$ and $r$.

\begin{problem}
$P = \answer[given]{2(r)(n)\sin(\pi/n)}$.

Since we're keeping the perimeter $P$ constant, we can solve for $r$...

$r = \answer[given]{P/(2(n)\sin(\pi/n))}$.

...and use it to rewrite area in terms of $n$ and $P$.

$A(n) = \answer[given]{P^2\cot(\pi/n)/(4n)}$.

\textbf{We recommend simplifying your area formula} so that it contains only one trig function. This will simplify finding its derivative.

$A(n) = \answer[given]{P^2\cot(\pi/n)/(4n)}$.

\bigskip

What values of $n$ make sense? That is, what is the domain of $A(n)$ in the context of this problem? 

\wordChoice{\choice{real numbers}\choice[correct]{integers}} \wordChoice{\choice[correct]{greater than or equal to}\choice{less than or equal to}} $\answer{3}$.

\begin{freeResponse}
In order to use calculus to find extreme values we need the function to be differentiable, which means it must at least be defined and continuous on an open interval. If we extend $A(n)$ to be defined on $[3,\infty)$ will any maximums or minimums we find still be maximums and minimums when we restrict $n$ to the integers? Why or why not?
\end{freeResponse}

\bigskip

\hspace{2cm}Now that we have $A(n)$ in terms of one variable we're ready to find the derivative! It may be helpful to simplify your formula for the area first. Rewrite it so that it only has one trig function.

\textbf{You will want to be able to copy and paste the derivative for later use}. Write it in a separate text document, and paste it into the answer box below to check your work.

$A'(n) = \answer[given]{P^2((4\pi/n)\csc^2(\pi/n)-4\cot(\pi/n))/(16n^2)}$.

\end{problem}

\bigskip

\begin{problem}
\hspace{2cm}Yikes! Ok, that's a beast, but let's not lose our heads. We're hunting critical points after all. Where do we look for critical points?
\begin{selectAll}
    \choice{Where $A(n_0)$ does not exist.}
    \choice[correct]{Where $A'(n_0)$ does not exist.}
    \choice{Where $A(n_0)$ is zero.}
    \choice[correct]{Where $A'(n_0)$ is zero.}
    \choice[correct]{At endpoints of the domain of $A(n)$}
    \choice{At endpoints of the domain of $A'(n)$}
\end{selectAll}
\end{problem}

\bigskip

\begin{problem}
Notice $A'(0)$ and $A'(1)$ do not exist. Are $n=0$ and $n=1$ critical points?
\begin{multipleChoice}
    \choice{Yes, if $A'(n_0)$ does not exist then $n_0$ is a critical point.}
    \choice{Yes, if $A'(n_0)$ does not exist then $n_0$ is a critical point.}
    \choice{No, $n=0$ and $n=1$ are not critical points because they are not in the domain of $A'(n)$.}
    \choice{No, $n=0$ and $n=1$ are not critical points because $A'(0) \ne 0$ and $A'(1) \ne 0$.}
    \choice[correct]{No, $n=0$ and $n=1$ are not critical points because $n=0$ and $n=1$ are not in the domain of $A(n)$.}
\end{multipleChoice}
\end{problem}

\bigskip

\hspace{2cm}Now to see if $A'(n)=0$ anywhere. This equation can't be solved algebraically, but let's persist! After all, algebra is not our only tool! Use the Desmos widget to plot a graph of $A'(n)$.

Enter the derivative you computed earlier. (You may need to retype "pi" if desmos does not interpret it as $\pi$.)

\[
\graph[panel]{y=}
\]

\begin{problem}
Where is $A'(n)=0$?

\begin{multipleChoice}
    \choice{$A'(n)=0$ at $n=50$.}
    \choice[correct]{$A'(n)$ is never zero, it is always positive for $n>=3$.}
    \choice{$A'(n)$ is never zero, it is always negative for $n>=3$.}
\end{multipleChoice}
\end{problem}

\bigskip

\begin{problem}
Recall that we extended the domain of $A(n)$ to $[3,\infty)$, so we have a critical point at $n= \answer[given]{3}$.
\end{problem}

\bigskip

\begin{problem}
Since $A'(n)$ is always \wordChoice{\choice[correct]{positive}\choice{negative}}, $A(n)$ is always \wordChoice{\choice[correct]{increasing}\choice{decreasing}}.\\
\smallskip
At the endpoint $n=3$ $A(n)$ has a \wordChoice{\choice{local maximum}\choice[correct]{local minimum}\choice{neither a maximum nor a minimum}}\\
\smallskip
Intuitively, do you think that as $n \to \infty$ $A(n)$ can \wordChoice{\choice[correct]{increase}\choice{decrease}} without bound? \wordChoice{\choice{yes}\choice[correct]{no}}.\\ %%This could be interesting to look at. Does a 'yes' indicate a lack of intuition or a lack of motivation?
\end{problem}

\medskip

\begin{problem}
Let's investigate! Hints are provided if you need them.\\
\[\lim_{n \to \infty}A(n) = \answer{P^2/(4(\pi))}\]
\begin{hint}
	It may be helpful to rewrite $A(n)$ in terms of sine and cosine:\\ $A(n) = \frac{ \answer{P^2\cos(\pi/n)} }{ \answer{4n\sin(\pi/n)} }$.
\end{hint}
\begin{hint}
    What is the limit of the numerator? \[\lim_{n \to \infty} P^2\cos(\frac{\pi}{n}) = \answer{P^2}\]
\end{hint}
\begin{hint}
  Now consider the denominator. Recall \[\lim_{h \to 0} \frac{\sin(h)}{h} = \answer{1}\]
\end{hint}
\begin{hint}
  Can we manufacture a denominator for $sin(\frac{\pi}{n})$ that matches the inner function $\frac{\pi}{n}?$\\
  \[4n\sin(\frac{\pi}{n}) = 4(\frac{1}{1/n})\sin(\frac{\pi}{n}) = \frac{4\,\answer{\pi}\,\sin(\frac{\pi}{n})}{\answer{\pi/n}}\]
\end{hint}
\begin{hint}
     Then \[\lim_{n \to \infty} 4n\sin(\frac{\pi}{n}) = \lim_{n \to \infty} \frac{4\pi\sin(\frac{\pi}{n})}{(\frac{\pi}{n})} = \answer{4\pi}.\]
\end{hint}
\end{problem}

\begin{problem}
What does this limit mean? Remember $A(n)$ represents
\begin{multipleChoice}
    \choice[correct]{The area of an n-sided regular polygon}
    \choice{An n-sided regular polygon}
    \choice{The perimeter of an n-sided regular polygon}
\end{multipleChoice}
\begin{freeResponse}
As $n$ approaches infinity, what happens to the shapes represented by $A(n)$?
\end{freeResponse}
\end{problem}

\begin{problem}
What shape has maximal area for a fixed perimeter $P$?
\begin{multipleChoice}
    \choice{A triangle}
    \choice{A parabola}
    \choice{A square}
    \choice[correct]{A circle}
\end{multipleChoice}
\end{problem}

\bigskip

\begin{problem}
If $P = 4\pi$ then \[\lim_{n \to \infty} A(n) = \answer{4\pi}\].\\
\begin{freeResponse}
Does this answer make sense with respect to the shape that you think maximizes the area for a fixed perimeter? Explain.
\end{freeResponse}
\end{problem}



\end{document}
