\documentclass{ximera}

%\usepackage{todonotes}

\newcommand{\todo}{}

\usepackage{esint} % for \oiint
\graphicspath{
{./}
{functionsOfSeveralVariables/}
{normalVectors/}
{lagrangeMultipliers/}
{vectorFields/}
{greensTheorem/}
{shapeOfThingsToCome/}
}


\usepackage{tkz-euclide}
\tikzset{>=stealth} %% cool arrow head
\tikzset{shorten <>/.style={ shorten >=#1, shorten <=#1 } } %% allows shorter vectors

\usetikzlibrary{backgrounds} %% for boxes around graphs
\usetikzlibrary{shapes,positioning}  %% Clouds and stars
\usetikzlibrary{matrix} %% for matrix
\usepgfplotslibrary{polar} %% for polar plots
\usetkzobj{all}
\usepackage[makeroom]{cancel} %% for strike outs
%\usepackage{mathtools} %% for pretty underbrace % Breaks Ximera
\usepackage{multicol}
\usepackage{pgffor} %% required for integral for loops


%% http://tex.stackexchange.com/questions/66490/drawing-a-tikz-arc-specifying-the-center
%% Draws beach ball
\tikzset{pics/carc/.style args={#1:#2:#3}{code={\draw[pic actions] (#1:#3) arc(#1:#2:#3);}}}



\usepackage{array}
\setlength{\extrarowheight}{+.1cm}   
\newdimen\digitwidth
\settowidth\digitwidth{9}
\def\divrule#1#2{
\noalign{\moveright#1\digitwidth
\vbox{\hrule width#2\digitwidth}}}





\newcommand{\RR}{\mathbb R}
\newcommand{\R}{\mathbb R}
\newcommand{\N}{\mathbb N}
\newcommand{\Z}{\mathbb Z}

%\newcommand{\sage}{\textsf{SageMath}}


%\renewcommand{\d}{\,d\!}
\renewcommand{\d}{\mathop{}\!d}
\newcommand{\dd}[2][]{\frac{\d #1}{\d #2}}
\newcommand{\pp}[2][]{\frac{\partial #1}{\partial #2}}
\renewcommand{\l}{\ell}
\newcommand{\ddx}{\frac{d}{\d x}}

\newcommand{\zeroOverZero}{\ensuremath{\boldsymbol{\tfrac{0}{0}}}}
\newcommand{\inftyOverInfty}{\ensuremath{\boldsymbol{\tfrac{\infty}{\infty}}}}
\newcommand{\zeroOverInfty}{\ensuremath{\boldsymbol{\tfrac{0}{\infty}}}}
\newcommand{\zeroTimesInfty}{\ensuremath{\small\boldsymbol{0\cdot \infty}}}
\newcommand{\inftyMinusInfty}{\ensuremath{\small\boldsymbol{\infty - \infty}}}
\newcommand{\oneToInfty}{\ensuremath{\boldsymbol{1^\infty}}}
\newcommand{\zeroToZero}{\ensuremath{\boldsymbol{0^0}}}
\newcommand{\inftyToZero}{\ensuremath{\boldsymbol{\infty^0}}}



\newcommand{\numOverZero}{\ensuremath{\boldsymbol{\tfrac{\#}{0}}}}
\newcommand{\dfn}{\textbf}
%\newcommand{\unit}{\,\mathrm}
\newcommand{\unit}{\mathop{}\!\mathrm}
\newcommand{\eval}[1]{\bigg[ #1 \bigg]}
\newcommand{\seq}[1]{\left( #1 \right)}
\renewcommand{\epsilon}{\varepsilon}
\renewcommand{\phi}{\varphi}


\renewcommand{\iff}{\Leftrightarrow}

\DeclareMathOperator{\arccot}{arccot}
\DeclareMathOperator{\arcsec}{arcsec}
\DeclareMathOperator{\arccsc}{arccsc}
\DeclareMathOperator{\si}{Si}
\DeclareMathOperator{\proj}{\vec{proj}}
\DeclareMathOperator{\scal}{scal}
\DeclareMathOperator{\sign}{sign}


%% \newcommand{\tightoverset}[2]{% for arrow vec
%%   \mathop{#2}\limits^{\vbox to -.5ex{\kern-0.75ex\hbox{$#1$}\vss}}}
\newcommand{\arrowvec}{\overrightarrow}
%\renewcommand{\vec}[1]{\arrowvec{\mathbf{#1}}}
\renewcommand{\vec}{\mathbf}
\newcommand{\veci}{{\boldsymbol{\hat{\imath}}}}
\newcommand{\vecj}{{\boldsymbol{\hat{\jmath}}}}
\newcommand{\veck}{{\boldsymbol{\hat{k}}}}
\newcommand{\vecl}{\boldsymbol{\l}}
\newcommand{\uvec}[1]{\mathbf{\hat{#1}}}
\newcommand{\utan}{\mathbf{\hat{t}}}
\newcommand{\unormal}{\mathbf{\hat{n}}}
\newcommand{\ubinormal}{\mathbf{\hat{b}}}

\newcommand{\dotp}{\bullet}
\newcommand{\cross}{\boldsymbol\times}
\newcommand{\grad}{\boldsymbol\nabla}
\newcommand{\divergence}{\grad\dotp}
\newcommand{\curl}{\grad\cross}
%\DeclareMathOperator{\divergence}{divergence}
%\DeclareMathOperator{\curl}[1]{\grad\cross #1}
\newcommand{\lto}{\mathop{\longrightarrow\,}\limits}

\renewcommand{\bar}{\overline}

\colorlet{textColor}{black} 
\colorlet{background}{white}
\colorlet{penColor}{blue!50!black} % Color of a curve in a plot
\colorlet{penColor2}{red!50!black}% Color of a curve in a plot
\colorlet{penColor3}{red!50!blue} % Color of a curve in a plot
\colorlet{penColor4}{green!50!black} % Color of a curve in a plot
\colorlet{penColor5}{orange!80!black} % Color of a curve in a plot
\colorlet{penColor6}{yellow!70!black} % Color of a curve in a plot
\colorlet{fill1}{penColor!20} % Color of fill in a plot
\colorlet{fill2}{penColor2!20} % Color of fill in a plot
\colorlet{fillp}{fill1} % Color of positive area
\colorlet{filln}{penColor2!20} % Color of negative area
\colorlet{fill3}{penColor3!20} % Fill
\colorlet{fill4}{penColor4!20} % Fill
\colorlet{fill5}{penColor5!20} % Fill
\colorlet{gridColor}{gray!50} % Color of grid in a plot

\newcommand{\surfaceColor}{violet}
\newcommand{\surfaceColorTwo}{redyellow}
\newcommand{\sliceColor}{greenyellow}




\pgfmathdeclarefunction{gauss}{2}{% gives gaussian
  \pgfmathparse{1/(#2*sqrt(2*pi))*exp(-((x-#1)^2)/(2*#2^2))}%
}


%%%%%%%%%%%%%
%% Vectors
%%%%%%%%%%%%%

%% Simple horiz vectors
\renewcommand{\vector}[1]{\left\langle #1\right\rangle}


%% %% Complex Horiz Vectors with angle brackets
%% \makeatletter
%% \renewcommand{\vector}[2][ , ]{\left\langle%
%%   \def\nextitem{\def\nextitem{#1}}%
%%   \@for \el:=#2\do{\nextitem\el}\right\rangle%
%% }
%% \makeatother

%% %% Vertical Vectors
%% \def\vector#1{\begin{bmatrix}\vecListA#1,,\end{bmatrix}}
%% \def\vecListA#1,{\if,#1,\else #1\cr \expandafter \vecListA \fi}

%%%%%%%%%%%%%
%% End of vectors
%%%%%%%%%%%%%

%\newcommand{\fullwidth}{}
%\newcommand{\normalwidth}{}



%% makes a snazzy t-chart for evaluating functions
%\newenvironment{tchart}{\rowcolors{2}{}{background!90!textColor}\array}{\endarray}

%%This is to help with formatting on future title pages.
\newenvironment{sectionOutcomes}{}{} 



%% Flowchart stuff
%\tikzstyle{startstop} = [rectangle, rounded corners, minimum width=3cm, minimum height=1cm,text centered, draw=black]
%\tikzstyle{question} = [rectangle, minimum width=3cm, minimum height=1cm, text centered, draw=black]
%\tikzstyle{decision} = [trapezium, trapezium left angle=70, trapezium right angle=110, minimum width=3cm, minimum height=1cm, text centered, draw=black]
%\tikzstyle{question} = [rectangle, rounded corners, minimum width=3cm, minimum height=1cm,text centered, draw=black]
%\tikzstyle{process} = [rectangle, minimum width=3cm, minimum height=1cm, text centered, draw=black]
%\tikzstyle{decision} = [trapezium, trapezium left angle=70, trapezium right angle=110, minimum width=3cm, minimum height=1cm, text centered, draw=black]

\outcome{Approximate a slant asymptote from the graph of a function.}

\title[Break-Ground:]{Zoom Out}

\begin{document}
\begin{abstract}
Two young mathematicians discuss what curves look like when one
``zooms out.''
\end{abstract}
\maketitle

Check out this dialogue between two calculus students (based on a true
story):

\begin{dialogue}
\item[Devyn] Riley, think about this function:
  \[
  f(x) = \frac{1}{x}.
  \]
\item[Riley] Hmmm. If you plot it, the graph looks like this:
\begin{image}
\begin{tikzpicture}
	\begin{axis}[
            domain=-4:4,
            ymax=10,
            ymin=-10,
            samples=100,
            axis lines =middle, xlabel=$x$, ylabel=$y$,
            every axis y label/.style={at=(current axis.above origin),anchor=south},
            every axis x label/.style={at=(current axis.right of origin),anchor=west}
          ]
	  \addplot [very thick, penColor, smooth, domain=(-3:-0.01)] {(1)/(x)};
      \addplot [very thick, penColor, smooth, domain=(0.01:3)] {(1)/(x)};
      \addplot [textColor, dashed] plot coordinates {(0,-20) (0,20)};
 
          
        \end{axis}
\end{tikzpicture}
\end{image}
\item[Devyn] Right! What I've noticed is that if $x$ gets really big or really small, then
  our function looks like a line.
\item[Riley] Yeah!  And if the $x$ values get really close to $0$, our function also looks like a different line! 
\item[Devyn] Whoa!  You're blowin' my mind. 
\end{dialogue}

\begin{problem}
Devyn and Riley have noticed that the function $f(x) = \frac{1}{x}$ looks like a horizontal line $y \ = \answer{0}$ for $x$ values far away from $0$ and looks like the vertical line $x \ = \answer{0}$ for $x$ values near $0$.  
\end{problem}

Let's investigate why this is the case using some tables.  \\

You may need to input the symbol for infinity to answer one of the following questions.  To do so, type $\verb|infty|$ or $\verb|infinity|$ or $\verb|oo|$.  

\begin{problem}
We've noticed that $f(x) = \frac{1}{x}$ looks like a horizontal line for $x$ values far away from $0$.  Fill in the tables below to answer the following question. 
  \[
  \begin{array}{l|l}
    x      & f(x) = \frac{1}{x}     \\ \hline
    1    & \begin{prompt}\answer{1}\end{prompt}\\
    10   & \begin{prompt}\answer{0.1}\end{prompt}\\
    100  & \begin{prompt}\answer{0.01}\end{prompt}\\
    1000 & \begin{prompt}\answer{0.001}\end{prompt} \\
    10000 & \begin{prompt}\answer{0.0001}\end{prompt} \\
    100000 & \begin{prompt}\answer{0.00001}\end{prompt} \\
  \end{array}
  \qquad\text{and}\qquad
  \begin{array}{l|l}
     x      & f(x) = \frac{1}{x}     \\ \hline
    -1    & \begin{prompt}\answer{-1}\end{prompt}\\
    -10   & \begin{prompt}\answer{-0.1}\end{prompt}\\
    -100  & \begin{prompt}\answer{-0.01}\end{prompt}\\
    -1000 & \begin{prompt}\answer{-0.001}\end{prompt} \\
    -10000 & \begin{prompt}\answer{-0.0001}\end{prompt} \\
    -100000 & \begin{prompt}\answer{-0.00001}\end{prompt} \\
    \end{array}
  \]
  
\begin{itemize}
  
\item In the first table, as the $x$ values became more positive, $y = \frac{1}{x}$ \wordChoice{\choice{increased}\choice[correct]{decreased}} toward $y = \answer{0}$.  Based on this, it appears that $\displaystyle\lim_{x \to \infty} \frac{1}{x} = \answer{0}$.

\item In the second table, as the $x$ values became more negative, $y = \frac{1}{x}$ \wordChoice{\choice[correct]{increased}\choice{decreased}} toward $y = \answer{0}$.  Based on this, it appears that $\displaystyle\lim_{x \to -\infty} \frac{1}{x} = \answer{0}$.

\end{itemize}

\end{problem}

\begin{problem}
We've also noticed that $f(x) = \frac{1}{x}$ looks like a vertical line for $x$ values near $0$.  Fill in the tables below to answer the following question. 
  \[
  \begin{array}{l|l}
    x      & f(x) = \frac{1}{x}     \\ \hline
    10    & \begin{prompt}\answer{0.1}\end{prompt}\\
    1  & \begin{prompt}\answer{1}\end{prompt}\\
    \frac{1}{10}  & \begin{prompt}\answer{10}\end{prompt}\\
    \frac{1}{100} & \begin{prompt}\answer{100}\end{prompt} \\
    \frac{1}{1000} & \begin{prompt}\answer{1000}\end{prompt} \\
    \frac{1}{10000} & \begin{prompt}\answer{10000}\end{prompt} \\
  \end{array}
  \qquad\text{and}\qquad
  \begin{array}{l|l}
    x      & f(x) = \frac{1}{x}     \\ \hline
    -10    & \begin{prompt}\answer{-0.1}\end{prompt}\\
    -1  & \begin{prompt}\answer{-1}\end{prompt}\\
    -\frac{1}{10}  & \begin{prompt}\answer{-10}\end{prompt}\\
    -\frac{1}{100} & \begin{prompt}\answer{-100}\end{prompt} \\
    -\frac{1}{1000} & \begin{prompt}\answer{-1000}\end{prompt} \\
    -\frac{1}{10000} & \begin{prompt}\answer{-10000}\end{prompt} \\
  \end{array}
  \]
  
\begin{itemize}
  
\item In the first table, as the $x$ values got closer to $0$, $y = \frac{1}{x}$ rapidly \wordChoice{\choice[correct]{increased}\choice{decreased}}.  Based on this, it appears that $\displaystyle\lim_{x \to 0^+} \frac{1}{x} = \answer{\infty}$.

\item In the second table, as the $x$ values got closer to $0$, $y = \frac{1}{x}$ rapidly \wordChoice{\choice{increased}\choice[correct]{decreased}}.  Based on this, it appears that $\displaystyle\lim_{x \to 0^-} \frac{1}{x} = \answer{-\infty}$.

\end{itemize}

\end{problem}


%%% \begin{xarmaBoost}
%%   Write down at least \textbf{five} questions for this lecture. After
%%   you have your questions, label them as ``Level 1,'' ``Level 2,'' or
%%   ``Level 3'' where:
%% \begin{description}
%% \item[Level 1] Means you know the answer, or know exactly how to do
%%   this problem.
%% \item[Level 2] Means you think you know how to do the problem.
%% \item[Level 3] Means you have no idea how to do the problem.
%% \end{description}
%% \begin{freeResponse}
%% \end{freeResponse}
%% \end{xarmaBoost}



\end{document}
