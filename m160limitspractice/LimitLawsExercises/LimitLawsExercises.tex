\documentclass[handout]{ximera}
%\documentclass[10pt,handout,twocolumn,twoside,wordchoicegiven]{xercises}
%\documentclass[10pt,handout,twocolumn,twoside,wordchoicegiven]{xourse}

%\usepackage{todonotes}

\newcommand{\todo}{}

\usepackage{esint} % for \oiint
\graphicspath{
{./}
{functionsOfSeveralVariables/}
{normalVectors/}
{lagrangeMultipliers/}
{vectorFields/}
{greensTheorem/}
{shapeOfThingsToCome/}
}


\usepackage{tkz-euclide}
\tikzset{>=stealth} %% cool arrow head
\tikzset{shorten <>/.style={ shorten >=#1, shorten <=#1 } } %% allows shorter vectors

\usetikzlibrary{backgrounds} %% for boxes around graphs
\usetikzlibrary{shapes,positioning}  %% Clouds and stars
\usetikzlibrary{matrix} %% for matrix
\usepgfplotslibrary{polar} %% for polar plots
\usetkzobj{all}
\usepackage[makeroom]{cancel} %% for strike outs
%\usepackage{mathtools} %% for pretty underbrace % Breaks Ximera
\usepackage{multicol}
\usepackage{pgffor} %% required for integral for loops


%% http://tex.stackexchange.com/questions/66490/drawing-a-tikz-arc-specifying-the-center
%% Draws beach ball
\tikzset{pics/carc/.style args={#1:#2:#3}{code={\draw[pic actions] (#1:#3) arc(#1:#2:#3);}}}



\usepackage{array}
\setlength{\extrarowheight}{+.1cm}   
\newdimen\digitwidth
\settowidth\digitwidth{9}
\def\divrule#1#2{
\noalign{\moveright#1\digitwidth
\vbox{\hrule width#2\digitwidth}}}





\newcommand{\RR}{\mathbb R}
\newcommand{\R}{\mathbb R}
\newcommand{\N}{\mathbb N}
\newcommand{\Z}{\mathbb Z}

%\newcommand{\sage}{\textsf{SageMath}}


%\renewcommand{\d}{\,d\!}
\renewcommand{\d}{\mathop{}\!d}
\newcommand{\dd}[2][]{\frac{\d #1}{\d #2}}
\newcommand{\pp}[2][]{\frac{\partial #1}{\partial #2}}
\renewcommand{\l}{\ell}
\newcommand{\ddx}{\frac{d}{\d x}}

\newcommand{\zeroOverZero}{\ensuremath{\boldsymbol{\tfrac{0}{0}}}}
\newcommand{\inftyOverInfty}{\ensuremath{\boldsymbol{\tfrac{\infty}{\infty}}}}
\newcommand{\zeroOverInfty}{\ensuremath{\boldsymbol{\tfrac{0}{\infty}}}}
\newcommand{\zeroTimesInfty}{\ensuremath{\small\boldsymbol{0\cdot \infty}}}
\newcommand{\inftyMinusInfty}{\ensuremath{\small\boldsymbol{\infty - \infty}}}
\newcommand{\oneToInfty}{\ensuremath{\boldsymbol{1^\infty}}}
\newcommand{\zeroToZero}{\ensuremath{\boldsymbol{0^0}}}
\newcommand{\inftyToZero}{\ensuremath{\boldsymbol{\infty^0}}}



\newcommand{\numOverZero}{\ensuremath{\boldsymbol{\tfrac{\#}{0}}}}
\newcommand{\dfn}{\textbf}
%\newcommand{\unit}{\,\mathrm}
\newcommand{\unit}{\mathop{}\!\mathrm}
\newcommand{\eval}[1]{\bigg[ #1 \bigg]}
\newcommand{\seq}[1]{\left( #1 \right)}
\renewcommand{\epsilon}{\varepsilon}
\renewcommand{\phi}{\varphi}


\renewcommand{\iff}{\Leftrightarrow}

\DeclareMathOperator{\arccot}{arccot}
\DeclareMathOperator{\arcsec}{arcsec}
\DeclareMathOperator{\arccsc}{arccsc}
\DeclareMathOperator{\si}{Si}
\DeclareMathOperator{\proj}{\vec{proj}}
\DeclareMathOperator{\scal}{scal}
\DeclareMathOperator{\sign}{sign}


%% \newcommand{\tightoverset}[2]{% for arrow vec
%%   \mathop{#2}\limits^{\vbox to -.5ex{\kern-0.75ex\hbox{$#1$}\vss}}}
\newcommand{\arrowvec}{\overrightarrow}
%\renewcommand{\vec}[1]{\arrowvec{\mathbf{#1}}}
\renewcommand{\vec}{\mathbf}
\newcommand{\veci}{{\boldsymbol{\hat{\imath}}}}
\newcommand{\vecj}{{\boldsymbol{\hat{\jmath}}}}
\newcommand{\veck}{{\boldsymbol{\hat{k}}}}
\newcommand{\vecl}{\boldsymbol{\l}}
\newcommand{\uvec}[1]{\mathbf{\hat{#1}}}
\newcommand{\utan}{\mathbf{\hat{t}}}
\newcommand{\unormal}{\mathbf{\hat{n}}}
\newcommand{\ubinormal}{\mathbf{\hat{b}}}

\newcommand{\dotp}{\bullet}
\newcommand{\cross}{\boldsymbol\times}
\newcommand{\grad}{\boldsymbol\nabla}
\newcommand{\divergence}{\grad\dotp}
\newcommand{\curl}{\grad\cross}
%\DeclareMathOperator{\divergence}{divergence}
%\DeclareMathOperator{\curl}[1]{\grad\cross #1}
\newcommand{\lto}{\mathop{\longrightarrow\,}\limits}

\renewcommand{\bar}{\overline}

\colorlet{textColor}{black} 
\colorlet{background}{white}
\colorlet{penColor}{blue!50!black} % Color of a curve in a plot
\colorlet{penColor2}{red!50!black}% Color of a curve in a plot
\colorlet{penColor3}{red!50!blue} % Color of a curve in a plot
\colorlet{penColor4}{green!50!black} % Color of a curve in a plot
\colorlet{penColor5}{orange!80!black} % Color of a curve in a plot
\colorlet{penColor6}{yellow!70!black} % Color of a curve in a plot
\colorlet{fill1}{penColor!20} % Color of fill in a plot
\colorlet{fill2}{penColor2!20} % Color of fill in a plot
\colorlet{fillp}{fill1} % Color of positive area
\colorlet{filln}{penColor2!20} % Color of negative area
\colorlet{fill3}{penColor3!20} % Fill
\colorlet{fill4}{penColor4!20} % Fill
\colorlet{fill5}{penColor5!20} % Fill
\colorlet{gridColor}{gray!50} % Color of grid in a plot

\newcommand{\surfaceColor}{violet}
\newcommand{\surfaceColorTwo}{redyellow}
\newcommand{\sliceColor}{greenyellow}




\pgfmathdeclarefunction{gauss}{2}{% gives gaussian
  \pgfmathparse{1/(#2*sqrt(2*pi))*exp(-((x-#1)^2)/(2*#2^2))}%
}


%%%%%%%%%%%%%
%% Vectors
%%%%%%%%%%%%%

%% Simple horiz vectors
\renewcommand{\vector}[1]{\left\langle #1\right\rangle}


%% %% Complex Horiz Vectors with angle brackets
%% \makeatletter
%% \renewcommand{\vector}[2][ , ]{\left\langle%
%%   \def\nextitem{\def\nextitem{#1}}%
%%   \@for \el:=#2\do{\nextitem\el}\right\rangle%
%% }
%% \makeatother

%% %% Vertical Vectors
%% \def\vector#1{\begin{bmatrix}\vecListA#1,,\end{bmatrix}}
%% \def\vecListA#1,{\if,#1,\else #1\cr \expandafter \vecListA \fi}

%%%%%%%%%%%%%
%% End of vectors
%%%%%%%%%%%%%

%\newcommand{\fullwidth}{}
%\newcommand{\normalwidth}{}



%% makes a snazzy t-chart for evaluating functions
%\newenvironment{tchart}{\rowcolors{2}{}{background!90!textColor}\array}{\endarray}

%%This is to help with formatting on future title pages.
\newenvironment{sectionOutcomes}{}{} 



%% Flowchart stuff
%\tikzstyle{startstop} = [rectangle, rounded corners, minimum width=3cm, minimum height=1cm,text centered, draw=black]
%\tikzstyle{question} = [rectangle, minimum width=3cm, minimum height=1cm, text centered, draw=black]
%\tikzstyle{decision} = [trapezium, trapezium left angle=70, trapezium right angle=110, minimum width=3cm, minimum height=1cm, text centered, draw=black]
%\tikzstyle{question} = [rectangle, rounded corners, minimum width=3cm, minimum height=1cm,text centered, draw=black]
%\tikzstyle{process} = [rectangle, minimum width=3cm, minimum height=1cm, text centered, draw=black]
%\tikzstyle{decision} = [trapezium, trapezium left angle=70, trapezium right angle=110, minimum width=3cm, minimum height=1cm, text centered, draw=black]

\outcome{Practice Limits.}
   

\title{Limit laws exercises}

\begin{document}
\begin{abstract}
Here is an opportunity for you to practice limit laws and the Squeeze Theorem.
\end{abstract}
\maketitle

\begin{exercise}
  Can this limit be directly computed by limit laws?
  \[
  \displaystyle\lim_{x\to -3}\frac{x^2+5x+6}{x+2} 
  \]
  \begin{multipleChoice}
    \choice[correct]{yes}
    \choice{no}
  \end{multipleChoice}
  \begin{question}
    Compute:
    \[
    \displaystyle\lim_{x\to -3}\frac{x^2+5x+6}{x+2} \begin{prompt} =\answer{0}\end{prompt}
    \]
    \begin{feedback}
      Since $f(x)=\frac{x^2+5x+6}{x+2}$ is a rational function and
      $\displaystyle\lim_{x\to -3} x+2 =-1 \neq 0$, you can use the quotient law.  The limit of the numerator and denominator can then be calculated using the limit laws because the numerator and denominator are both polynomials. 
    \end{feedback}
  \end{question}
\end{exercise}

\begin{exercise}
  Can this limit be directly computed by limit laws?
  \[
  \displaystyle\lim_{x\to -3}\frac{x^2+5x+6}{x+3} 
  \]
  \begin{multipleChoice}
    \choice{yes}
    \choice[correct]{no}
    
    \begin{feedback}[correct]
      Since $f(x)=\frac{x^2+5x+6}{x+3}$ is a rational function, you may want to use the quotient law; however, $\displaystyle\lim_{x\to -3} x+3 = 0$, so you cannot use this limit law.  Because the quotient law cannot be used, this limit cannot be evaluated with the limit laws \textit{unless} we find a way to deal with the limit of the denominator going to zero.  Stay tuned for more on this in a few sections.
    \end{feedback}
    
  \end{multipleChoice}

\end{exercise}

\begin{exercise}
Can this limit be directly computed by limit laws?
  \[
  \displaystyle\lim_{x\to 0} \left(-x^6\cos\left(\frac{\pi}{x}\right)\right)
  \]
  
  \begin{multipleChoice}
      \choice{Yes}
      \choice[correct]{No}
  \end{multipleChoice}
  
\begin{exercise}
  
Right - the limit laws will not help you in this situation.  Here is a graph of this function to help you see what's going on.  Use the + button to zoom in.
  \[
   \graph{f(x)=-x^6*\cos\left(\left(\frac{\pi}{x}\right)\right)}
  \]
  
Based on the graph, you would predict that 

\[
  \displaystyle\lim_{x\to 0} \left(-x^6\cos\left(\frac{\pi}{x}\right)\right) = \answer{0}.
  \]
\begin{exercise}

\begin{hint}
If you are struggling to find the upper and lower bounds of  $\cos\left(\frac{\pi}{x}\right)$, it might help to think about the graph of $y = \cos(x)$.  What is the highest function value $\cos(x)$ will achieve?  What is the lowest? 
\end{hint}

Let's verify this prediction using the Squeeze Theorem.
 
First, you could notice that

\[ \answer{-1} \leq \cos\left(\frac{\pi}{x}\right) \leq \answer{1} \]

so that
   
\[ \answer{-x^6} \leq -x^6\cos\left(\frac{\pi}{x}\right) \leq \answer{x^6}. \]

Before moving on, graph the two functions you entered in the last two answer boxes using the Desmos app above.  Do you see how these two graphs ``squeeze" $-x^6\cos\left(\frac{\pi}{x}\right)$ near $x=0$?

Now, notice that 

\[ \displaystyle\lim_{x \to \answer{0}} \answer{-x^6} = \answer{0} = \displaystyle\lim_{x \to \answer{0}} \answer{x^6},\]

so the Squeeze Theorem allows us to conclude that 

\[ \displaystyle\lim_{x \to 0}  -x^6\cos\left(\frac{\pi}{x}\right) = \answer{0} \]

\end{exercise}
\end{exercise}
\end{exercise}

\begin{exercise}
  Can this limit be directly computed by limit laws?
  \[
  \displaystyle\lim_{x\to 42} 1337
  \]
  \begin{multipleChoice}
    \choice[correct]{yes}
    \choice{no}   
  \end{multipleChoice}
\begin{question}
    Compute:
    \[
    \displaystyle\lim_{x\to 42} 1337 \begin{prompt} =\answer{1337}\end{prompt}
    \]
  \end{question}

\end{exercise}

\begin{exercise}
  Can this limit be directly computed by limit laws?
  \[
  \displaystyle\lim_{x\to -1} \frac{1}{|x|}
  \]
  \begin{multipleChoice}
    \choice[correct]{yes}
    \choice{no}
  \end{multipleChoice}
  \begin{question}
    Compute:
    \[
    \displaystyle\lim_{x\to -1} \frac{1}{|x|} \begin{prompt} =\answer{1}\end{prompt}
    \]
    \begin{feedback}
      Since $f(x)=\frac{1}{|x|}$ is a rational function, we would like to use the quotient law.  Before doing so, we must check that the limit of the denominator is not equal to $0$.  When $x < 0, |x| = -x$, so $\displaystyle\lim_{x\to -1} |x| = \displaystyle\lim_{x\to -1} -x = -(-1) \neq 0$ and so the quotient law can be utilized. The limit of the the numerator can also be calculated because the numerator is a constant function.  
    \end{feedback}
  \end{question}
\end{exercise}

\begin{exercise}
  Can this limit be directly computed by limit laws?
  \[
  \displaystyle\lim_{x\to 0} \frac{1}{|x|}
  \]
  \begin{multipleChoice}
    \choice{yes}
    \choice[correct]{no}
    
     \begin{feedback}
      Again, because $f(x)=\frac{1}{|x|}$ is a rational function, we would like to use the quotient law.  However, $\displaystyle\lim_{x\to 0} |x| = 0$, so the limit of the denominator is $0$, and the quotient law cannot be used.  We need techniques beyond the limit laws to evaluate this limit. 
    \end{feedback}
    
  \end{multipleChoice}
\end{exercise}

\begin{exercise}

Fill in the blanks to evaluate the following limit using the limit laws. 
  \[
  \displaystyle\lim_{x\to 4} \frac{(3x-7)(\sqrt{x})}{5x}  \]
First, you could use the quotient law to conclude that 

\[
  \displaystyle\lim_{x\to 4} \frac{(3x-7)(\sqrt{x})}{5x} = \frac{\displaystyle\lim_{x\to 4} (3x-7)(\sqrt{x})}{\displaystyle\lim_{x\to 4} 5x} \]
by the \wordChoice{\choice{Constant Multiple Law}\choice{Sum/Difference Law}\choice{Product Law}\choice[correct]{Quotient Law}}, which you can use because 

\[
  \displaystyle\lim_{x\to 4} 5x \neq \answer{0}. \]
  
The numerator can be now be re-written using the \wordChoice{\choice{Constant Multiple Law}\choice{Sum/Difference Law}\choice[correct]{Product Law}\choice{Quotient Law}}: 

\[ \frac{\displaystyle\lim_{x\to 4} (3x-7)(\sqrt{x})}{\displaystyle\lim_{x\to 4} 5x} = \frac{\left(\displaystyle\lim_{x\to 4} 3x-7 \right) \left(\displaystyle\lim_{x\to 4}\sqrt{x}\right)}{\displaystyle\lim_{x\to 4} 5x} \]

Now, you can evaluate all of the limits that appear individually: 

\begin{itemize}

\item $\displaystyle\lim_{x\to 4} 3x-7 = \answer{5}$ by the \wordChoice{\choice[correct]{Constant Multiple Law}\choice[correct]{Sum/Difference Law}\choice{Product Law}\choice{Quotient Law}} and the \wordChoice{\choice[correct]{Constant Multiple Law}\choice[correct]{Sum/Difference Law}\choice{Product Law}\choice{Quotient Law}}.

\item $\displaystyle\lim_{x\to 4} \sqrt{x} = \answer{2}$

\item $\displaystyle\lim_{x \to 4} 5x = \answer{20}$

\end{itemize}

Combining all this information together, you can conclude that 

\[
  \displaystyle\lim_{x\to 4} \frac{(3x-7)(\sqrt{x})}{5x} = \answer{\frac{1}{2}}.  \]

\end{exercise}

\begin{exercise}

Suppose that $\displaystyle\lim_{x \to -4} Q(x) = 3, \displaystyle\lim_{x\to -4} R(x) = 7,$ and $\displaystyle\lim_{x\to -4} S(x) = -2$.  Using this information, evaluate the following limits. 

\begin{itemize}

\item $\displaystyle\lim_{x \to -4} \left( Q(x) -5S(x) - R(x) \right) = \answer{6}$

\item $\displaystyle\lim_{x \to -4} Q(x)\left(S^2(x) + 5 \right) = \answer{27}$

\item $\displaystyle\lim_{x \to -4} \frac{2S(x) + Q(x)}{R(x)} = \answer{\frac{-1}{7}}$

\end{itemize}

\begin{hint}

Note: $S^2(x) = (S(x))^2$

\end{hint}

\end{exercise}

\begin{exercise}

Use the fact that $\displaystyle\lim_{x \to 0} \frac{\sin x}{x} = 1$ to evaluate the following limits. 

\begin{itemize}

\item $\displaystyle\lim_{x \to 0} \frac{ \sin x}{5x} = \answer{\frac{1}{5}}$

\item $\displaystyle\lim_{x \to 0} \frac{\tan x}{x} = \answer{1}$

\end{itemize}
\end{exercise}

\begin{exercise}

Consider the statement below, and then indicate whether it is sometimes, always, or never true.

\begin{center} ``If $\displaystyle\lim_{x \to 10} \frac{f(x)}{g(x)}$ cannot be evaluated directly using the quotient law, then $\displaystyle\lim_{x \to 10} \frac{f(x)}{g(x)}$ does not exist." \end{center}

This statement is \wordChoice{\choice[correct]{sometimes}\choice{always}\choice{never}} true.

\begin{feedback}[correct]
Just because you cannot apply a limit law to evaluate a limit does not mean that the limit doesn't exist!  You will see many examples of limits that cannot be evaluated directly with limit laws but do exist in the next section. 
\end{feedback}

\end{exercise}







\end{document}
