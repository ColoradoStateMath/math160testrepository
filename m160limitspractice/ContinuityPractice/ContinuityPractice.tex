\documentclass[handout]{ximera}
%\documentclass[10pt,handout,twocolumn,twoside,wordchoicegiven]{xercises}
%\documentclass[10pt,handout,twocolumn,twoside,wordchoicegiven]{xourse}

%\author{Steven Gubkin}
%\license{Creative Commons 3.0 By-NC}
%\usepackage{todonotes}

\newcommand{\todo}{}

\usepackage{esint} % for \oiint
\graphicspath{
{./}
{functionsOfSeveralVariables/}
{normalVectors/}
{lagrangeMultipliers/}
{vectorFields/}
{greensTheorem/}
{shapeOfThingsToCome/}
}


\usepackage{tkz-euclide}
\tikzset{>=stealth} %% cool arrow head
\tikzset{shorten <>/.style={ shorten >=#1, shorten <=#1 } } %% allows shorter vectors

\usetikzlibrary{backgrounds} %% for boxes around graphs
\usetikzlibrary{shapes,positioning}  %% Clouds and stars
\usetikzlibrary{matrix} %% for matrix
\usepgfplotslibrary{polar} %% for polar plots
\usetkzobj{all}
\usepackage[makeroom]{cancel} %% for strike outs
%\usepackage{mathtools} %% for pretty underbrace % Breaks Ximera
\usepackage{multicol}
\usepackage{pgffor} %% required for integral for loops


%% http://tex.stackexchange.com/questions/66490/drawing-a-tikz-arc-specifying-the-center
%% Draws beach ball
\tikzset{pics/carc/.style args={#1:#2:#3}{code={\draw[pic actions] (#1:#3) arc(#1:#2:#3);}}}



\usepackage{array}
\setlength{\extrarowheight}{+.1cm}   
\newdimen\digitwidth
\settowidth\digitwidth{9}
\def\divrule#1#2{
\noalign{\moveright#1\digitwidth
\vbox{\hrule width#2\digitwidth}}}





\newcommand{\RR}{\mathbb R}
\newcommand{\R}{\mathbb R}
\newcommand{\N}{\mathbb N}
\newcommand{\Z}{\mathbb Z}

%\newcommand{\sage}{\textsf{SageMath}}


%\renewcommand{\d}{\,d\!}
\renewcommand{\d}{\mathop{}\!d}
\newcommand{\dd}[2][]{\frac{\d #1}{\d #2}}
\newcommand{\pp}[2][]{\frac{\partial #1}{\partial #2}}
\renewcommand{\l}{\ell}
\newcommand{\ddx}{\frac{d}{\d x}}

\newcommand{\zeroOverZero}{\ensuremath{\boldsymbol{\tfrac{0}{0}}}}
\newcommand{\inftyOverInfty}{\ensuremath{\boldsymbol{\tfrac{\infty}{\infty}}}}
\newcommand{\zeroOverInfty}{\ensuremath{\boldsymbol{\tfrac{0}{\infty}}}}
\newcommand{\zeroTimesInfty}{\ensuremath{\small\boldsymbol{0\cdot \infty}}}
\newcommand{\inftyMinusInfty}{\ensuremath{\small\boldsymbol{\infty - \infty}}}
\newcommand{\oneToInfty}{\ensuremath{\boldsymbol{1^\infty}}}
\newcommand{\zeroToZero}{\ensuremath{\boldsymbol{0^0}}}
\newcommand{\inftyToZero}{\ensuremath{\boldsymbol{\infty^0}}}



\newcommand{\numOverZero}{\ensuremath{\boldsymbol{\tfrac{\#}{0}}}}
\newcommand{\dfn}{\textbf}
%\newcommand{\unit}{\,\mathrm}
\newcommand{\unit}{\mathop{}\!\mathrm}
\newcommand{\eval}[1]{\bigg[ #1 \bigg]}
\newcommand{\seq}[1]{\left( #1 \right)}
\renewcommand{\epsilon}{\varepsilon}
\renewcommand{\phi}{\varphi}


\renewcommand{\iff}{\Leftrightarrow}

\DeclareMathOperator{\arccot}{arccot}
\DeclareMathOperator{\arcsec}{arcsec}
\DeclareMathOperator{\arccsc}{arccsc}
\DeclareMathOperator{\si}{Si}
\DeclareMathOperator{\proj}{\vec{proj}}
\DeclareMathOperator{\scal}{scal}
\DeclareMathOperator{\sign}{sign}


%% \newcommand{\tightoverset}[2]{% for arrow vec
%%   \mathop{#2}\limits^{\vbox to -.5ex{\kern-0.75ex\hbox{$#1$}\vss}}}
\newcommand{\arrowvec}{\overrightarrow}
%\renewcommand{\vec}[1]{\arrowvec{\mathbf{#1}}}
\renewcommand{\vec}{\mathbf}
\newcommand{\veci}{{\boldsymbol{\hat{\imath}}}}
\newcommand{\vecj}{{\boldsymbol{\hat{\jmath}}}}
\newcommand{\veck}{{\boldsymbol{\hat{k}}}}
\newcommand{\vecl}{\boldsymbol{\l}}
\newcommand{\uvec}[1]{\mathbf{\hat{#1}}}
\newcommand{\utan}{\mathbf{\hat{t}}}
\newcommand{\unormal}{\mathbf{\hat{n}}}
\newcommand{\ubinormal}{\mathbf{\hat{b}}}

\newcommand{\dotp}{\bullet}
\newcommand{\cross}{\boldsymbol\times}
\newcommand{\grad}{\boldsymbol\nabla}
\newcommand{\divergence}{\grad\dotp}
\newcommand{\curl}{\grad\cross}
%\DeclareMathOperator{\divergence}{divergence}
%\DeclareMathOperator{\curl}[1]{\grad\cross #1}
\newcommand{\lto}{\mathop{\longrightarrow\,}\limits}

\renewcommand{\bar}{\overline}

\colorlet{textColor}{black} 
\colorlet{background}{white}
\colorlet{penColor}{blue!50!black} % Color of a curve in a plot
\colorlet{penColor2}{red!50!black}% Color of a curve in a plot
\colorlet{penColor3}{red!50!blue} % Color of a curve in a plot
\colorlet{penColor4}{green!50!black} % Color of a curve in a plot
\colorlet{penColor5}{orange!80!black} % Color of a curve in a plot
\colorlet{penColor6}{yellow!70!black} % Color of a curve in a plot
\colorlet{fill1}{penColor!20} % Color of fill in a plot
\colorlet{fill2}{penColor2!20} % Color of fill in a plot
\colorlet{fillp}{fill1} % Color of positive area
\colorlet{filln}{penColor2!20} % Color of negative area
\colorlet{fill3}{penColor3!20} % Fill
\colorlet{fill4}{penColor4!20} % Fill
\colorlet{fill5}{penColor5!20} % Fill
\colorlet{gridColor}{gray!50} % Color of grid in a plot

\newcommand{\surfaceColor}{violet}
\newcommand{\surfaceColorTwo}{redyellow}
\newcommand{\sliceColor}{greenyellow}




\pgfmathdeclarefunction{gauss}{2}{% gives gaussian
  \pgfmathparse{1/(#2*sqrt(2*pi))*exp(-((x-#1)^2)/(2*#2^2))}%
}


%%%%%%%%%%%%%
%% Vectors
%%%%%%%%%%%%%

%% Simple horiz vectors
\renewcommand{\vector}[1]{\left\langle #1\right\rangle}


%% %% Complex Horiz Vectors with angle brackets
%% \makeatletter
%% \renewcommand{\vector}[2][ , ]{\left\langle%
%%   \def\nextitem{\def\nextitem{#1}}%
%%   \@for \el:=#2\do{\nextitem\el}\right\rangle%
%% }
%% \makeatother

%% %% Vertical Vectors
%% \def\vector#1{\begin{bmatrix}\vecListA#1,,\end{bmatrix}}
%% \def\vecListA#1,{\if,#1,\else #1\cr \expandafter \vecListA \fi}

%%%%%%%%%%%%%
%% End of vectors
%%%%%%%%%%%%%

%\newcommand{\fullwidth}{}
%\newcommand{\normalwidth}{}



%% makes a snazzy t-chart for evaluating functions
%\newenvironment{tchart}{\rowcolors{2}{}{background!90!textColor}\array}{\endarray}

%%This is to help with formatting on future title pages.
\newenvironment{sectionOutcomes}{}{} 



%% Flowchart stuff
%\tikzstyle{startstop} = [rectangle, rounded corners, minimum width=3cm, minimum height=1cm,text centered, draw=black]
%\tikzstyle{question} = [rectangle, minimum width=3cm, minimum height=1cm, text centered, draw=black]
%\tikzstyle{decision} = [trapezium, trapezium left angle=70, trapezium right angle=110, minimum width=3cm, minimum height=1cm, text centered, draw=black]
%\tikzstyle{question} = [rectangle, rounded corners, minimum width=3cm, minimum height=1cm,text centered, draw=black]
%\tikzstyle{process} = [rectangle, minimum width=3cm, minimum height=1cm, text centered, draw=black]
%\tikzstyle{decision} = [trapezium, trapezium left angle=70, trapezium right angle=110, minimum width=3cm, minimum height=1cm, text centered, draw=black]

\outcome{Practice Limits.}
   

\title{Continuity practice}

\begin{document}
\begin{abstract}
Here is an opportunity for you to practice using the definition of continuity. 
\end{abstract}
\maketitle

\begin{exercise}

Consider the function $f(x) = \frac{x+4}{x^2-16}.$  Is this function continuous on $\RR = (\infty, \infty)?$  

\begin{multipleChoice}
    \choice{Yes}
    \choice[correct]{No}
\end{multipleChoice}

\begin{exercise}
Now that you know this function is not continuous everywhere, select all of the $x$ values at which this function is not continuous. 

\begin{selectAll}
    \choice[correct]{$x = 4$}
    \choice{$x=-16$}
    \choice{$x=16$}
    \choice[correct]{$x=-4$}
    \choice{$x=0$}
\end{selectAll}

\begin{exercise}

You got it - $f$ is discontinuous at $x=-4$ and $x=-4$.  Let's use limits and function values to determine what type of discontinuity $f$ has at $x=-4$.

First, make a prediction.  Any prediction you make is correct because it's what you think currently, so take a guess!  At $x=-4$, I predict that $f$ has a(n) \wordChoice{\choice[correct]{removable discontinuity/hole}\choice[correct]{jump discontinuity}\choice[correct]{infinite discontinuity}\choice[correct]{oscillating discontinuity}}.  

Now, evaluate the following in order to make a conclusion about the type of discontinuity $f$ has at $x=-4$.  If a limit or function value does not exist, write DNE.

\begin{itemize}

\item $\displaystyle\lim_{x \to -4} f(x) = \answer{\frac{-1}{8}}$

\item $f(-4) = \answer{DNE}$

\end{itemize}

Based on this, you can conclude that at $x=-4$ $f$ has a(n) \wordChoice{\choice[correct]{removable discontinuity/hole}\choice{jump discontinuity}\choice{infinite discontinuity}\choice{oscillating discontinuity}}.  

\end{exercise}
\end{exercise}
\end{exercise}

\begin{exercise}

Consider the function $g(x) = 5x^4 - \pi x^2 - 5$  Is this function continuous on $\RR = (\infty, \infty)?$  

\begin{multipleChoice}
    \choice[correct]{Yes}
    \choice{No}
\end{multipleChoice}

\begin{feedback}[correct]

Absolutely.  $g(x)$ is a polynomial, and polynomials are continuous on $(-\infty, \infty)$.

\end{feedback}

\end{exercise}

\begin{exercise}
Consider the piece-wise function 

\[
h(x) = \begin{cases}
  3x+5  & x<0 \\
  x^2 & x \geq 0
\end{cases}
\]

Is this function continuous on $\RR = (\infty, \infty)?$  

\begin{multipleChoice}
    \choice{Yes}
    \choice[correct]{No}
\end{multipleChoice}

\begin{exercise}

You got it: if you look closely, this function is discontinuous at $x=0$.  Select the reason why this function is discontinuous at $x=0$ below.  

\begin{multipleChoice}
    \choice{$h(0)$ does not exist.}
    \choice[correct]{$\displaystyle\lim_{x \to 0} h(x)$ does not exist.}
    \choice[correct]{$h(0) \neq \displaystyle\lim_{x \to 0} h(x)$}
    
\begin{feedback}[correct]
The main reason why $h(x)$ is discontinuous at $x=0$ is because $\displaystyle\lim_{x \to 0} h(x)$ does not exist.  You could also say that $h(0) \neq \displaystyle\lim_{x \to 0} h(x)$ because $h(0)$ exists and $\displaystyle\lim_{x \to 0} h(x)$ doesn't.
\end{feedback}

\end{multipleChoice}
\end{exercise}
\end{exercise}

\begin{exercise}
Consider the piece-wise function 

\[
j(x) = \begin{cases}
  x^3  & x < 1 \\
  5 & x=1 \\
  2x-1 & x > 1
\end{cases}
\]

Is this function continuous on $\RR = (\infty, \infty)?$  

\begin{multipleChoice}
    \choice{Yes}
    \choice[correct]{No}
\end{multipleChoice}

\begin{exercise}

Good thinking: this function is discontinuous at $x=1$.  Select the reason why this function is discontinuous at $x=1$ below.

\begin{multipleChoice}
    \choice{$j(1)$ does not exist.}
    \choice{$\displaystyle\lim_{x \to 1} j(x)$ does not exist.}
    \choice[correct]{$j(1) \neq \displaystyle\lim_{x \to 1} j(x)$}
\end{multipleChoice}

\begin{exercise}

Let's change $j(x)$ just slightly: 

\[
j(x) = \begin{cases}
  x^3  & x < 1 \\
  1 & x=1 \\ 
  2x-1 & x > 1
\end{cases}
\]

With this change, is this $j(x)$ continuous on $\RR = (\infty, \infty)?$  

\begin{multipleChoice}
    \choice[correct]{Yes}
    \choice{No}
    
\begin{feedback}[correct]
You'll notice that changing a single number resulted in $j(x)$ being continuous on $\RR$.  The small details really do matter!   
\end{feedback}
\end{multipleChoice}

\end{exercise}
\end{exercise}
\end{exercise}

\begin{exercise}

Consider the statement below, and then indicate whether it is sometimes, always, or never true.

\begin{center} ``If $f$ is continuous at $x=-2$, then $\displaystyle\lim_{x\to -2} f(x)$ exists." \end{center}

This statement is \wordChoice{\choice{sometimes}\choice[correct]{always}\choice{never}} true.

\begin{hint}

You may want to review the definition of continuity on \href{https://ximera.osu.edu/math160fa17/m160exam1content/limitLaws/digInContinuity}{this card} if you are struggling with this question. 

\end{hint}

\end{exercise}

\begin{exercise}

Consider the statement below, and then indicate whether it is sometimes, always, or never true.

\begin{center} ``If $\displaystyle\lim_{x\to -2} f(x)$ exists, then $f$ is continuous at $x=-2$." \end{center}

This statement is \wordChoice{\choice[correct]{sometimes}\choice{always}\choice{never}} true.

\begin{feedback}[correct]

Now that you know this statement is sometimes true, try to draw an example of a graph for which it is true and an example of a graph for which it is false.  

\end{feedback}

\end{exercise}

\begin{exercise}

Consider the statement below, and then indicate whether it is sometimes, always, or never true.

\begin{center} ``If $f$ is a continuous function on $(-5, 5)$, $f(-5) = -3$, and $f(5) = 3$, then there is an $x$-value $c$ in $(-5,5)$ such that $f(c) = 0$." \end{center}

This statement is \wordChoice{\choice[correct]{sometimes}\choice{always}\choice{never}} true.

\begin{feedback}[correct]

At first glance, this seems like an example of the Intermediate Value Theorem which would \textit{always} be true; however, there's a small (but important) difference between this statement and the IVT.  Can you spot it?

\end{feedback}

\end{exercise}


\end{document}
