\documentclass[handout]{ximera}
%\documentclass[10pt,handout,twocolumn,twoside,wordchoicegiven]{xercises}
%\documentclass[10pt,handout,twocolumn,twoside,wordchoicegiven]{xourse}

%\usepackage{todonotes}

\newcommand{\todo}{}

\usepackage{esint} % for \oiint
\graphicspath{
{./}
{functionsOfSeveralVariables/}
{normalVectors/}
{lagrangeMultipliers/}
{vectorFields/}
{greensTheorem/}
{shapeOfThingsToCome/}
}


\usepackage{tkz-euclide}
\tikzset{>=stealth} %% cool arrow head
\tikzset{shorten <>/.style={ shorten >=#1, shorten <=#1 } } %% allows shorter vectors

\usetikzlibrary{backgrounds} %% for boxes around graphs
\usetikzlibrary{shapes,positioning}  %% Clouds and stars
\usetikzlibrary{matrix} %% for matrix
\usepgfplotslibrary{polar} %% for polar plots
\usetkzobj{all}
\usepackage[makeroom]{cancel} %% for strike outs
%\usepackage{mathtools} %% for pretty underbrace % Breaks Ximera
\usepackage{multicol}
\usepackage{pgffor} %% required for integral for loops


%% http://tex.stackexchange.com/questions/66490/drawing-a-tikz-arc-specifying-the-center
%% Draws beach ball
\tikzset{pics/carc/.style args={#1:#2:#3}{code={\draw[pic actions] (#1:#3) arc(#1:#2:#3);}}}



\usepackage{array}
\setlength{\extrarowheight}{+.1cm}   
\newdimen\digitwidth
\settowidth\digitwidth{9}
\def\divrule#1#2{
\noalign{\moveright#1\digitwidth
\vbox{\hrule width#2\digitwidth}}}





\newcommand{\RR}{\mathbb R}
\newcommand{\R}{\mathbb R}
\newcommand{\N}{\mathbb N}
\newcommand{\Z}{\mathbb Z}

%\newcommand{\sage}{\textsf{SageMath}}


%\renewcommand{\d}{\,d\!}
\renewcommand{\d}{\mathop{}\!d}
\newcommand{\dd}[2][]{\frac{\d #1}{\d #2}}
\newcommand{\pp}[2][]{\frac{\partial #1}{\partial #2}}
\renewcommand{\l}{\ell}
\newcommand{\ddx}{\frac{d}{\d x}}

\newcommand{\zeroOverZero}{\ensuremath{\boldsymbol{\tfrac{0}{0}}}}
\newcommand{\inftyOverInfty}{\ensuremath{\boldsymbol{\tfrac{\infty}{\infty}}}}
\newcommand{\zeroOverInfty}{\ensuremath{\boldsymbol{\tfrac{0}{\infty}}}}
\newcommand{\zeroTimesInfty}{\ensuremath{\small\boldsymbol{0\cdot \infty}}}
\newcommand{\inftyMinusInfty}{\ensuremath{\small\boldsymbol{\infty - \infty}}}
\newcommand{\oneToInfty}{\ensuremath{\boldsymbol{1^\infty}}}
\newcommand{\zeroToZero}{\ensuremath{\boldsymbol{0^0}}}
\newcommand{\inftyToZero}{\ensuremath{\boldsymbol{\infty^0}}}



\newcommand{\numOverZero}{\ensuremath{\boldsymbol{\tfrac{\#}{0}}}}
\newcommand{\dfn}{\textbf}
%\newcommand{\unit}{\,\mathrm}
\newcommand{\unit}{\mathop{}\!\mathrm}
\newcommand{\eval}[1]{\bigg[ #1 \bigg]}
\newcommand{\seq}[1]{\left( #1 \right)}
\renewcommand{\epsilon}{\varepsilon}
\renewcommand{\phi}{\varphi}


\renewcommand{\iff}{\Leftrightarrow}

\DeclareMathOperator{\arccot}{arccot}
\DeclareMathOperator{\arcsec}{arcsec}
\DeclareMathOperator{\arccsc}{arccsc}
\DeclareMathOperator{\si}{Si}
\DeclareMathOperator{\proj}{\vec{proj}}
\DeclareMathOperator{\scal}{scal}
\DeclareMathOperator{\sign}{sign}


%% \newcommand{\tightoverset}[2]{% for arrow vec
%%   \mathop{#2}\limits^{\vbox to -.5ex{\kern-0.75ex\hbox{$#1$}\vss}}}
\newcommand{\arrowvec}{\overrightarrow}
%\renewcommand{\vec}[1]{\arrowvec{\mathbf{#1}}}
\renewcommand{\vec}{\mathbf}
\newcommand{\veci}{{\boldsymbol{\hat{\imath}}}}
\newcommand{\vecj}{{\boldsymbol{\hat{\jmath}}}}
\newcommand{\veck}{{\boldsymbol{\hat{k}}}}
\newcommand{\vecl}{\boldsymbol{\l}}
\newcommand{\uvec}[1]{\mathbf{\hat{#1}}}
\newcommand{\utan}{\mathbf{\hat{t}}}
\newcommand{\unormal}{\mathbf{\hat{n}}}
\newcommand{\ubinormal}{\mathbf{\hat{b}}}

\newcommand{\dotp}{\bullet}
\newcommand{\cross}{\boldsymbol\times}
\newcommand{\grad}{\boldsymbol\nabla}
\newcommand{\divergence}{\grad\dotp}
\newcommand{\curl}{\grad\cross}
%\DeclareMathOperator{\divergence}{divergence}
%\DeclareMathOperator{\curl}[1]{\grad\cross #1}
\newcommand{\lto}{\mathop{\longrightarrow\,}\limits}

\renewcommand{\bar}{\overline}

\colorlet{textColor}{black} 
\colorlet{background}{white}
\colorlet{penColor}{blue!50!black} % Color of a curve in a plot
\colorlet{penColor2}{red!50!black}% Color of a curve in a plot
\colorlet{penColor3}{red!50!blue} % Color of a curve in a plot
\colorlet{penColor4}{green!50!black} % Color of a curve in a plot
\colorlet{penColor5}{orange!80!black} % Color of a curve in a plot
\colorlet{penColor6}{yellow!70!black} % Color of a curve in a plot
\colorlet{fill1}{penColor!20} % Color of fill in a plot
\colorlet{fill2}{penColor2!20} % Color of fill in a plot
\colorlet{fillp}{fill1} % Color of positive area
\colorlet{filln}{penColor2!20} % Color of negative area
\colorlet{fill3}{penColor3!20} % Fill
\colorlet{fill4}{penColor4!20} % Fill
\colorlet{fill5}{penColor5!20} % Fill
\colorlet{gridColor}{gray!50} % Color of grid in a plot

\newcommand{\surfaceColor}{violet}
\newcommand{\surfaceColorTwo}{redyellow}
\newcommand{\sliceColor}{greenyellow}




\pgfmathdeclarefunction{gauss}{2}{% gives gaussian
  \pgfmathparse{1/(#2*sqrt(2*pi))*exp(-((x-#1)^2)/(2*#2^2))}%
}


%%%%%%%%%%%%%
%% Vectors
%%%%%%%%%%%%%

%% Simple horiz vectors
\renewcommand{\vector}[1]{\left\langle #1\right\rangle}


%% %% Complex Horiz Vectors with angle brackets
%% \makeatletter
%% \renewcommand{\vector}[2][ , ]{\left\langle%
%%   \def\nextitem{\def\nextitem{#1}}%
%%   \@for \el:=#2\do{\nextitem\el}\right\rangle%
%% }
%% \makeatother

%% %% Vertical Vectors
%% \def\vector#1{\begin{bmatrix}\vecListA#1,,\end{bmatrix}}
%% \def\vecListA#1,{\if,#1,\else #1\cr \expandafter \vecListA \fi}

%%%%%%%%%%%%%
%% End of vectors
%%%%%%%%%%%%%

%\newcommand{\fullwidth}{}
%\newcommand{\normalwidth}{}



%% makes a snazzy t-chart for evaluating functions
%\newenvironment{tchart}{\rowcolors{2}{}{background!90!textColor}\array}{\endarray}

%%This is to help with formatting on future title pages.
\newenvironment{sectionOutcomes}{}{} 



%% Flowchart stuff
%\tikzstyle{startstop} = [rectangle, rounded corners, minimum width=3cm, minimum height=1cm,text centered, draw=black]
%\tikzstyle{question} = [rectangle, minimum width=3cm, minimum height=1cm, text centered, draw=black]
%\tikzstyle{decision} = [trapezium, trapezium left angle=70, trapezium right angle=110, minimum width=3cm, minimum height=1cm, text centered, draw=black]
%\tikzstyle{question} = [rectangle, rounded corners, minimum width=3cm, minimum height=1cm,text centered, draw=black]
%\tikzstyle{process} = [rectangle, minimum width=3cm, minimum height=1cm, text centered, draw=black]
%\tikzstyle{decision} = [trapezium, trapezium left angle=70, trapezium right angle=110, minimum width=3cm, minimum height=1cm, text centered, draw=black]

\outcome{Practice Limits.}
   

\title{Limits with infinity exercises}

\begin{document}
\begin{abstract}
Here is an opportunity for you to practice evaluating limits that involve infinity.  
\end{abstract}
\maketitle

You may need to input the symbol for infinity to answer one or more of the following questions.  To do so, type $\verb|infty|$ or $\verb|infinity|$ or $\verb|oo|$. 

%% Vertical Asymptote 1
\begin{exercise}
Your goal in this exercise is to evaluate $\displaystyle\lim_{x \to -3} \frac{1}{(x+3)^3}$ (if it exists).  First, consider the corresponding one-sided limits:

$$\lim_{x \to -3^-} \frac{1}{(x+3)^3} \text{ and } \lim_{x \to -3^+} \frac{1}{(x+3)^3}.$$

As $x$ approaches $-3$ from the left:  

\begin{itemize}

\item The numerator approaches $\answer{1}$, which is a \wordChoice{\choice[correct]{positive}\choice{negative}} number. 

\item $(x+3)^3$ is \wordChoice{\choice{positive}\choice[correct]{negative}} and approaches $\answer{0}.$ 

\end{itemize}

Therefore, 

 \[ \lim_{x \to -3^-} \frac{1}{(x+3)^3} = \answer{-\infty}.\]
 
As $x$ approaches $-3$ from the right: 

\begin{itemize}

\item The numerator approaches $\answer{1}$, which is a \wordChoice{\choice[correct]{positive}\choice{negative}} number. 

\item $(x+3)^3$ is \wordChoice{\choice[correct]{positive}\choice{negative}} and approaches $\answer{0}.$ 

\end{itemize}

Therefore, 

 \[ \lim_{x \to -3^+} \frac{1}{(x+3)^3} = \answer{\infty}. \]

Putting all of this information together, you can conclude that

\[ \lim_{x \to -3} \frac{1}{(x+3)^3} = \answer{DNE}. \]

\begin{exercise}

From the above limit calculations, you can say that $\frac{1}{(x+3)^3}$ \wordChoice{\choice[correct]{does}\choice{does not}} have a \wordChoice{\choice{horizontal}\choice[correct]{vertical}} asymptote at $x =-3$. 

\end{exercise}

\end{exercise}

%% Vertical Asymptote 2
\begin{exercise}
Your goal in this exercise is to evaluate $\displaystyle\lim_{x \to 7} \frac{-3x+2}{(x-7)^6}$ (if it exists).  First, consider the corresponding one-sided limits:

$$\displaystyle\lim_{x \to 7^-} \frac{-3x+2}{(x-7)^6} \text{ and } \displaystyle\lim_{x \to 7^+} \frac{-3x+2}{(x-7)^6}.$$

As $x$ approaches $7$ from the left:  

\begin{itemize}

\item The function $-3x+2$ approaches $\answer{-19}$, which is a \wordChoice{\choice{positive}\choice[correct]{negative}} number. 

\item $(x-7)^6$ is \wordChoice{\choice[correct]{positive}\choice{negative}} and approaches $\answer{0}$.

\end{itemize}

Therefore, 

 \[ \displaystyle\lim_{x \to 7^-} \frac{-3x+2}{(x-7)^6} = \answer{-\infty}.\]
 
As $x$ approaches $7$ from the right: 

\begin{itemize}

\item The function $-3x+2$ approaches $\answer{-19}$, which is a \wordChoice{\choice{positive}\choice[correct]{negative}} number. 

\item $(x-7)^6$ is \wordChoice{\choice[correct]{positive}\choice{negative}} and approaches $\answer{0}.$

\end{itemize}

Therefore, 

 \[ \displaystyle\lim_{x \to 7^+} \frac{-3x+2}{(x-7)^6} = \answer{-\infty}. \]

Putting all of this information together, you can conclude that

\[ \displaystyle\lim_{x \to 7} \frac{-3x+2}{(x-7)^6} = \answer{-\infty}. \]

\begin{exercise}

From the above limit calculations, you can say that $\frac{-3x+2}{(x-7)^3}$ \wordChoice{\choice[correct]{does}\choice{does not}} have a \wordChoice{\choice{horizontal}\choice[correct]{vertical}} asymptote at $x =7$. 

\end{exercise}

\end{exercise}

%% Vertical Asymptote 3
\begin{exercise}

Consider the function $f(x) = \frac{x^2+x-6}{x^2-3x+2}$.  The two $x$-values at which $f(x)$ may have a vertical asymptote are

\begin{hint}
On this problem, it will help to factor the numerator and denominator of $f(x)$.

\end{hint}

\begin{multipleChoice}
    \choice{$x = -2, 6$}
    \choice[correct]{$x = 1, 2$}
    \choice{$x = -3, 1$}
    \choice{$x = 2, -3$}
    
    \begin{feedback}[correct]
    Good thinking!  For a rational function like $f(x)$, there could be an asymptote at all the $x$ values where $f$ is undefined.  The only way for $f$ to be undefined is if the denominator of $f$ is 0.  Since $x^2 -3x + 2 = (x-1)(x-2)$, the $x$ values that make the denominator 0 are $x = 1, 2$, so these are the potential values at which $f$ may have a vertical asymptote. 
    \end{feedback}
\end{multipleChoice}
    
\begin{exercise}
Now, determine if $x = 1$ and $x = 2$ are asymptotes of $f$ or not. 

As $x$ approaches $1$ from the left:  

\begin{itemize}

\item The function $x^2 + x -6$ approaches $\answer{-4}$, which is a \wordChoice{\choice{positive}\choice[correct]{negative}} number. 

\item $x^2 - 3x + 2$ is \wordChoice{\choice[correct]{positive}\choice{negative}} and approaches $\answer{0}$.

\end{itemize}

Therefore, 

\[ \displaystyle\lim_{x \to 1^-} \frac{x^2+x-6}{x^2-3x+2} = \answer{-\infty} \]

and based on this limit information, you can conclude that

\begin{multipleChoice}
    \choice[correct]{$x = 1$ is a vertical asymptote of $f$.}
    \choice{$x = 1$ is not a vertical asymptote of $f$.}
    \choice{$x = 1$ may or may not be a vertical asymptete of $f$.  There isn't enough information to decide yet.}
\end{multipleChoice}

\begin{exercise}
Alright, now you know that $x=1$ is a vertical asymptote of $f$.  What about $x=2$?  Well, 

\[ \displaystyle\lim_{x \to 2^-} \frac{x^2+x-6}{x^2-3x+2} = \answer{5} \]

and based on this limit information, you can conclude that

\begin{multipleChoice}
    \choice{$x = 2$ is a vertical asymptote of $f$.}
    \choice{$x = 2$ is not a vertical asymptote of $f$.}
    \choice[correct]{$x = 2$ may or may not be a vertical asymptote of $f$.  There isn't enough information to decide yet.}
\end{multipleChoice}

\begin{exercise}
Because the limit information we have so far isn't enough to make any conclusions, evaluate the other one-sided limit: 

\[ \displaystyle\lim_{x \to 2^+} \frac{x^2+x-6}{x^2-3x+2} = \answer{5}. \]

\begin{exercise}

Now, from all information you have obtained about the limit of $f(x)$ at $x=2$, you can conclude that 

\begin{multipleChoice}
    \choice{$x = 2$ is a vertical asymptote of $f$.}
    \choice[correct]{$x = 2$ is not a vertical asymptote of $f$.}
    \choice{$x = 2$ may or may not be a vertical asymptote of $f$.  There isn't enough information to decide yet.}
    
    \begin{feedback}[correct]
    It turns out that $\displaystyle\lim_{x \to 2} \frac{x^2+x-6}{x^2-3x+2} = 5$ while $f(2)$ does not exist.  This indicates that there is a hole at $x=2$ and not a vertical asymptote!  In fact, if you factor $f(x)$, you'll realize that a factor of $(x-2)$ appears precisely once in the numerator and denominator.  Because of this, when you evaluate $\displaystyle\lim_{x \to 2} \frac{x^2+x-6}{x^2-3x+2}$, the $(x-2)$'s will cancel out, and you'll find that the limit exists.  This kind of 'cancellation set-up' is a tell-tale sign that the function at hand has a hole at $x=2$ rather than a vertical asymptote.  
    \end{feedback}
    
\end{multipleChoice}

\end{exercise}

\end{exercise}

\end{exercise}

\end{exercise}

\end{exercise}


\begin{exercise}

Evaluate the limit. 

\[ \displaystyle\lim_{x \to \infty} \frac{3x +7 - x^5}{x^4 + 8x^2 -3} = \answer{-\infty} \]

\begin{exercise}

To evaluate the above limit, you could multiply the numerator and denominator of $\frac{3x +7 - x^5}{x^4 + 8x^2 -3}$ by $\cfrac{1}{x^{\answer{4}}}.$  (Type in the \textbf{smallest} exponent that would work)

\end{exercise}

\end{exercise}

\begin{exercise}

Evaluate the limit. 

\[ \displaystyle\lim_{x \to -\infty} \frac{5x^6 + 3.5x^3 -70x^2 + 2\pi}{3x - \frac{1}{3}x^3+ 6x^6} = \answer{\frac{5}{6}} \]

\begin{exercise}

To evaluate the above limit, you could multiply the numerator and denominator of $\frac{5x^6 + 3.5x^3 -70x^2 + 2\pi}{3x - \frac{1}{3}x^3+ 6x^6}$ by $\cfrac{1}{x^{\answer{6}}}.$ (Type in the \textbf{smallest} exponent that would work)

\end{exercise}

\end{exercise}

\begin{exercise}

Evaluate the limits. 

\[ \displaystyle\lim_{x \to \infty} \frac{x - 2x^2 + 3x^3 -4x^4}{7x+5-x^5} = \answer{0} \]

\[ \displaystyle\lim_{x \to -\infty} \frac{x - 2x^2 + 3x^3 -4x^4}{7x+5-x^5} = \answer{0} \]

\begin{exercise}

To evaluate the above limits, you could multiply the numerator and denominator of $\frac{x - 2x^2 + 3x^3 -4x^4}{7x+5-x^5}$ by $\cfrac{1}{x^{\answer{5}}}.$ (Type in the \textbf{smallest} exponent that would work)

\end{exercise}

\end{exercise}

\begin{exercise}

Evaluate the limit. 

\[ \displaystyle\lim_{x \to \infty} \frac{\sqrt{2x^2+5}}{-3x} = \answer{-\frac{\sqrt{2}}{3}} \]

\begin{exercise}

Based on this limit, you can conclude that $\frac{\sqrt{2x^2+5}}{-3x}$ has a horizontal asymptote of $\answer{y} \ = \answer{-\frac{\sqrt{2}}{3}}$.

\end{exercise}

\end{exercise}

\begin{exercise}

The function $f(x) = \frac{-1 + 2|x|}{x}$ has a vertical asymptote of $\answer{x} \ = \answer{0}$.  This function also has two vertical asymptotes: One below the $x$-axis, $\answer{y} \ = \answer{-2}$, and one above the $x$-axis, $\answer{y} \ = \answer{2}$.  

\begin{hint}

Use the limit definition of vertical and horizontal asymptote in order to answer this question.  (Yes, that means you'll have to calculate some limits!)

\end{hint}

\end{exercise}

\begin{exercise}

A function can have a maximum of $\answer{2}$ horizontal asymptotes.  

\begin{feedback}[correct]

Could you explain to a fellow classmate why a function can have no more than 2 vertical asymptotes using the limit definition of a horizontal asymptote?  

\end{feedback}

\end{exercise}

\begin{exercise}

Consider the statement below, and then indicate whether it is sometimes, always, or never true.

\begin{center} ``A function, $f(x)$, can have both a vertical and a horizontal asymptote of $x=-3$." \end{center}

This statement is \wordChoice{\choice{sometimes}\choice{always}\choice[correct]{never}} true.

\end{exercise}





\end{document}
