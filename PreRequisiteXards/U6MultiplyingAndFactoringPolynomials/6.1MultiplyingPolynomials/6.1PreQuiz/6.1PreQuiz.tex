\documentclass{ximera}

%\usepackage{todonotes}

\newcommand{\todo}{}

\usepackage{esint} % for \oiint
\graphicspath{
{./}
{functionsOfSeveralVariables/}
{normalVectors/}
{lagrangeMultipliers/}
{vectorFields/}
{greensTheorem/}
{shapeOfThingsToCome/}
}


\usepackage{tkz-euclide}
\tikzset{>=stealth} %% cool arrow head
\tikzset{shorten <>/.style={ shorten >=#1, shorten <=#1 } } %% allows shorter vectors

\usetikzlibrary{backgrounds} %% for boxes around graphs
\usetikzlibrary{shapes,positioning}  %% Clouds and stars
\usetikzlibrary{matrix} %% for matrix
\usepgfplotslibrary{polar} %% for polar plots
\usetkzobj{all}
\usepackage[makeroom]{cancel} %% for strike outs
%\usepackage{mathtools} %% for pretty underbrace % Breaks Ximera
\usepackage{multicol}
\usepackage{pgffor} %% required for integral for loops


%% http://tex.stackexchange.com/questions/66490/drawing-a-tikz-arc-specifying-the-center
%% Draws beach ball
\tikzset{pics/carc/.style args={#1:#2:#3}{code={\draw[pic actions] (#1:#3) arc(#1:#2:#3);}}}



\usepackage{array}
\setlength{\extrarowheight}{+.1cm}   
\newdimen\digitwidth
\settowidth\digitwidth{9}
\def\divrule#1#2{
\noalign{\moveright#1\digitwidth
\vbox{\hrule width#2\digitwidth}}}





\newcommand{\RR}{\mathbb R}
\newcommand{\R}{\mathbb R}
\newcommand{\N}{\mathbb N}
\newcommand{\Z}{\mathbb Z}

%\newcommand{\sage}{\textsf{SageMath}}


%\renewcommand{\d}{\,d\!}
\renewcommand{\d}{\mathop{}\!d}
\newcommand{\dd}[2][]{\frac{\d #1}{\d #2}}
\newcommand{\pp}[2][]{\frac{\partial #1}{\partial #2}}
\renewcommand{\l}{\ell}
\newcommand{\ddx}{\frac{d}{\d x}}

\newcommand{\zeroOverZero}{\ensuremath{\boldsymbol{\tfrac{0}{0}}}}
\newcommand{\inftyOverInfty}{\ensuremath{\boldsymbol{\tfrac{\infty}{\infty}}}}
\newcommand{\zeroOverInfty}{\ensuremath{\boldsymbol{\tfrac{0}{\infty}}}}
\newcommand{\zeroTimesInfty}{\ensuremath{\small\boldsymbol{0\cdot \infty}}}
\newcommand{\inftyMinusInfty}{\ensuremath{\small\boldsymbol{\infty - \infty}}}
\newcommand{\oneToInfty}{\ensuremath{\boldsymbol{1^\infty}}}
\newcommand{\zeroToZero}{\ensuremath{\boldsymbol{0^0}}}
\newcommand{\inftyToZero}{\ensuremath{\boldsymbol{\infty^0}}}



\newcommand{\numOverZero}{\ensuremath{\boldsymbol{\tfrac{\#}{0}}}}
\newcommand{\dfn}{\textbf}
%\newcommand{\unit}{\,\mathrm}
\newcommand{\unit}{\mathop{}\!\mathrm}
\newcommand{\eval}[1]{\bigg[ #1 \bigg]}
\newcommand{\seq}[1]{\left( #1 \right)}
\renewcommand{\epsilon}{\varepsilon}
\renewcommand{\phi}{\varphi}


\renewcommand{\iff}{\Leftrightarrow}

\DeclareMathOperator{\arccot}{arccot}
\DeclareMathOperator{\arcsec}{arcsec}
\DeclareMathOperator{\arccsc}{arccsc}
\DeclareMathOperator{\si}{Si}
\DeclareMathOperator{\proj}{\vec{proj}}
\DeclareMathOperator{\scal}{scal}
\DeclareMathOperator{\sign}{sign}


%% \newcommand{\tightoverset}[2]{% for arrow vec
%%   \mathop{#2}\limits^{\vbox to -.5ex{\kern-0.75ex\hbox{$#1$}\vss}}}
\newcommand{\arrowvec}{\overrightarrow}
%\renewcommand{\vec}[1]{\arrowvec{\mathbf{#1}}}
\renewcommand{\vec}{\mathbf}
\newcommand{\veci}{{\boldsymbol{\hat{\imath}}}}
\newcommand{\vecj}{{\boldsymbol{\hat{\jmath}}}}
\newcommand{\veck}{{\boldsymbol{\hat{k}}}}
\newcommand{\vecl}{\boldsymbol{\l}}
\newcommand{\uvec}[1]{\mathbf{\hat{#1}}}
\newcommand{\utan}{\mathbf{\hat{t}}}
\newcommand{\unormal}{\mathbf{\hat{n}}}
\newcommand{\ubinormal}{\mathbf{\hat{b}}}

\newcommand{\dotp}{\bullet}
\newcommand{\cross}{\boldsymbol\times}
\newcommand{\grad}{\boldsymbol\nabla}
\newcommand{\divergence}{\grad\dotp}
\newcommand{\curl}{\grad\cross}
%\DeclareMathOperator{\divergence}{divergence}
%\DeclareMathOperator{\curl}[1]{\grad\cross #1}
\newcommand{\lto}{\mathop{\longrightarrow\,}\limits}

\renewcommand{\bar}{\overline}

\colorlet{textColor}{black} 
\colorlet{background}{white}
\colorlet{penColor}{blue!50!black} % Color of a curve in a plot
\colorlet{penColor2}{red!50!black}% Color of a curve in a plot
\colorlet{penColor3}{red!50!blue} % Color of a curve in a plot
\colorlet{penColor4}{green!50!black} % Color of a curve in a plot
\colorlet{penColor5}{orange!80!black} % Color of a curve in a plot
\colorlet{penColor6}{yellow!70!black} % Color of a curve in a plot
\colorlet{fill1}{penColor!20} % Color of fill in a plot
\colorlet{fill2}{penColor2!20} % Color of fill in a plot
\colorlet{fillp}{fill1} % Color of positive area
\colorlet{filln}{penColor2!20} % Color of negative area
\colorlet{fill3}{penColor3!20} % Fill
\colorlet{fill4}{penColor4!20} % Fill
\colorlet{fill5}{penColor5!20} % Fill
\colorlet{gridColor}{gray!50} % Color of grid in a plot

\newcommand{\surfaceColor}{violet}
\newcommand{\surfaceColorTwo}{redyellow}
\newcommand{\sliceColor}{greenyellow}




\pgfmathdeclarefunction{gauss}{2}{% gives gaussian
  \pgfmathparse{1/(#2*sqrt(2*pi))*exp(-((x-#1)^2)/(2*#2^2))}%
}


%%%%%%%%%%%%%
%% Vectors
%%%%%%%%%%%%%

%% Simple horiz vectors
\renewcommand{\vector}[1]{\left\langle #1\right\rangle}


%% %% Complex Horiz Vectors with angle brackets
%% \makeatletter
%% \renewcommand{\vector}[2][ , ]{\left\langle%
%%   \def\nextitem{\def\nextitem{#1}}%
%%   \@for \el:=#2\do{\nextitem\el}\right\rangle%
%% }
%% \makeatother

%% %% Vertical Vectors
%% \def\vector#1{\begin{bmatrix}\vecListA#1,,\end{bmatrix}}
%% \def\vecListA#1,{\if,#1,\else #1\cr \expandafter \vecListA \fi}

%%%%%%%%%%%%%
%% End of vectors
%%%%%%%%%%%%%

%\newcommand{\fullwidth}{}
%\newcommand{\normalwidth}{}



%% makes a snazzy t-chart for evaluating functions
%\newenvironment{tchart}{\rowcolors{2}{}{background!90!textColor}\array}{\endarray}

%%This is to help with formatting on future title pages.
\newenvironment{sectionOutcomes}{}{} 



%% Flowchart stuff
%\tikzstyle{startstop} = [rectangle, rounded corners, minimum width=3cm, minimum height=1cm,text centered, draw=black]
%\tikzstyle{question} = [rectangle, minimum width=3cm, minimum height=1cm, text centered, draw=black]
%\tikzstyle{decision} = [trapezium, trapezium left angle=70, trapezium right angle=110, minimum width=3cm, minimum height=1cm, text centered, draw=black]
%\tikzstyle{question} = [rectangle, rounded corners, minimum width=3cm, minimum height=1cm,text centered, draw=black]
%\tikzstyle{process} = [rectangle, minimum width=3cm, minimum height=1cm, text centered, draw=black]
%\tikzstyle{decision} = [trapezium, trapezium left angle=70, trapezium right angle=110, minimum width=3cm, minimum height=1cm, text centered, draw=black]

\title{Pre-Quiz}

\begin{document}
\begin{abstract}
Answer the following questions honestly to assess your own abilities. This pre-quiz will not be graded. It can be used for you to determine your own strengths and weaknesses.
\end{abstract}
\maketitle

\begin{question} 
    Could you factor the following polynomial? 
    $$2x^2 - 4$$
    
  \begin{multipleChoice}
      \choice[correct]{Yes, I could factor this polynomial.}
      \choice[correct]{Yes, but I would need to review some resources first.}
      \choice[correct]{No, I would need to learn/relearn how to factor.}
      \choice[correct]{I don't know.}
      \choice[correct]{I don't want to answer this question right now.}
  \end{multipleChoice}
\end{question}

\begin{question} 
    Could you factor this polynomial?
    $$x^2 + 9x -36$$
  \begin{multipleChoice}
      \choice[correct]{Yes, I could factor this polynomial.}
      \choice[correct]{Yes, but I would need to review some resources first.}
      \choice[correct]{No, I would need to learn/relearn how to factor.}
      \choice[correct]{I don't know.}
      \choice[correct]{I don't want to answer this question right now.}
  \end{multipleChoice}
\end{question}

\begin{question} 
    Could you find the solutions to this equality?
    $$3x^2 -2 = 5x$$
  \begin{multipleChoice}
      \choice[correct]{Yes, I could find the solutions to this equality.}
      \choice[correct]{Yes, but I would need to review some resources first.}
      \choice[correct]{No, I would need to learn/relearn how to do this.}
      \choice[correct]{I don't know.}
      \choice[correct]{I don't want to answer this right now.}
  \end{multipleChoice}
\end{question}

\begin{question} 
    Could you determine if the following expressions are equal?
    $$x^2 - 9 \text{ and } (x-3)^2$$
  \begin{multipleChoice}
      \choice[correct]{Yes, I could determine whether these expressions are equal.}
      \choice[correct]{Yes I could, but I would need some resources to determine if they are equal.}
      \choice[correct]{No, I would need to learn/relearn how to do this.}
      \choice[correct]{I don't know.}
      \choice[correct]{I don't want to analyze this task right now.}
  \end{multipleChoice}
\end{question}

\begin{question} 
    Could you simplify the following rational expression?
    $$\cfrac{x^2 - 4x}{x^2 + 2}$$
  \begin{multipleChoice}
      \choice[correct]{Yes, I could simplify this rational expression.}
      \choice[correct]{Yes I could, but I would need to review some resources first.}
      \choice[correct]{No, I would need to learn/relearn how to simplify this rational expression.}
      \choice[correct]{I don't know.}
      \choice[correct]{I don't want to answer this question right now.}
  \end{multipleChoice}
\end{question}

\begin{question} 
    Could you determine if the expression $\sqrt{4x^2 + 6}$ is equivalent to another expression?
  \begin{multipleChoice}
      \choice[correct]{Yes, I could determine if two expressions are equal.}
      \choice[correct]{Yes I could, but I would need to review some resources first.}
      \choice[correct]{No, I would need to learn/relearn how to determine if the expressions are equal.}
      \choice[correct]{I don't know.}
      \choice[correct]{I don't want to answer this question right now.}
  \end{multipleChoice}
\end{question}

%Content Questions

\begin{problem} 
\begin{problem}
    How does the following polynomial factor?
    $$2x^2 - 4$$
    
    \begin{hint}
    If you are having difficulties factoring this polynomial using common factors, you may want to review \href{https://ximera.osu.edu/math160fa17/m160prerequisites/prerequisiteVideos/fundamentalsOfFactoring}{this card}.
    \end{hint}
    
  \begin{multipleChoice}
      \choice{$(x-2)(x+2)$}
      \choice{$(x-2)^2$}
      \choice[correct]{$2(x-\sqrt{2})(x+\sqrt{2})$}
      \choice{$2(x-2)^2$}
      \choice{$2(x-\sqrt{2})^2$}
      \choice{$(x-\sqrt{2})(x+\sqrt{2})$}
      \choice{$(x-\sqrt{2})^2$}
      \choice[correct]{This expression cannot be factored.}
      

  \end{multipleChoice}
  
\end{problem}
\begin{question}

    If you didn't, why didn't you answer the above question?

  \begin{multipleChoice}
      \choice[correct]{I don't know the answer to this question yet.}
      \choice[correct]{I don't want to answer this question right now.}
      \choice[correct]{I have already answered this question correctly.}
      

     
  \end{multipleChoice}
\end{question}
\end{problem}

\begin{problem}
\begin{problem} 
    Factor the following polynomial:
    $$x^2 + 9x -36$$
    
    \begin{hint}
    If you have forgotten how to factor polynomials, working through \href{https://ximera.osu.edu/math160fa17/m160prerequisites/prerequisiteVideos/fundamentalsOfFactoring}{this card} will help to refresh your memory. 
    \end{hint}
    
  \begin{multipleChoice}
      \choice[correct]{$(x+12)(x-3)$}
      \choice{$(x+4)(x-5)$}
      \choice{$(x+3)(x-12)$}
      \choice{$(x-3)(x-12)$}
      \choice{This polynomial doesn't factor.}
      
      
  \end{multipleChoice}
  
\end{problem}
\begin{question}

    If you didn't, why didn't you answer the above question?
    
    \begin{multipleChoice}
      \choice[correct]{I don't know the answer to this question yet.}
      \choice[correct]{I don't want to answer this question right now.}
      \choice[correct]{I have already answered this question correctly.}

      
  \end{multipleChoice}
\end{question}
\end{problem}

\begin{problem} 
\begin{problem}

    Find the solutions to the following equality:
    $$3x^2 -2 = 5x$$
    
    
  \begin{multipleChoice}
      \choice{$x = -2, \cfrac{1}{3}$}
      \choice{$x = \cfrac{5}{2} + \cfrac{\sqrt{33}}{2}, \cfrac{5}{2} - \cfrac{\sqrt{33}}{2}$}
      \choice[correct]{$x = -\cfrac{1}{3}, 2$}
      \choice{$x = -\cfrac{5}{2} + \cfrac{\sqrt{33}}{2}, -\cfrac{5}{2} - \cfrac{\sqrt{33}}{2} $}
      

     
    \end{multipleChoice}
    
\end{problem}
\begin{question}

    If you didn't, why didn't you answer the above question?

    \begin{multipleChoice}
      \choice[correct]{I don't know the answer to this question yet.}
      \choice[correct]{I don't want to answer this question right now.}
      \choice[correct]{I have already answered this question correctly.}

  \end{multipleChoice}
\end{question}
\end{problem}

\begin{problem} 
\begin{problem}
    Are the following expressions equal?
    $$x^2 - 9 \text{ and } (x-3)^2$$
    
    \begin{hint}
    If you would like a quick recap of what it means to square a function, you may want to look at \href{https://ximera.osu.edu/math160fa17/m160prerequisites/prerequisiteVideos/distribAndMultPolys}{this card}.
    \end{hint}
    
  \begin{multipleChoice}
      \choice{Yes, they are equal.}
      \choice[correct]{No, they are not equal.}
      

        
  \end{multipleChoice}

\end{problem}

\begin{question}
  
    If you didn't, why didn't you answer the above question?
  
  \begin{multipleChoice}
      \choice[correct]{I don't know the answer to this question yet.}
      \choice[correct]{I don't want to answer this question right now.}
      \choice[correct]{I have already answered this question correctly.}
      
      
  \end{multipleChoice}
\end{question}
\end{problem}

\begin{problem} 
\begin{problem}
    Which of the following are valid simplifications of the rational expression $\cfrac{x^2 - 4x}{x^2 + 2}$? (Mark all correct simplifications)
    
    \begin{hint}
    If you are struggling to simplify this rational expression, you may want review how to simplify rational expressions on \href{https://ximera.osu.edu/math160fa17/m160prerequisites/PreRequisiteXards/U6MultiplyingAndFactoringPolynomials/6.3SimplifyingRationalFunctions/digInSimplifyingRationalFunctions}{this card}.
    \end{hint}

  \begin{selectAll}
      \choice{$\frac{-4x}{2}$}
      \choice{$-2x$}
      \choice{$\frac{2x}{x+\sqrt{2}}$}
      \choice{$\frac{2x}{x+2}$}
      \choice{$-2$}
      \choice[correct]{The rational expression does not simplify any more.}
      

  \end{selectAll}
\end{problem}
\begin{question}

    If you didn't, why didn't you answer the above question?
    
    \begin{multipleChoice}
      \choice[correct]{I don't know the answer to this question yet.}
      \choice[correct]{I don't want to answer this question right now.}
      \choice[correct]{I have already answered this question correctly.}
      

    \end{multipleChoice}
\end{question}
\end{problem}

\begin{problem}
\begin{problem}
    Which of the following expressions is/are equivalent to $\sqrt{4x^2 +6}$? (Mark all that are equivalent)
    
        
  \begin{selectAll}
      \choice{$2x + \sqrt{6}$}
      \choice{$2x + 3$}
      \choice[correct]{$2\sqrt{x^2 +\frac{3}{2}}$}
      \choice{$\sqrt{4x^2} + \sqrt{6}$}
      \choice{$4x + 6$}
      \choice{$2x + \sqrt{6}$}
      \choice{$\sqrt{4}\sqrt{x^2} + \sqrt{6}$}
      \choice{$2|x| + \sqrt{6}$}
      \choice{The expression is not equivalent to any of the above options.}
      

  \end{selectAll}
  
\end{problem}

\begin{question}
  
    If you didn't, why didn't you answer the above question?
  
  \begin{multipleChoice}
      \choice[correct]{I don't know the answer to this question yet.}
      \choice[correct]{I don't want to answer this question right now.}
      \choice[correct]{I have already answered this question correctly.}

      
  \end{multipleChoice}
  
\end{question}

\end{problem}

%Strategy Reflection

\begin{center} \textbf{Strategy Reflection} \end{center}

\begin{question}

Below are statements about strategies that you may or may not have used while completing the above questions.  Mark true or false to indicate whether each statement describes how you took this pre-quiz. 

\begin{question}

    A. I answered the questions in order, assuming that the easier ones would come first.

    \begin{multipleChoice}
        \choice[correct]{True}
        \choice[correct]{False}
    \end{multipleChoice}
    
\end{question}
\begin{question}
    
    B. I answered the questions that I knew how to do first.

    \begin{multipleChoice}
        \choice[correct]{True}
        \choice[correct]{False}
    \end{multipleChoice}
    
\end{question}
\begin{question}
    
    C. I read through all of the questions first before answering any.

    \begin{multipleChoice}
        \choice[correct]{True}
        \choice[correct]{False}
    \end{multipleChoice}
    
\end{question}
\begin{question}
    
    D. After reading a question, I worked backwards and used process of elimination to narrow down my options by checking if the answers were correct.

    \begin{multipleChoice}
        \choice[correct]{True}
        \choice[correct]{False}
    \end{multipleChoice}
    
\end{question}  
\begin{question}    
    
    E. When working on a problem, I solved the question by myself without the answers first, and then tried to select the appropriate answer.

    \begin{multipleChoice}
        \choice[correct]{True}
        \choice[correct]{False}
    \end{multipleChoice}
    
\end{question}

\begin{question}    
    
    F. I factored polynomials using the \href{https://people.richland.edu/james/misc/acmeth.html}{`AC' Method}.

    \begin{multipleChoice}
        \choice[correct]{True}
        \choice[correct]{False}
    \end{multipleChoice}
    
\end{question}

\begin{question}    
    
    G. I expanded the polynomials to determine if the expressions were equal.

    \begin{multipleChoice}
        \choice[correct]{True}
        \choice[correct]{False}
    \end{multipleChoice}
    
\end{question}

\begin{question}    
    
    H. I evaluated expressions at different $x$ values to determine if two functions were equivalent.

    \begin{multipleChoice}
        \choice[correct]{True}
        \choice[correct]{False}
    \end{multipleChoice}
    
\end{question}
      
\begin{question}    
    
    I. I used online resources to find the solutions.

    \begin{multipleChoice}
        \choice[correct]{True}
        \choice[correct]{False}
    \end{multipleChoice}
    
\end{question}
\begin{question}    
    
    J. I used online resources to remind me how to answer the questions, but then solved them myself.

    \begin{multipleChoice}
        \choice[correct]{True}
        \choice[correct]{False}
    \end{multipleChoice}
    
\end{question}

\begin{question}    
    
    K. I used a calculator to find the answers.

    \begin{multipleChoice}
        \choice[correct]{True}
        \choice[correct]{False}
    \end{multipleChoice}
    
\end{question}

\begin{question}    
    
    L. I used a calculator to check the solutions that I found by hand.

    \begin{multipleChoice}
        \choice[correct]{True}
        \choice[correct]{False}
    \end{multipleChoice}
    
\end{question}

\begin{question}    
    
    M. I used strategies when completing this pre-quiz, but I don't remember the names of the strategies.

    \begin{multipleChoice}
        \choice[correct]{True}
        \choice[correct]{False}
    \end{multipleChoice}
    
\end{question}
\begin{question}    
    
    N. I used other strategies that are not listed here.

    \begin{multipleChoice}
        \choice[correct]{True}
        \choice[correct]{False}
    \end{multipleChoice}
   
\end{question}
\begin{question}    
    
    O. I did not use any strategies when completing this pre-quiz.

    \begin{multipleChoice}
        \choice[correct]{True}
        \choice[correct]{False}
    \end{multipleChoice}

\end{question}
\end{question}

\begin{question}
Please list any additional strategies that you used when taking this pre-quiz. If you didn't use any additional strategies, just type `NA.`
   \begin{freeResponse}
   \end{freeResponse}
\end{question}

%%% \begin{xarmaBoost}
%%   Write down at least \textbf{five} questions for this lecture. After
%%   you have your questions, label them as ``Level 1,'' ``Level 2,'' or
%%   ``Level 3'' where:
%% \begin{description}
%% \item[Level 1] Means you know the answer, or know exactly how to do
%%   this problem.
%% \item[Level 2] Means you think you know how to do the problem.
%% \item[Level 3] Means you have no idea how to do the problem.
%% \end{description}
%% \begin{freeResponse}
%% \end{freeResponse}
%% \end{xarmaBoost}



\end{document}
