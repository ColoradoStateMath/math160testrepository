\documentclass{ximera}

%\usepackage{todonotes}

\newcommand{\todo}{}

\usepackage{esint} % for \oiint
\graphicspath{
{./}
{functionsOfSeveralVariables/}
{normalVectors/}
{lagrangeMultipliers/}
{vectorFields/}
{greensTheorem/}
{shapeOfThingsToCome/}
}


\usepackage{tkz-euclide}
\tikzset{>=stealth} %% cool arrow head
\tikzset{shorten <>/.style={ shorten >=#1, shorten <=#1 } } %% allows shorter vectors

\usetikzlibrary{backgrounds} %% for boxes around graphs
\usetikzlibrary{shapes,positioning}  %% Clouds and stars
\usetikzlibrary{matrix} %% for matrix
\usepgfplotslibrary{polar} %% for polar plots
\usetkzobj{all}
\usepackage[makeroom]{cancel} %% for strike outs
%\usepackage{mathtools} %% for pretty underbrace % Breaks Ximera
\usepackage{multicol}
\usepackage{pgffor} %% required for integral for loops


%% http://tex.stackexchange.com/questions/66490/drawing-a-tikz-arc-specifying-the-center
%% Draws beach ball
\tikzset{pics/carc/.style args={#1:#2:#3}{code={\draw[pic actions] (#1:#3) arc(#1:#2:#3);}}}



\usepackage{array}
\setlength{\extrarowheight}{+.1cm}   
\newdimen\digitwidth
\settowidth\digitwidth{9}
\def\divrule#1#2{
\noalign{\moveright#1\digitwidth
\vbox{\hrule width#2\digitwidth}}}





\newcommand{\RR}{\mathbb R}
\newcommand{\R}{\mathbb R}
\newcommand{\N}{\mathbb N}
\newcommand{\Z}{\mathbb Z}

%\newcommand{\sage}{\textsf{SageMath}}


%\renewcommand{\d}{\,d\!}
\renewcommand{\d}{\mathop{}\!d}
\newcommand{\dd}[2][]{\frac{\d #1}{\d #2}}
\newcommand{\pp}[2][]{\frac{\partial #1}{\partial #2}}
\renewcommand{\l}{\ell}
\newcommand{\ddx}{\frac{d}{\d x}}

\newcommand{\zeroOverZero}{\ensuremath{\boldsymbol{\tfrac{0}{0}}}}
\newcommand{\inftyOverInfty}{\ensuremath{\boldsymbol{\tfrac{\infty}{\infty}}}}
\newcommand{\zeroOverInfty}{\ensuremath{\boldsymbol{\tfrac{0}{\infty}}}}
\newcommand{\zeroTimesInfty}{\ensuremath{\small\boldsymbol{0\cdot \infty}}}
\newcommand{\inftyMinusInfty}{\ensuremath{\small\boldsymbol{\infty - \infty}}}
\newcommand{\oneToInfty}{\ensuremath{\boldsymbol{1^\infty}}}
\newcommand{\zeroToZero}{\ensuremath{\boldsymbol{0^0}}}
\newcommand{\inftyToZero}{\ensuremath{\boldsymbol{\infty^0}}}



\newcommand{\numOverZero}{\ensuremath{\boldsymbol{\tfrac{\#}{0}}}}
\newcommand{\dfn}{\textbf}
%\newcommand{\unit}{\,\mathrm}
\newcommand{\unit}{\mathop{}\!\mathrm}
\newcommand{\eval}[1]{\bigg[ #1 \bigg]}
\newcommand{\seq}[1]{\left( #1 \right)}
\renewcommand{\epsilon}{\varepsilon}
\renewcommand{\phi}{\varphi}


\renewcommand{\iff}{\Leftrightarrow}

\DeclareMathOperator{\arccot}{arccot}
\DeclareMathOperator{\arcsec}{arcsec}
\DeclareMathOperator{\arccsc}{arccsc}
\DeclareMathOperator{\si}{Si}
\DeclareMathOperator{\proj}{\vec{proj}}
\DeclareMathOperator{\scal}{scal}
\DeclareMathOperator{\sign}{sign}


%% \newcommand{\tightoverset}[2]{% for arrow vec
%%   \mathop{#2}\limits^{\vbox to -.5ex{\kern-0.75ex\hbox{$#1$}\vss}}}
\newcommand{\arrowvec}{\overrightarrow}
%\renewcommand{\vec}[1]{\arrowvec{\mathbf{#1}}}
\renewcommand{\vec}{\mathbf}
\newcommand{\veci}{{\boldsymbol{\hat{\imath}}}}
\newcommand{\vecj}{{\boldsymbol{\hat{\jmath}}}}
\newcommand{\veck}{{\boldsymbol{\hat{k}}}}
\newcommand{\vecl}{\boldsymbol{\l}}
\newcommand{\uvec}[1]{\mathbf{\hat{#1}}}
\newcommand{\utan}{\mathbf{\hat{t}}}
\newcommand{\unormal}{\mathbf{\hat{n}}}
\newcommand{\ubinormal}{\mathbf{\hat{b}}}

\newcommand{\dotp}{\bullet}
\newcommand{\cross}{\boldsymbol\times}
\newcommand{\grad}{\boldsymbol\nabla}
\newcommand{\divergence}{\grad\dotp}
\newcommand{\curl}{\grad\cross}
%\DeclareMathOperator{\divergence}{divergence}
%\DeclareMathOperator{\curl}[1]{\grad\cross #1}
\newcommand{\lto}{\mathop{\longrightarrow\,}\limits}

\renewcommand{\bar}{\overline}

\colorlet{textColor}{black} 
\colorlet{background}{white}
\colorlet{penColor}{blue!50!black} % Color of a curve in a plot
\colorlet{penColor2}{red!50!black}% Color of a curve in a plot
\colorlet{penColor3}{red!50!blue} % Color of a curve in a plot
\colorlet{penColor4}{green!50!black} % Color of a curve in a plot
\colorlet{penColor5}{orange!80!black} % Color of a curve in a plot
\colorlet{penColor6}{yellow!70!black} % Color of a curve in a plot
\colorlet{fill1}{penColor!20} % Color of fill in a plot
\colorlet{fill2}{penColor2!20} % Color of fill in a plot
\colorlet{fillp}{fill1} % Color of positive area
\colorlet{filln}{penColor2!20} % Color of negative area
\colorlet{fill3}{penColor3!20} % Fill
\colorlet{fill4}{penColor4!20} % Fill
\colorlet{fill5}{penColor5!20} % Fill
\colorlet{gridColor}{gray!50} % Color of grid in a plot

\newcommand{\surfaceColor}{violet}
\newcommand{\surfaceColorTwo}{redyellow}
\newcommand{\sliceColor}{greenyellow}




\pgfmathdeclarefunction{gauss}{2}{% gives gaussian
  \pgfmathparse{1/(#2*sqrt(2*pi))*exp(-((x-#1)^2)/(2*#2^2))}%
}


%%%%%%%%%%%%%
%% Vectors
%%%%%%%%%%%%%

%% Simple horiz vectors
\renewcommand{\vector}[1]{\left\langle #1\right\rangle}


%% %% Complex Horiz Vectors with angle brackets
%% \makeatletter
%% \renewcommand{\vector}[2][ , ]{\left\langle%
%%   \def\nextitem{\def\nextitem{#1}}%
%%   \@for \el:=#2\do{\nextitem\el}\right\rangle%
%% }
%% \makeatother

%% %% Vertical Vectors
%% \def\vector#1{\begin{bmatrix}\vecListA#1,,\end{bmatrix}}
%% \def\vecListA#1,{\if,#1,\else #1\cr \expandafter \vecListA \fi}

%%%%%%%%%%%%%
%% End of vectors
%%%%%%%%%%%%%

%\newcommand{\fullwidth}{}
%\newcommand{\normalwidth}{}



%% makes a snazzy t-chart for evaluating functions
%\newenvironment{tchart}{\rowcolors{2}{}{background!90!textColor}\array}{\endarray}

%%This is to help with formatting on future title pages.
\newenvironment{sectionOutcomes}{}{} 



%% Flowchart stuff
%\tikzstyle{startstop} = [rectangle, rounded corners, minimum width=3cm, minimum height=1cm,text centered, draw=black]
%\tikzstyle{question} = [rectangle, minimum width=3cm, minimum height=1cm, text centered, draw=black]
%\tikzstyle{decision} = [trapezium, trapezium left angle=70, trapezium right angle=110, minimum width=3cm, minimum height=1cm, text centered, draw=black]
%\tikzstyle{question} = [rectangle, rounded corners, minimum width=3cm, minimum height=1cm,text centered, draw=black]
%\tikzstyle{process} = [rectangle, minimum width=3cm, minimum height=1cm, text centered, draw=black]
%\tikzstyle{decision} = [trapezium, trapezium left angle=70, trapezium right angle=110, minimum width=3cm, minimum height=1cm, text centered, draw=black]

\title[Dig-In]{Maximums and minimums}
\author{MooMaster}

\outcome{Define absolute maximum and absolute minimum.}
\outcome{Find the absolute maximum and minimum using a graph.}
\outcome{Define relative maximum and relative minimum.}
\outcome{Find relative maxima and minima using a graph.}
\outcome{Compare and contrast relative and absolute maxima and minima.}
\outcome{Given a graph without an absolute maximum or minimum, explain why the graph has no absolute maximum or minimum.}
\outcome{Given a graph without any extrema, explain why the graph has no extrema.}
  

\begin{document}

\begin{abstract}
On this card, we investigate what is meant by the maxima and minima of a function.  
\end{abstract}
\maketitle

To motivate why you should care about maxima and minima, let's start with a realistic example. 

%% Motivating Example - Maximizing Profit and Minimizing Cost
\begin{example}
The company you work for sells fancy speaker systems.  As the resident mathematician, you've determined that your company's profit (in dollars) is given by 
\[ P(x) = -0.02x^2 + 150x-200000 \]
where $x$ represents the number of speaker systems your company produces.  How many speaker systems should your company produce in order to maximize profits? 
\begin{explanation}
From just the formula for the profit function, this question is seemingly pretty difficult, so let's plot the profit function to understand what this function looks like.  

\begin{center} \includegraphics[scale=0.5]{extrema1.png} \end{center}

\underline{\hspace{5in}}

Aha!  Now it's plain to see that the maximum amount of profit is achieved when 3,750 speaker systems are produced.  According to the graph, the maximum profit that would be earned in this situation is $\$81250$.  We say that the profit function $P(x)$ has a maximum of $81250$ that is achieved at $x=3750$.  

\end{explanation}
\end{example}

The above example illustrates one reason why you may want to identify the maximum of a function like $P(x)$.  You may also want to find the minimum of a function.  For example, you may want to determine how many speakers your company should produce to minimize production cost.  In this case, you would want to locate the minimum of your company's cost function.  Examples such as these motivate the next definition. 

%%Definition - Absolute/Global Maximum/Minimum.
\begin{definition}\hfil\index{maximum/minimum!absolute}
\begin{enumerate}
\item A function $f$ has an \dfn{absolute maximum} at $x=a$, if $f(a)\ge
  f(x)$ for every $x$ in the domain of the function.
\item A function $f$ has an \dfn{absolute minimum} at $x=a$, if $f(a)\le
  f(x)$ for every $x$ in the domain of the function.
\end{enumerate} 
An \dfn{absolute extremum}\index{extremum!absolute} is either a
absolute maximum or absolute global minimum.  
\end{definition}

\begin{explanation}
Although this definition is worded formally, it is simply defining the absolute maximum of a function, $f(x)$, to be the largest output or $f$-value that the function achieves.  If there is, indeed, a maximum at $x=a$, then the function value at $x=a$, $f(a)$, will be greater than (or equal to) all of the other function values for the function, so $f(a) \geq f(x)$ for all $x$ in the domain of $f$.  \\

Similarly, if there is, indeed, an absolute minimum of $f(x)$ at $x=a$ that means that $f(a)$ is the smallest output or $f$-value that the function achieves.  In other words, $f(a) \leq f(x)$ for all $x$ in the domain of $f$. \\
\end{explanation}

\begin{warning}
It is also common to say \textit{global} maximum, \textit{global} minimum, and \textit{global} extremum.  Be aware that these words convey the same meaning as the words above! 
\end{warning}

\begin{example}
Find the absolute extrema of $P(x) = -0.02x^2 + 150x-200000$.
\begin{explanation}
Finding the absolute extrema of $P(x)$ is the same as finding the absolute maximum and absolute minimum of $P(x)$.  Let's start with the absolute maximum.  \\

From before, we know that the largest profit is $\$81250$, so the absolute maximum of $P(x)$ is $\answer{81250}$, which is achieved when $x \ = \answer{3750}$.  The domain of $P$ is $(-\infty, \infty)$ and $P(3750) = 81250 \geq P(x)$ for every $x$ in the $(\infty, \infty)$, so our observations from before are consistent with the formal definition of absolute maximum. \\

\begin{explanation}

Now, let's move on to identifying the absolute minimum of $P$.  The absolute minimum occurs at $x=a$ if $P(a) \leq P(x)$ for all $x$ in the domain of $P$, which is $(\infty, \infty)$.  Referring to the graph of $P(x)$ above, you can see that $P(x)$ \wordChoice{\choice{does}\choice[correct]{does not}} have an absolute minimum. 

\begin{feedback}[correct]
You got it.  We can determine that $P(x)$ has no absolute minimum because $P$ has no smallest output value: the $P$-values continue to decrease both as $x$ becomes very positive and as $x$ becomes very negative. 
\end{feedback}

\begin{explanation}

Let's write a quick conclusion to finish off this example. \\

On the \wordChoice{\choice[correct]{open}\choice{closed}} interval $( \answer{-\infty}, \answer{\infty} )$, the function $P(x)$ \wordChoice{\choice[correct]{is}\choice{is not}} continuous, and \wordChoice{\choice{does}\choice[correct]{does not}} have BOTH an absolute minimum and an absolute maximum.  

\end{explanation}

\end{explanation}

\end{explanation}

\end{example}

\begin{exercise}
Use the graph of $y = f(x)$ below to identify the absolute extrema of $f$.

\begin{center} \includegraphics[scale=0.5]{extrema2.png} \end{center}

The function $f(x)$ has an absolute minimum of $\answer{0}$ at $x \ = \answer{-1} \text{ and } x \ = \answer{7}$.  

\begin{exercise}
You got it!  Notice that there is only one minimum ($y=0$), and it is achieved at two distinct $x$-values.  \\

The function $f(x)$ \wordChoice{\choice{does}\choice[correct]{does not}} have an absolute maximum.  

\begin{explanation}
The fact that $f$ does not have an absolute maximum surprises many students at first, so let's think about this a bit more.  There is a hole in the graph at the point that would correspond to an absolute maximum of $y=4$ if the hole were filled in.  Since there is a hole, the maximum cannot be $y=4$ because there is no $x$ value so that $f(x) = 4$.  So what's the next highest $f$-value this function achieves?  What about $y=3.9$?  Looking at the graph, you'll notice that $f$ does achieve that $y$-value, but $f$ also achieves the bigger $y$-value $y=3.99$.  And $f$ also achieves the even bigger $y=3.999$.  So is the absolute maximum $y=3.999$?  Nope!  We could keep playing this game forever.  You might guess the absolute maximum of $f$ is $y=3.999$ with $3$ $9$'s, but then I could come back and point out that $f$ achieves $y=3.9999$ with $4$ $9$'s, which is even bigger.  We could play this game forever, too!  The fact that we could play this game forever relies on the fact that the real numbers are \textit{dense}: given the two real numbers $3.999$ and $4$, you can always find a number $c$ such that $3.999 < c < 4$.  For example, $c = 3.9999$ would work in this case.  
\end{explanation}


\end{exercise}
\end{exercise}
 


\end{document}
