\documentclass{ximera}

%\usepackage{todonotes}

\newcommand{\todo}{}

\usepackage{esint} % for \oiint
\graphicspath{
{./}
{functionsOfSeveralVariables/}
{normalVectors/}
{lagrangeMultipliers/}
{vectorFields/}
{greensTheorem/}
{shapeOfThingsToCome/}
}


\usepackage{tkz-euclide}
\tikzset{>=stealth} %% cool arrow head
\tikzset{shorten <>/.style={ shorten >=#1, shorten <=#1 } } %% allows shorter vectors

\usetikzlibrary{backgrounds} %% for boxes around graphs
\usetikzlibrary{shapes,positioning}  %% Clouds and stars
\usetikzlibrary{matrix} %% for matrix
\usepgfplotslibrary{polar} %% for polar plots
\usetkzobj{all}
\usepackage[makeroom]{cancel} %% for strike outs
%\usepackage{mathtools} %% for pretty underbrace % Breaks Ximera
\usepackage{multicol}
\usepackage{pgffor} %% required for integral for loops


%% http://tex.stackexchange.com/questions/66490/drawing-a-tikz-arc-specifying-the-center
%% Draws beach ball
\tikzset{pics/carc/.style args={#1:#2:#3}{code={\draw[pic actions] (#1:#3) arc(#1:#2:#3);}}}



\usepackage{array}
\setlength{\extrarowheight}{+.1cm}   
\newdimen\digitwidth
\settowidth\digitwidth{9}
\def\divrule#1#2{
\noalign{\moveright#1\digitwidth
\vbox{\hrule width#2\digitwidth}}}





\newcommand{\RR}{\mathbb R}
\newcommand{\R}{\mathbb R}
\newcommand{\N}{\mathbb N}
\newcommand{\Z}{\mathbb Z}

%\newcommand{\sage}{\textsf{SageMath}}


%\renewcommand{\d}{\,d\!}
\renewcommand{\d}{\mathop{}\!d}
\newcommand{\dd}[2][]{\frac{\d #1}{\d #2}}
\newcommand{\pp}[2][]{\frac{\partial #1}{\partial #2}}
\renewcommand{\l}{\ell}
\newcommand{\ddx}{\frac{d}{\d x}}

\newcommand{\zeroOverZero}{\ensuremath{\boldsymbol{\tfrac{0}{0}}}}
\newcommand{\inftyOverInfty}{\ensuremath{\boldsymbol{\tfrac{\infty}{\infty}}}}
\newcommand{\zeroOverInfty}{\ensuremath{\boldsymbol{\tfrac{0}{\infty}}}}
\newcommand{\zeroTimesInfty}{\ensuremath{\small\boldsymbol{0\cdot \infty}}}
\newcommand{\inftyMinusInfty}{\ensuremath{\small\boldsymbol{\infty - \infty}}}
\newcommand{\oneToInfty}{\ensuremath{\boldsymbol{1^\infty}}}
\newcommand{\zeroToZero}{\ensuremath{\boldsymbol{0^0}}}
\newcommand{\inftyToZero}{\ensuremath{\boldsymbol{\infty^0}}}



\newcommand{\numOverZero}{\ensuremath{\boldsymbol{\tfrac{\#}{0}}}}
\newcommand{\dfn}{\textbf}
%\newcommand{\unit}{\,\mathrm}
\newcommand{\unit}{\mathop{}\!\mathrm}
\newcommand{\eval}[1]{\bigg[ #1 \bigg]}
\newcommand{\seq}[1]{\left( #1 \right)}
\renewcommand{\epsilon}{\varepsilon}
\renewcommand{\phi}{\varphi}


\renewcommand{\iff}{\Leftrightarrow}

\DeclareMathOperator{\arccot}{arccot}
\DeclareMathOperator{\arcsec}{arcsec}
\DeclareMathOperator{\arccsc}{arccsc}
\DeclareMathOperator{\si}{Si}
\DeclareMathOperator{\proj}{\vec{proj}}
\DeclareMathOperator{\scal}{scal}
\DeclareMathOperator{\sign}{sign}


%% \newcommand{\tightoverset}[2]{% for arrow vec
%%   \mathop{#2}\limits^{\vbox to -.5ex{\kern-0.75ex\hbox{$#1$}\vss}}}
\newcommand{\arrowvec}{\overrightarrow}
%\renewcommand{\vec}[1]{\arrowvec{\mathbf{#1}}}
\renewcommand{\vec}{\mathbf}
\newcommand{\veci}{{\boldsymbol{\hat{\imath}}}}
\newcommand{\vecj}{{\boldsymbol{\hat{\jmath}}}}
\newcommand{\veck}{{\boldsymbol{\hat{k}}}}
\newcommand{\vecl}{\boldsymbol{\l}}
\newcommand{\uvec}[1]{\mathbf{\hat{#1}}}
\newcommand{\utan}{\mathbf{\hat{t}}}
\newcommand{\unormal}{\mathbf{\hat{n}}}
\newcommand{\ubinormal}{\mathbf{\hat{b}}}

\newcommand{\dotp}{\bullet}
\newcommand{\cross}{\boldsymbol\times}
\newcommand{\grad}{\boldsymbol\nabla}
\newcommand{\divergence}{\grad\dotp}
\newcommand{\curl}{\grad\cross}
%\DeclareMathOperator{\divergence}{divergence}
%\DeclareMathOperator{\curl}[1]{\grad\cross #1}
\newcommand{\lto}{\mathop{\longrightarrow\,}\limits}

\renewcommand{\bar}{\overline}

\colorlet{textColor}{black} 
\colorlet{background}{white}
\colorlet{penColor}{blue!50!black} % Color of a curve in a plot
\colorlet{penColor2}{red!50!black}% Color of a curve in a plot
\colorlet{penColor3}{red!50!blue} % Color of a curve in a plot
\colorlet{penColor4}{green!50!black} % Color of a curve in a plot
\colorlet{penColor5}{orange!80!black} % Color of a curve in a plot
\colorlet{penColor6}{yellow!70!black} % Color of a curve in a plot
\colorlet{fill1}{penColor!20} % Color of fill in a plot
\colorlet{fill2}{penColor2!20} % Color of fill in a plot
\colorlet{fillp}{fill1} % Color of positive area
\colorlet{filln}{penColor2!20} % Color of negative area
\colorlet{fill3}{penColor3!20} % Fill
\colorlet{fill4}{penColor4!20} % Fill
\colorlet{fill5}{penColor5!20} % Fill
\colorlet{gridColor}{gray!50} % Color of grid in a plot

\newcommand{\surfaceColor}{violet}
\newcommand{\surfaceColorTwo}{redyellow}
\newcommand{\sliceColor}{greenyellow}




\pgfmathdeclarefunction{gauss}{2}{% gives gaussian
  \pgfmathparse{1/(#2*sqrt(2*pi))*exp(-((x-#1)^2)/(2*#2^2))}%
}


%%%%%%%%%%%%%
%% Vectors
%%%%%%%%%%%%%

%% Simple horiz vectors
\renewcommand{\vector}[1]{\left\langle #1\right\rangle}


%% %% Complex Horiz Vectors with angle brackets
%% \makeatletter
%% \renewcommand{\vector}[2][ , ]{\left\langle%
%%   \def\nextitem{\def\nextitem{#1}}%
%%   \@for \el:=#2\do{\nextitem\el}\right\rangle%
%% }
%% \makeatother

%% %% Vertical Vectors
%% \def\vector#1{\begin{bmatrix}\vecListA#1,,\end{bmatrix}}
%% \def\vecListA#1,{\if,#1,\else #1\cr \expandafter \vecListA \fi}

%%%%%%%%%%%%%
%% End of vectors
%%%%%%%%%%%%%

%\newcommand{\fullwidth}{}
%\newcommand{\normalwidth}{}



%% makes a snazzy t-chart for evaluating functions
%\newenvironment{tchart}{\rowcolors{2}{}{background!90!textColor}\array}{\endarray}

%%This is to help with formatting on future title pages.
\newenvironment{sectionOutcomes}{}{} 



%% Flowchart stuff
%\tikzstyle{startstop} = [rectangle, rounded corners, minimum width=3cm, minimum height=1cm,text centered, draw=black]
%\tikzstyle{question} = [rectangle, minimum width=3cm, minimum height=1cm, text centered, draw=black]
%\tikzstyle{decision} = [trapezium, trapezium left angle=70, trapezium right angle=110, minimum width=3cm, minimum height=1cm, text centered, draw=black]
%\tikzstyle{question} = [rectangle, rounded corners, minimum width=3cm, minimum height=1cm,text centered, draw=black]
%\tikzstyle{process} = [rectangle, minimum width=3cm, minimum height=1cm, text centered, draw=black]
%\tikzstyle{decision} = [trapezium, trapezium left angle=70, trapezium right angle=110, minimum width=3cm, minimum height=1cm, text centered, draw=black]
\title[Prerequisite Videos:]{[Video] Introduction to Logarithms}
\author{Ben Sencindiver}

\outcome{Understand logarithmic functions.}

\begin{document}
\begin{abstract}
  In this series of videos, we will explore how logarithmic functions
  work and what they are.
\end{abstract}
\maketitle

The following videos talk about logarithms: 

%% Introduction and Question 1
\textbf{Introduction and Question 1: The definition of logarithm and what
logarithms means}
\begin{question}
%% Labeling this expandable option
\begin{flushright}
{\color{blue}(\emph{Click the arrow to the right to see the Introduction video and first question.})}
\end{flushright}
\begin{center}
\begin{expandable}
\youtube{p9bSDw_Dxqo}
%% Multiple Choice Question 1
{\color{blue}(\emph{Click the arrow to the right to see the answers 
to the question posed at the end of the video.})}
\begin{expandable}
What is $\log_5(125)$?
\begin{prompt}
\[
\log_5(125) \text{ is exactly }\answer[given]{3}.
\]
\end{prompt}
%% Example 1
\begin{flushright}
{\color{blue}(\emph{Click the arrow to the right to see an example.})}
\end{flushright}
\begin{expandable}
Example 1
\youtube{3K-sipQyZrQ}
\end{expandable}
\end{expandable}
\end{expandable}
\end{center}
\end{question}


%% Question 2
\textbf{Question 2: Logarithms and the number $1$}
\begin{question}
%% Labeling this expandable option
\begin{flushright}
{\color{blue}(\emph{Click the arrow to the right to see the second question.})}
\end{flushright}
\begin{center}
\begin{expandable}
\youtube{2CEPHloxJ9E}
%% Multiple Choice Question 2
{\color{blue}(\emph{Click the arrow to the right to see the answers 
to the question posed at the end of the video.})}
\begin{expandable}
What is $\log_4(1)$?
\begin{prompt}
\[
\log_4(1) \text{ is exactly }\answer[given]{0}
\]
\end{prompt}
%% Example 2
\begin{flushright}
{\color{blue}(\emph{Click the arrow to the right to see an example.})}
\end{flushright}
\begin{expandable}
Example 2
\youtube{887-Dbnlhmc}
\end{expandable}
\end{expandable}
\end{expandable}
\end{center}
\end{question}


%% Question 3
\textbf{Question 3: Logarithms, fractions, and reciprocals}
\begin{question}
%% Labeling this expandable option
\begin{flushright}
{\color{blue}(\emph{Click the arrow to the right to see the third question.})}
\end{flushright}
\begin{center}
\begin{expandable}
\youtube{lMX5Mg0wNz4}
%% Multiple Choice Question 3
{\color{blue}(\emph{Click the arrow to the right to see the answers 
to the question posed at the end of the video.})}
\begin{expandable}
What is $\log_2\Big(\frac{1}{32}\Big)$?
\begin{prompt}
\[
\log_2\Big(\frac{1}{32} \Big) \text{ is exactly }\answer[given]{-5}
\]
\end{prompt}
%% Example 3
\begin{flushright}
{\color{blue}(\emph{Click the arrow to the right to see an example.})}
\end{flushright}
\begin{expandable}
Example 3
\youtube{ksDJFkogOwo}
\end{expandable}
\end{expandable}
\end{expandable}
\end{center}
\end{question}



%% Properties of Logarithmic
\textbf{Properties of Logarithms: Dealing with exponentials, products,
and division inside your logarithmic function}
\begin{explanation}
%% Labeling this expandable option
\begin{flushright}
{\color{blue}(\emph{Click the arrow to the right to see the nifty properties of logarithms.})}
\end{flushright}
\begin{center}
\begin{expandable}
\youtube{xjutyDI73m0}
{\color{blue}(\emph{Click the arrow to the right to see the property
about dealing with exponents inside of your logarithmic function from the video.})}
\begin{expandable}
Property 1
\[
\log_b(a^m) = m\log_b(a) \text{ when }a,b>0\text{ and }b\neq 1
\]
\end{expandable}
{\color{blue}(\emph{Click the arrow to the right to see the property
about dealing with exponents inside of your logarithmic function from the video.})}
\begin{expandable}
Property 2
\[
\log_b (xy) = \log_b(x) + \log_b(y) \text{ when }b\neq 1, \text{ and both }x \text{ and }y\text{ are constants.}
\]
\end{expandable}
{\color{blue}(\emph{Click the arrow to the right to see the property
about dealing with exponents inside of your logarithmic function from the video.})}
\begin{expandable}
Property 3
\[
\log_b \Big(\frac{x}{y}\Big) = \log_b(x) - \log_b(y) \text{ when }b,x,y>0,\text{ and }b\neq 1.
\]
\end{expandable}
\end{expandable}
\end{center}
\end{explanation}



%% Question 4
\textbf{Question 4: Using properties of logarithms}
\begin{question}
%% Labeling this expandable option
\begin{flushright}
{\color{blue}(\emph{Click the arrow to the right to see the fourth question.})}
\end{flushright}
\begin{center}
\begin{expandable}
\youtube{fM98tz2h3hk}
%% Multiple Choice Question 4
{\color{blue}(\emph{Click the arrow to the right to see the question
posed at the end of the video.})}
\begin{expandable}
The number $\log_2(56)$ can be written as a sum of constant numbers
and  logarithms where the inputs of the log base $2$ functions are prime.
Simplify the expression so it is a sum of constants and logarithmic
expressions of the form $n\log_2(p)$ where $p$ is prime. If a portion
of the logarithmic expressions can be simplified to an integer, do so.\\
\begin{center}
$\log_2(56)$ can be written as a sum of $\answer[given]{1}$ term(s)
$n\log_2(p)$ (where the $p$'s are distinct primes) and a constant integer.
\end{center}
\begin{feedback}
\[
\log_2(56) = \answer[given]{1}\cdot \log_2 (\answer[given]{7}) + \answer[given]{3}
\]
\end{feedback}
%% Example 4
\begin{flushright}
{\color{blue}(\emph{Click the arrow to the right to see an example.})}
\end{flushright}
\begin{expandable}
Example 4
\youtube{Sb6SHhwMagM}
\end{expandable}
\end{expandable}
\end{expandable}
\end{center}
\end{question}


\end{document}
