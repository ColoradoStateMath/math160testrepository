\documentclass{ximera}

%\usepackage{todonotes}

\newcommand{\todo}{}

\usepackage{esint} % for \oiint
\graphicspath{
{./}
{functionsOfSeveralVariables/}
{normalVectors/}
{lagrangeMultipliers/}
{vectorFields/}
{greensTheorem/}
{shapeOfThingsToCome/}
}


\usepackage{tkz-euclide}
\tikzset{>=stealth} %% cool arrow head
\tikzset{shorten <>/.style={ shorten >=#1, shorten <=#1 } } %% allows shorter vectors

\usetikzlibrary{backgrounds} %% for boxes around graphs
\usetikzlibrary{shapes,positioning}  %% Clouds and stars
\usetikzlibrary{matrix} %% for matrix
\usepgfplotslibrary{polar} %% for polar plots
\usetkzobj{all}
\usepackage[makeroom]{cancel} %% for strike outs
%\usepackage{mathtools} %% for pretty underbrace % Breaks Ximera
\usepackage{multicol}
\usepackage{pgffor} %% required for integral for loops


%% http://tex.stackexchange.com/questions/66490/drawing-a-tikz-arc-specifying-the-center
%% Draws beach ball
\tikzset{pics/carc/.style args={#1:#2:#3}{code={\draw[pic actions] (#1:#3) arc(#1:#2:#3);}}}



\usepackage{array}
\setlength{\extrarowheight}{+.1cm}   
\newdimen\digitwidth
\settowidth\digitwidth{9}
\def\divrule#1#2{
\noalign{\moveright#1\digitwidth
\vbox{\hrule width#2\digitwidth}}}





\newcommand{\RR}{\mathbb R}
\newcommand{\R}{\mathbb R}
\newcommand{\N}{\mathbb N}
\newcommand{\Z}{\mathbb Z}

%\newcommand{\sage}{\textsf{SageMath}}


%\renewcommand{\d}{\,d\!}
\renewcommand{\d}{\mathop{}\!d}
\newcommand{\dd}[2][]{\frac{\d #1}{\d #2}}
\newcommand{\pp}[2][]{\frac{\partial #1}{\partial #2}}
\renewcommand{\l}{\ell}
\newcommand{\ddx}{\frac{d}{\d x}}

\newcommand{\zeroOverZero}{\ensuremath{\boldsymbol{\tfrac{0}{0}}}}
\newcommand{\inftyOverInfty}{\ensuremath{\boldsymbol{\tfrac{\infty}{\infty}}}}
\newcommand{\zeroOverInfty}{\ensuremath{\boldsymbol{\tfrac{0}{\infty}}}}
\newcommand{\zeroTimesInfty}{\ensuremath{\small\boldsymbol{0\cdot \infty}}}
\newcommand{\inftyMinusInfty}{\ensuremath{\small\boldsymbol{\infty - \infty}}}
\newcommand{\oneToInfty}{\ensuremath{\boldsymbol{1^\infty}}}
\newcommand{\zeroToZero}{\ensuremath{\boldsymbol{0^0}}}
\newcommand{\inftyToZero}{\ensuremath{\boldsymbol{\infty^0}}}



\newcommand{\numOverZero}{\ensuremath{\boldsymbol{\tfrac{\#}{0}}}}
\newcommand{\dfn}{\textbf}
%\newcommand{\unit}{\,\mathrm}
\newcommand{\unit}{\mathop{}\!\mathrm}
\newcommand{\eval}[1]{\bigg[ #1 \bigg]}
\newcommand{\seq}[1]{\left( #1 \right)}
\renewcommand{\epsilon}{\varepsilon}
\renewcommand{\phi}{\varphi}


\renewcommand{\iff}{\Leftrightarrow}

\DeclareMathOperator{\arccot}{arccot}
\DeclareMathOperator{\arcsec}{arcsec}
\DeclareMathOperator{\arccsc}{arccsc}
\DeclareMathOperator{\si}{Si}
\DeclareMathOperator{\proj}{\vec{proj}}
\DeclareMathOperator{\scal}{scal}
\DeclareMathOperator{\sign}{sign}


%% \newcommand{\tightoverset}[2]{% for arrow vec
%%   \mathop{#2}\limits^{\vbox to -.5ex{\kern-0.75ex\hbox{$#1$}\vss}}}
\newcommand{\arrowvec}{\overrightarrow}
%\renewcommand{\vec}[1]{\arrowvec{\mathbf{#1}}}
\renewcommand{\vec}{\mathbf}
\newcommand{\veci}{{\boldsymbol{\hat{\imath}}}}
\newcommand{\vecj}{{\boldsymbol{\hat{\jmath}}}}
\newcommand{\veck}{{\boldsymbol{\hat{k}}}}
\newcommand{\vecl}{\boldsymbol{\l}}
\newcommand{\uvec}[1]{\mathbf{\hat{#1}}}
\newcommand{\utan}{\mathbf{\hat{t}}}
\newcommand{\unormal}{\mathbf{\hat{n}}}
\newcommand{\ubinormal}{\mathbf{\hat{b}}}

\newcommand{\dotp}{\bullet}
\newcommand{\cross}{\boldsymbol\times}
\newcommand{\grad}{\boldsymbol\nabla}
\newcommand{\divergence}{\grad\dotp}
\newcommand{\curl}{\grad\cross}
%\DeclareMathOperator{\divergence}{divergence}
%\DeclareMathOperator{\curl}[1]{\grad\cross #1}
\newcommand{\lto}{\mathop{\longrightarrow\,}\limits}

\renewcommand{\bar}{\overline}

\colorlet{textColor}{black} 
\colorlet{background}{white}
\colorlet{penColor}{blue!50!black} % Color of a curve in a plot
\colorlet{penColor2}{red!50!black}% Color of a curve in a plot
\colorlet{penColor3}{red!50!blue} % Color of a curve in a plot
\colorlet{penColor4}{green!50!black} % Color of a curve in a plot
\colorlet{penColor5}{orange!80!black} % Color of a curve in a plot
\colorlet{penColor6}{yellow!70!black} % Color of a curve in a plot
\colorlet{fill1}{penColor!20} % Color of fill in a plot
\colorlet{fill2}{penColor2!20} % Color of fill in a plot
\colorlet{fillp}{fill1} % Color of positive area
\colorlet{filln}{penColor2!20} % Color of negative area
\colorlet{fill3}{penColor3!20} % Fill
\colorlet{fill4}{penColor4!20} % Fill
\colorlet{fill5}{penColor5!20} % Fill
\colorlet{gridColor}{gray!50} % Color of grid in a plot

\newcommand{\surfaceColor}{violet}
\newcommand{\surfaceColorTwo}{redyellow}
\newcommand{\sliceColor}{greenyellow}




\pgfmathdeclarefunction{gauss}{2}{% gives gaussian
  \pgfmathparse{1/(#2*sqrt(2*pi))*exp(-((x-#1)^2)/(2*#2^2))}%
}


%%%%%%%%%%%%%
%% Vectors
%%%%%%%%%%%%%

%% Simple horiz vectors
\renewcommand{\vector}[1]{\left\langle #1\right\rangle}


%% %% Complex Horiz Vectors with angle brackets
%% \makeatletter
%% \renewcommand{\vector}[2][ , ]{\left\langle%
%%   \def\nextitem{\def\nextitem{#1}}%
%%   \@for \el:=#2\do{\nextitem\el}\right\rangle%
%% }
%% \makeatother

%% %% Vertical Vectors
%% \def\vector#1{\begin{bmatrix}\vecListA#1,,\end{bmatrix}}
%% \def\vecListA#1,{\if,#1,\else #1\cr \expandafter \vecListA \fi}

%%%%%%%%%%%%%
%% End of vectors
%%%%%%%%%%%%%

%\newcommand{\fullwidth}{}
%\newcommand{\normalwidth}{}



%% makes a snazzy t-chart for evaluating functions
%\newenvironment{tchart}{\rowcolors{2}{}{background!90!textColor}\array}{\endarray}

%%This is to help with formatting on future title pages.
\newenvironment{sectionOutcomes}{}{} 



%% Flowchart stuff
%\tikzstyle{startstop} = [rectangle, rounded corners, minimum width=3cm, minimum height=1cm,text centered, draw=black]
%\tikzstyle{question} = [rectangle, minimum width=3cm, minimum height=1cm, text centered, draw=black]
%\tikzstyle{decision} = [trapezium, trapezium left angle=70, trapezium right angle=110, minimum width=3cm, minimum height=1cm, text centered, draw=black]
%\tikzstyle{question} = [rectangle, rounded corners, minimum width=3cm, minimum height=1cm,text centered, draw=black]
%\tikzstyle{process} = [rectangle, minimum width=3cm, minimum height=1cm, text centered, draw=black]
%\tikzstyle{decision} = [trapezium, trapezium left angle=70, trapezium right angle=110, minimum width=3cm, minimum height=1cm, text centered, draw=black]
\title[Prerequisite Videos: ]{Fundamentals of Factoring}
\author{Ben Sencindiver}

\outcome{Understand common techniques in factoring, such as common terms and factoring quadratic polynomials.}

\begin{document}
\begin{abstract}
  We go through common prerequisite topics about fundamentals of factoring. 
  We look at common factors and factoring quadratic polynomials.
\end{abstract}
\maketitle

The following videos will cover topics about factoring polynomials:

\section{Fundamentals of Factoring}
%% Factoring Polynomials Warning
\textbf{Factoring Video Warning!}
\youtube{TM7k2L8_EiE}


%% Introduction and Question 1
\textbf{Introduction and Question 1: Factoring With Common Factors}
\begin{question}
%% Labeling this expandable option
\begin{flushright}
{\color{blue}(\emph{Click the arrow to the right to see the Introduction video and first question.})}
\end{flushright}
\begin{center}
\begin{expandable}
\youtube{TUbeYSyZn9c}
%% Multiple Choice Question 1
{\color{blue}(\emph{Click the arrow to the right to see the  question
posed at the end of the video.})}
\begin{expandable}
Which of the following ways does $2x^2 + 6x$ factor?
\begin{multipleChoice}
\choice[correct]{$2x(x+3)$}
\choice{$x+3$}
\choice{$(2x+1)(x+3)$}
\choice{$12x^3$}
\end{multipleChoice}
%% Example 1
\begin{flushright}
{\color{blue}(\emph{Click the arrow to the right to see an example.})}
\end{flushright}
\begin{expandable}
Example 1
\youtube{C8R35E7YSRs}\\

Checking The Answer
\youtube{T2CSjylE7tQ}
\end{expandable}
\end{expandable}
\end{expandable}
\end{center}
\end{question}


%% Question 2
\textbf{Question 2: Factoring Quadratic Polynomials 1}
\begin{question}
%% Labeling this expandable option
\begin{flushright}
{\color{blue}(\emph{Click the arrow to the right to see the second question.})}
\end{flushright}
\begin{center}
\begin{expandable}
\youtube{X0QG7o6uK-g}
%% Multiple Choice Question 2
{\color{blue}(\emph{Click the arrow to the right to see the  question
posed at the end of the video.})}
\begin{expandable}
Which of the following ways does $x^2-7x+12$ factor?
\begin{multipleChoice}
\choice{$(x-6)(x-2)$}
\choice{$(x+6)(x+2)$}
\choice{$(x+3)(x+4)$}
\choice{$(x+3)(x-4)$}
\choice{$(x-3)(x+4)$}
\choice[correct]{$(x-3)(x-4)$}
\end{multipleChoice}
%% Example 2
\begin{flushright}
{\color{blue}(\emph{Click the arrow to the right to see an example.})}
\end{flushright}
\begin{expandable}
Example 2
\youtube{HO94wtpVbOI}\\

Checking The Answer
\youtube{EoNINH_axX8}
\end{expandable}
\end{expandable}
\end{expandable}
\end{center}
\end{question}


%% Question 3
\textbf{Question 3: Factoring Quadratic Polynomials 2}
\begin{question}
%% Labeling this expandable option
\begin{flushright}
{\color{blue}(\emph{Click the arrow to the right to see the third question.})}
\end{flushright}
\begin{center}
\begin{expandable}
\youtube{CBwScC7Gvmc}
%% Multiple Choice Question 3
{\color{blue}(\emph{Click the arrow to the right to see the  question
posed at the end of the video.})}
\begin{expandable}
Which of the following ways does $x^2+16$ factor (over the real numbers)?
\begin{multipleChoice}
\choice{$(x+4)(x-4)$}
\choice{$(x-4)^2$}
\choice{$(x+4)^2$}
\choice[correct]{$x^2+16$ doesn't factor over the real numbers}
\end{multipleChoice}
%% Example 3
\begin{flushright}
{\color{blue}(\emph{Click the arrow to the right to see an example.})}
\end{flushright}
\begin{expandable}
Example 3 and Checking Our Answer
\youtube{foBo-331SW4}
\end{expandable}
\end{expandable}
\end{expandable}
\end{center}
\end{question}


\section{Advanced Factoring}


%% Question 4
\textbf{Question 4: Factoring by Grouping}
\begin{question}
%% Labeling this expandable option
\begin{flushright}
{\color{blue}(\emph{Click the arrow to the right to see the fourth question.})}
\end{flushright}
\begin{center}
\begin{expandable}
\youtube{ALZSGSwaWV4}
%% Multiple Choice Question 4
{\color{blue}(\emph{Click the arrow to the right to see the  question
posed at the end of the video.})}
\begin{expandable}
What does $3x^3+x^2-6x-2$ factor into?
\begin{multipleChoice}
\choice{You can't factor polynomials of degree 3}
\choice{This particular degree 3 polynomial doesn't factor}
\choice[correct]{$(3x+1)(x-1)(x+1)$}
\choice{$(3x+1)(x^2-1)$}
\choice{It doesn't factor}
\end{multipleChoice}
%% Example 4
\begin{flushright}
{\color{blue}(\emph{Click the arrow to the right to see an example.})}
\end{flushright}
\begin{expandable}
Example 4
\youtube{MVW8Et9cqf4}
Checking Our Answer %%% We need to upload this video
\end{expandable}
\end{expandable}
\end{expandable}
\end{center}
\end{question}


\section{Polynomial Long Division}

%% Question 5
\textbf{Introduction and Question 5: Polynomial Long Division}
\begin{question}
%% Labeling this expandable option
\begin{flushright}
{\color{blue}(\emph{Click the arrow to the right to see the fifth question.})}
\end{flushright}
\begin{center}
\begin{expandable}
\youtube{nrDNGBpl5iU}
%% Multiple Choice Question 5
{\color{blue}(\emph{Click the arrow to the right to see the  question
posed at the end of the video.})}
\begin{expandable}
How does $x^5 - 5x^3 + 4x$ factor if either $x=-1$ or $x=1$ is a root of the expression?
\begin{multipleChoice}
\choice{$x(x^4-5x^2+4)$ is the furthest the polynomial factors.}
\choice{$x(x-1)(x^4-5x^2+4)$ is the furthest the polynomial factors.}
\choice{$x(x+1)(x^4-5x^2+4)$ is the furthest the polynomial factors.}
\choice[correct]{$x(x+1)(x-1)(x+2)(x-2)$}
\choice{$x(x-1)^2(x+1)(x-4)$ is the furthest the polynomial factors.}
\choice{$x(x^2-1)^2(x^2-4)$ is the furthest the polynomial factors.}
\choice{The polynomial doesn't factor.}
\end{multipleChoice}
%% Example 5
\begin{flushright}
{\color{blue}(\emph{Click the arrow to the right to see an example.})}
\end{flushright}
\begin{expandable}
Example 5
\youtube{JeT_Ux23t7I}
\end{expandable}
\end{expandable}
\end{expandable}
\end{center}
\end{question}


\section{Synthetic Division}

%% Intro Video for Synthetic Division
\textbf{Introduction to Synthetic Division}
\begin{explanation}
{\color{blue}(\emph{Click the arrow to the right an introduction to 
synthetic division.})}
\begin{expandable}
\youtube{xo0fWyRMVzU}
\end{expandable}
\end{explanation}


%% Question 6
\textbf{Question 6: Synthetic Long Division}
\begin{question}
%% Labeling this expandable option
\begin{flushright}
{\color{blue}(\emph{Click the arrow to the right to see the sixth question.})}
\end{flushright}
\begin{center}
\begin{expandable}
\youtube{I6C_2Emuzdw}
%% Multiple Choice Question 6
{\color{blue}(\emph{Click the arrow to the right to see the  question
posed at the end of the video.})}
\begin{expandable}
Suppose you know that $x+3$ is a factor of the polynomial $f(x) = x^3 - 7x + 6$. What is the result of dividing $f(x)$ by $x+3$?
\begin{multipleChoice}
\choice[correct]{$x^2-3x+2$}
\choice{$x^3+3x^2+2x+12$}
\choice{$x^2 - 7x + 2$}
\choice{$x^2-5$}
\choice{$x^2 + 3x + 2$}
\choice{The result is not a polynomial.}
\end{multipleChoice}
%% Example 6
\begin{flushright}
{\color{blue}(\emph{Click the arrow to the right to see an example.})}
\end{flushright}
\begin{expandable}
Example 6
\youtube{pQim1QZ1KJw}
\end{expandable}
\end{expandable}
\end{expandable}
\end{center}
\end{question}


%% Question 7
\textbf{Question 7: Imaginary Roots}
\begin{question}
%% Labeling this expandable option
\begin{flushright}
{\color{blue}(\emph{Click the arrow to the right to see the seventh question.})}
\end{flushright}
\begin{center}
\begin{expandable}
\youtube{tYXKlNEZ2gY}
%% Multiple Choice Question 7
{\color{blue}(\emph{Click the arrow to the right to see the  question
posed at the end of the video.})}
\begin{expandable}
What are all of the imaginary roots of $x^3 - 3x^2 + 6x - 4$?
\begin{multipleChoice}
\choice{$x=-1$, $x=1+\sqrt{3}i$, and $x= 1-\sqrt{3}i$}
\choice[correct]{$x=1$, $x=1+\sqrt{3}i$, and $x= 1-\sqrt{3}i$}
\choice{$x=-1$, $x=1+\sqrt{3}i$, and $x= -1-\sqrt{3}i$}
\choice{The polynomial has no roots.}
\choice{$x=1$ is the only root of the polynomial}
\choice{$x=-1$ is the only root of the polynomial}
\end{multipleChoice}
%%
\begin{hint}
The graph of the function in the video (when the question is posed)
may have some useful information.
\end{hint}
\begin{hint}
If you know one root of the polynomial, synthetic division may be useful.
\end{hint}
%% Example 7
\begin{flushright}
{\color{blue}(\emph{Click the arrow to the right to see an example.})}
\end{flushright}
\begin{expandable}
Example 7
\youtube{0QRCPb6fbTI}
Checking Our Answer %%% We need to upload this video
\end{expandable}
\end{expandable}
\end{expandable}
\end{center}
\end{question}


\section{Wrap-up of Factoring Polynomials}

\begin{expandable}
Wrap-up
\youtube{OFAKAo7aemI}
\end{expandable}











\end{document}