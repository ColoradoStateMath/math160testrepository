\documentclass{ximera}

%\usepackage{todonotes}

\newcommand{\todo}{}

\usepackage{esint} % for \oiint
\graphicspath{
{./}
{functionsOfSeveralVariables/}
{normalVectors/}
{lagrangeMultipliers/}
{vectorFields/}
{greensTheorem/}
{shapeOfThingsToCome/}
}


\usepackage{tkz-euclide}
\tikzset{>=stealth} %% cool arrow head
\tikzset{shorten <>/.style={ shorten >=#1, shorten <=#1 } } %% allows shorter vectors

\usetikzlibrary{backgrounds} %% for boxes around graphs
\usetikzlibrary{shapes,positioning}  %% Clouds and stars
\usetikzlibrary{matrix} %% for matrix
\usepgfplotslibrary{polar} %% for polar plots
\usetkzobj{all}
\usepackage[makeroom]{cancel} %% for strike outs
%\usepackage{mathtools} %% for pretty underbrace % Breaks Ximera
\usepackage{multicol}
\usepackage{pgffor} %% required for integral for loops


%% http://tex.stackexchange.com/questions/66490/drawing-a-tikz-arc-specifying-the-center
%% Draws beach ball
\tikzset{pics/carc/.style args={#1:#2:#3}{code={\draw[pic actions] (#1:#3) arc(#1:#2:#3);}}}



\usepackage{array}
\setlength{\extrarowheight}{+.1cm}   
\newdimen\digitwidth
\settowidth\digitwidth{9}
\def\divrule#1#2{
\noalign{\moveright#1\digitwidth
\vbox{\hrule width#2\digitwidth}}}





\newcommand{\RR}{\mathbb R}
\newcommand{\R}{\mathbb R}
\newcommand{\N}{\mathbb N}
\newcommand{\Z}{\mathbb Z}

%\newcommand{\sage}{\textsf{SageMath}}


%\renewcommand{\d}{\,d\!}
\renewcommand{\d}{\mathop{}\!d}
\newcommand{\dd}[2][]{\frac{\d #1}{\d #2}}
\newcommand{\pp}[2][]{\frac{\partial #1}{\partial #2}}
\renewcommand{\l}{\ell}
\newcommand{\ddx}{\frac{d}{\d x}}

\newcommand{\zeroOverZero}{\ensuremath{\boldsymbol{\tfrac{0}{0}}}}
\newcommand{\inftyOverInfty}{\ensuremath{\boldsymbol{\tfrac{\infty}{\infty}}}}
\newcommand{\zeroOverInfty}{\ensuremath{\boldsymbol{\tfrac{0}{\infty}}}}
\newcommand{\zeroTimesInfty}{\ensuremath{\small\boldsymbol{0\cdot \infty}}}
\newcommand{\inftyMinusInfty}{\ensuremath{\small\boldsymbol{\infty - \infty}}}
\newcommand{\oneToInfty}{\ensuremath{\boldsymbol{1^\infty}}}
\newcommand{\zeroToZero}{\ensuremath{\boldsymbol{0^0}}}
\newcommand{\inftyToZero}{\ensuremath{\boldsymbol{\infty^0}}}



\newcommand{\numOverZero}{\ensuremath{\boldsymbol{\tfrac{\#}{0}}}}
\newcommand{\dfn}{\textbf}
%\newcommand{\unit}{\,\mathrm}
\newcommand{\unit}{\mathop{}\!\mathrm}
\newcommand{\eval}[1]{\bigg[ #1 \bigg]}
\newcommand{\seq}[1]{\left( #1 \right)}
\renewcommand{\epsilon}{\varepsilon}
\renewcommand{\phi}{\varphi}


\renewcommand{\iff}{\Leftrightarrow}

\DeclareMathOperator{\arccot}{arccot}
\DeclareMathOperator{\arcsec}{arcsec}
\DeclareMathOperator{\arccsc}{arccsc}
\DeclareMathOperator{\si}{Si}
\DeclareMathOperator{\proj}{\vec{proj}}
\DeclareMathOperator{\scal}{scal}
\DeclareMathOperator{\sign}{sign}


%% \newcommand{\tightoverset}[2]{% for arrow vec
%%   \mathop{#2}\limits^{\vbox to -.5ex{\kern-0.75ex\hbox{$#1$}\vss}}}
\newcommand{\arrowvec}{\overrightarrow}
%\renewcommand{\vec}[1]{\arrowvec{\mathbf{#1}}}
\renewcommand{\vec}{\mathbf}
\newcommand{\veci}{{\boldsymbol{\hat{\imath}}}}
\newcommand{\vecj}{{\boldsymbol{\hat{\jmath}}}}
\newcommand{\veck}{{\boldsymbol{\hat{k}}}}
\newcommand{\vecl}{\boldsymbol{\l}}
\newcommand{\uvec}[1]{\mathbf{\hat{#1}}}
\newcommand{\utan}{\mathbf{\hat{t}}}
\newcommand{\unormal}{\mathbf{\hat{n}}}
\newcommand{\ubinormal}{\mathbf{\hat{b}}}

\newcommand{\dotp}{\bullet}
\newcommand{\cross}{\boldsymbol\times}
\newcommand{\grad}{\boldsymbol\nabla}
\newcommand{\divergence}{\grad\dotp}
\newcommand{\curl}{\grad\cross}
%\DeclareMathOperator{\divergence}{divergence}
%\DeclareMathOperator{\curl}[1]{\grad\cross #1}
\newcommand{\lto}{\mathop{\longrightarrow\,}\limits}

\renewcommand{\bar}{\overline}

\colorlet{textColor}{black} 
\colorlet{background}{white}
\colorlet{penColor}{blue!50!black} % Color of a curve in a plot
\colorlet{penColor2}{red!50!black}% Color of a curve in a plot
\colorlet{penColor3}{red!50!blue} % Color of a curve in a plot
\colorlet{penColor4}{green!50!black} % Color of a curve in a plot
\colorlet{penColor5}{orange!80!black} % Color of a curve in a plot
\colorlet{penColor6}{yellow!70!black} % Color of a curve in a plot
\colorlet{fill1}{penColor!20} % Color of fill in a plot
\colorlet{fill2}{penColor2!20} % Color of fill in a plot
\colorlet{fillp}{fill1} % Color of positive area
\colorlet{filln}{penColor2!20} % Color of negative area
\colorlet{fill3}{penColor3!20} % Fill
\colorlet{fill4}{penColor4!20} % Fill
\colorlet{fill5}{penColor5!20} % Fill
\colorlet{gridColor}{gray!50} % Color of grid in a plot

\newcommand{\surfaceColor}{violet}
\newcommand{\surfaceColorTwo}{redyellow}
\newcommand{\sliceColor}{greenyellow}




\pgfmathdeclarefunction{gauss}{2}{% gives gaussian
  \pgfmathparse{1/(#2*sqrt(2*pi))*exp(-((x-#1)^2)/(2*#2^2))}%
}


%%%%%%%%%%%%%
%% Vectors
%%%%%%%%%%%%%

%% Simple horiz vectors
\renewcommand{\vector}[1]{\left\langle #1\right\rangle}


%% %% Complex Horiz Vectors with angle brackets
%% \makeatletter
%% \renewcommand{\vector}[2][ , ]{\left\langle%
%%   \def\nextitem{\def\nextitem{#1}}%
%%   \@for \el:=#2\do{\nextitem\el}\right\rangle%
%% }
%% \makeatother

%% %% Vertical Vectors
%% \def\vector#1{\begin{bmatrix}\vecListA#1,,\end{bmatrix}}
%% \def\vecListA#1,{\if,#1,\else #1\cr \expandafter \vecListA \fi}

%%%%%%%%%%%%%
%% End of vectors
%%%%%%%%%%%%%

%\newcommand{\fullwidth}{}
%\newcommand{\normalwidth}{}



%% makes a snazzy t-chart for evaluating functions
%\newenvironment{tchart}{\rowcolors{2}{}{background!90!textColor}\array}{\endarray}

%%This is to help with formatting on future title pages.
\newenvironment{sectionOutcomes}{}{} 



%% Flowchart stuff
%\tikzstyle{startstop} = [rectangle, rounded corners, minimum width=3cm, minimum height=1cm,text centered, draw=black]
%\tikzstyle{question} = [rectangle, minimum width=3cm, minimum height=1cm, text centered, draw=black]
%\tikzstyle{decision} = [trapezium, trapezium left angle=70, trapezium right angle=110, minimum width=3cm, minimum height=1cm, text centered, draw=black]
%\tikzstyle{question} = [rectangle, rounded corners, minimum width=3cm, minimum height=1cm,text centered, draw=black]
%\tikzstyle{process} = [rectangle, minimum width=3cm, minimum height=1cm, text centered, draw=black]
%\tikzstyle{decision} = [trapezium, trapezium left angle=70, trapezium right angle=110, minimum width=3cm, minimum height=1cm, text centered, draw=black]

\title[Dig-In:]{The Work-Energy Theorem}

\begin{document}
\begin{abstract}
  Substitution is given a physical meaning. 
\end{abstract}
\maketitle

In physics, we take measurable quantities from the real world, and
attempt to find meaningful relationships between them. A basic example
of this would be the physical ideal of \dfn{force}. Force applied to an
object changes the motion of an object. Here's the deal though, at a
basic level
\[
\mathrm{force} = \mathrm{mass} \cdot \mathrm{acceleration}.
\]
and while we can put a physical interpretation to this arithmetical
definition, at the end of the day force is simply ``mass times
acceleration.'' The SI unit of force is a \dfn{newton}, which is defined to be
\[
1\unit{N} = 1\unit{kg}\cdot \unit{m}/\unit{s}^2. 
\]
\begin{problem}
  To get a feel for what a newton is, consider this: if an apple has a
  mass of $0.1\unit{kg}$, what force would an apple exert on your hand
  due to the acceleration due to gravity?
  \begin{prompt}
  \[
  F = (0.1) \cdot (-9.8) = \answer{-0.98}\unit{N},
  \]
  \end{prompt}
\begin{feedback}
  Hence the ``weight'' of an apple is approximately $1\unit{N}$.
\end{feedback}
\end{problem}


In a similar way, the idea of \dfn{kinetic energy}, is ``energy''
objects have from motion. It is defined by the formula
\[
E_k = \frac{m \cdot v^2}{2}.
\]
The SI unit of energy is a \dfn{joule}, which is defined to be
\[
1\unit{J} = 1\unit{kg}\cdot \unit{m}^2/\unit{s}^2 = 1\unit{N}\cdot\unit{m}. 
\]
To get a feel for the ``size'' of a joule, consider this: if an apple
has a mass of $0.1\unit{kg}$ and it is dropped from a height of
$1\unit{m}$, then approximately $1$ joule of energy is released when
it hits the ground. Let's see if we can explain why this is true.


\begin{example}
  If an apple has a mass of $0.1\unit{kg}$ and it is dropped from a
  height of $1\unit{m}$, how much energy is released when it hits the
  ground? Assume that the acceleration due to gravity is
  $-9.8\unit{m}/\unit{s}^2$.
  \begin{explanation}
    First we need to find the velocity at which the apple hits the
    ground. Let $a(t)=-9.8$ represent acceleration at time $t$ and $v(t)$
    represent velocity. Since $v(0) = 0$, we know that
    \begin{align*}
      v(t) &= \int_0^t a(x) \d x\\
      &= \int_0^t \answer[given]{-9.8} \d x \\
      &= \answer[given]{-9.8t}.
    \end{align*}
    Now we need to know how long it takes for the apple to hit the
    ground, after being dropped from a height of $1$ meter. For this
    we'll need a formula for position. Here $s(0) = 1$, so we'll need
    to use an indefinite integral:
    \begin{align*}
      s(t)&=\int v(t) \d t\\
      &= \int \answer[given]{-9.8t} \d t \\
      &= \answer[given]{\frac{-9.8 t^2}{2}} + C
    \end{align*}
    Since $s(0)= 1$, write with me
    \[
    s(0) = \frac{-9.8 \cdot 0}{2} + C = \answer[given]{1},
    \]
    hence $C =1$ and $s(t) = \frac{-9.8 t^2}{2}+1$. Solving the equation
    \[
    s(t) = 0
    \]
    for $t$ tells us the time the apple hits the ground. Write with me
    \begin{align*}
      s(t) &= 0 \\
      \frac{-9.8 t^2}{2} +1 &= 0\\
      -9.8 t^2 &=-2\\
      t^2 &=\frac{2}{9.8}\\
      t &=\sqrt{\frac{2}{9.8}}.
    \end{align*}
    So the apple hits the ground after $\sqrt{\frac{2}{9.8}}$ seconds.
    Finally, the formula for kinetic energy is
    \begin{align*}
    E_k &= \frac{m \cdot v^2}{2} \\
    &= \frac{0.1 \cdot \left(a\cdot t\right)^2}{2} \\
    &= \frac{0.1 \cdot \left((-9.8)\cdot \sqrt{\frac{2}{9.8}}\right)^2}{2} \\
    &= \frac{0.1 \cdot (9.8)^2\cdot\frac{2}{9.8}}{2} \\
    &= \frac{0.1 \cdot 9.8\cdot 2}{2} \\
    &= 0.98.
    \end{align*}
    Ah! So the kinetic energy released by an apple dropped from a
    height of $1$ meter is approximately $1$ joule.
  \end{explanation}
\end{example}


Finally \dfn{work} is defined to be accumulated force over a
distance. Note, there must be some force \textit{in the direction} (or
opposite direction) that the object is moving for it to be considered
\textit{work}.

\begin{problem}
  Which of the following are examples where work of this kind is being done?
  \begin{selectAll}
    \choice{studying calculus}
    \choice[correct]{a car applying breaks to come to a stop over a distance of $100\unit{ft}$}
    \choice[correct]{a young mathematician climbing a mountain}
    \choice{a young mathematician standing still, holding a $1000$ page calculus book for $10$ minutes}
    \choice{a young mathematician walking around with a $1000$ page calculus book}
    \choice[correct]{a young mathematician picking up a $1000$ page calculus book}
  \end{selectAll}
  \begin{feedback}
    While studying calculus may ``feel'' like work, it is not
    (typically) an example of an accumulated force over a distance,
    and hence no work is done.

    On the other hand, a car applying breaks is a change in motion, and
    hence a force is applied. Since this force is applied over a
    distance, work is done.

    Climbing a mountain is also an example of work, as one is applying
    force to overcome the acceleration due to gravity, over the
    distance that one is climbing.

    No work is done when holding a calculus book, as there is no
    accumulated force over a distance.

    It is also the case that no work is done when one walks around
    with a calculus book, this is because the ``force'' is in a
    direction perpendicular to the motion.

    Finally, when one picks up a calculus book, you are moving the
    book against the force due to the acceleration due to
    gravity. Hence work is done.
  \end{feedback}
\end{problem}
We can write the definition of work in the language of calculus as,
\[
W = \int_{s_0}^{s_1} F(s) \d s.
\]
The SI unit of work is also a \dfn{joule}. To help understand this,
$1$ joule is approximately how much work is done when you raise an
apple one meter.

Let's again see why this is true.
\begin{example}
  If an apple has a mass of $0.1\unit{kg}$, how much work is required
  to lift this apple $1$ meter?  Assume that the acceleration due to
  gravity is $-9.8\unit{m}/\unit{s}^2$.
  \begin{explanation}
    Well, work is computed by
    \[
    W = \int_{s_0}^{s_1} F(s) \d s.
    \]
    Since force is mass times acceleration,
    \begin{align*}
      F(s) &= 0.1\cdot \answer[given]{(-9.8)} \\
      &= \answer[given]{-0.98}.
    \end{align*}
    So, our integral becomes
    \begin{align*}
      \int_{0}^{1} \answer[given]{-0.98} \d s &= \eval{\answer[given]{-0.98 s}}_0^1\\
      &=\answer[given]{-0.98}.
    \end{align*}
    Ah! So when lifting an apple $1$ meter, requires $\answer[given]{-0.98}$ joules of
    work. The sign is negative since we are lifting \textbf{against}
    the gravitational force.
  \end{explanation}
\end{example}



Now we have a question:
\begin{quote}
 \textbf{Why do work and kinetic energy have the same units?}
\end{quote}

One way to answer this is via the \textit{Work-Energy Theorem}.

\begin{theorem}[Work-Energy Theorem]\index{Work-Energy Theorem}
  Suppose that an object of mass $m$ is moving along a straight
  line. If $s_0$ and $s_1$ are the the starting and ending positions,
  $v_0$ and $v_1$ are the the starting and ending velocities, and
  $F(s)$ is the force acting on the object for any given position,
  then
  \[
  W = \int_{s_0}^{s_1} F(s) \d s = \frac{m\cdot v_1^2}{2} - \frac{m\cdot v_0^2}{2}.
  \]
  \begin{explanation}
    First we need to get all of our symbolism out in the open. Let:
    \begin{itemize}
    \item $s(t)$ represent position with respect to time,
    \item $v(t)$ represent velocity with respect to time,
    \item $a(s)$ represent acceleration with respect to position,
    \item $t_0$ represent the starting time,
    \item $t_1$ represent the ending time,
    \end{itemize}
    then we also have that
    \begin{itemize}
    \item $s(t_0)$ represents the starting position, $s_0$,
    \item $s(t_1)$ represents the ending position, $s_1$,
    \item $v(t_0)$ represents the starting velocity, $v_0$,
    \item $v(t_1)$ represents the ending velocity, $v_1$.
    \end{itemize}
    Now write with me,
    \[
    W = \int_{s_0}^{s_1} F(s) \d s = \int_{s(t_0)}^{s(t_1)} F(s) \d s
    \]
    here we are working with functions of distance. We will use the
    substitution formula,
    \begin{image}
    \begin{tikzpicture}[scale=1,every node/.style={transform shape}]
      \draw [->, line width=10, penColor!10!background] (1,0)--(-0.5,0);
    \node at (0,0) {
      $\int_a^b f'(g(x)) g'(x) \d x =\int_{g(a)}^{g(b)} f'(g) \d g$
    };
    \end{tikzpicture}
  \end{image} 
    transforming from right to left, to see that
    \[
    \int_{s(t_0)}^{s(t_1)} F(s) \d s = \int_{t_0}^{t_1} F(s(t)) s'(t) \d t
    \]
    and we are now working with functions of time. Since $s'(t) =
    v(t)$, we may write
    \[
    \int_{t_0}^{t_1} F(s(t)) s'(t) \d t = \int_{t_0}^{t_1} F(s(t)) v(t) \d t
    \]
    and now remember that $F=m\cdot a$, so
    \[
    \int_{t_0}^{t_1} F(s(t)) v(t) \d t = \int_{t_0}^{t_1} m\cdot a(s(t)) v(t) \d t.
    \]
    However, $a(s(t)) = v'(t)$, so rearranging we have,
    \[
    \int_{t_0}^{t_1} m\cdot a(s(t)) v(t) \d t = m\cdot \int_{t_0}^{t_1} v(t) v'(t)\d t.
    \]
    Now we apply the substitution formula again, this time we will transform left to right
    \begin{image}
    \begin{tikzpicture}[scale=1,every node/.style={transform shape}]
      \draw [->, line width=10, penColor!10!background] (-.5,0)--(1,0);
      \node at (0,0) {
        $\int_a^b f'(g(x)) g'(x) \d x =\int_{g(a)}^{g(b)} f'(g) \d g$
      };
    \end{tikzpicture}
    \end{image}
    and so we see
    \[
    m\cdot \int_{t_0}^{t_1} v(t) v'(t)\d t = m\cdot \int_{v(t_0)}^{v(t_1)} v \d v
    \]
    and we are working with functions of velocity.  At last, setting
    $v(t_0) = v_0$ and $v(t_1) = v_1$, we can evaluate this integral,
    \begin{align*}
      m \cdot \int_{v_0}^{v_1} v \d v &= m \cdot \eval{\frac{v^2}{2}}_{v_0}^{v_1}\\
      &=\frac{m\cdot v_1^2}{2} - \frac{m\cdot v_0^2}{2}.
    \end{align*}
  \end{explanation}
\end{theorem}

The Work-Energy theorem says that:
\begin{image}
  \begin{tikzpicture}[scale=2,every node/.style={transform shape}]
    \node at (0,0) {
      $\color{green!70!black!70!blue}\int_{s_0}^{s_1}\color{purple!50!blue!90!black}F(s) \color{green!70!black!70!blue}\d s\color{black} = \color{blue!70!green}\frac{m\cdot v_1^2}{2} - \frac{m\cdot v_0^2}{2}$
      };
  \end{tikzpicture}
\end{image}
This could be interpreted as:
\begin{quote}\large\textbf{The \textcolor{green!70!black!70!blue}{accumulated} \textcolor{purple!50!blue!90!black}{force} \textcolor{green!70!black!70!blue}{over distance} is the \textcolor{blue!70!green}{change in kinetic energy}.}
\end{quote}
Moreover, this answers our initial question of why work and kinetic
energy have the same units.  In essence, energy powers work.


\end{document}
      
