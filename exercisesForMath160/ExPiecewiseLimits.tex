\documentclass[handout]{ximera}
%\documentclass[10pt,handout,twocolumn,twoside,wordchoicegiven]{xercises}
%\documentclass[10pt,handout,twocolumn,twoside,wordchoicegiven]{xourse}

%\author{Steven Gubkin}
%\license{Creative Commons 3.0 By-NC}
%\usepackage{todonotes}

\newcommand{\todo}{}

\usepackage{esint} % for \oiint
\graphicspath{
{./}
{functionsOfSeveralVariables/}
{normalVectors/}
{lagrangeMultipliers/}
{vectorFields/}
{greensTheorem/}
{shapeOfThingsToCome/}
}


\usepackage{tkz-euclide}
\tikzset{>=stealth} %% cool arrow head
\tikzset{shorten <>/.style={ shorten >=#1, shorten <=#1 } } %% allows shorter vectors

\usetikzlibrary{backgrounds} %% for boxes around graphs
\usetikzlibrary{shapes,positioning}  %% Clouds and stars
\usetikzlibrary{matrix} %% for matrix
\usepgfplotslibrary{polar} %% for polar plots
\usetkzobj{all}
\usepackage[makeroom]{cancel} %% for strike outs
%\usepackage{mathtools} %% for pretty underbrace % Breaks Ximera
\usepackage{multicol}
\usepackage{pgffor} %% required for integral for loops


%% http://tex.stackexchange.com/questions/66490/drawing-a-tikz-arc-specifying-the-center
%% Draws beach ball
\tikzset{pics/carc/.style args={#1:#2:#3}{code={\draw[pic actions] (#1:#3) arc(#1:#2:#3);}}}



\usepackage{array}
\setlength{\extrarowheight}{+.1cm}   
\newdimen\digitwidth
\settowidth\digitwidth{9}
\def\divrule#1#2{
\noalign{\moveright#1\digitwidth
\vbox{\hrule width#2\digitwidth}}}





\newcommand{\RR}{\mathbb R}
\newcommand{\R}{\mathbb R}
\newcommand{\N}{\mathbb N}
\newcommand{\Z}{\mathbb Z}

%\newcommand{\sage}{\textsf{SageMath}}


%\renewcommand{\d}{\,d\!}
\renewcommand{\d}{\mathop{}\!d}
\newcommand{\dd}[2][]{\frac{\d #1}{\d #2}}
\newcommand{\pp}[2][]{\frac{\partial #1}{\partial #2}}
\renewcommand{\l}{\ell}
\newcommand{\ddx}{\frac{d}{\d x}}

\newcommand{\zeroOverZero}{\ensuremath{\boldsymbol{\tfrac{0}{0}}}}
\newcommand{\inftyOverInfty}{\ensuremath{\boldsymbol{\tfrac{\infty}{\infty}}}}
\newcommand{\zeroOverInfty}{\ensuremath{\boldsymbol{\tfrac{0}{\infty}}}}
\newcommand{\zeroTimesInfty}{\ensuremath{\small\boldsymbol{0\cdot \infty}}}
\newcommand{\inftyMinusInfty}{\ensuremath{\small\boldsymbol{\infty - \infty}}}
\newcommand{\oneToInfty}{\ensuremath{\boldsymbol{1^\infty}}}
\newcommand{\zeroToZero}{\ensuremath{\boldsymbol{0^0}}}
\newcommand{\inftyToZero}{\ensuremath{\boldsymbol{\infty^0}}}



\newcommand{\numOverZero}{\ensuremath{\boldsymbol{\tfrac{\#}{0}}}}
\newcommand{\dfn}{\textbf}
%\newcommand{\unit}{\,\mathrm}
\newcommand{\unit}{\mathop{}\!\mathrm}
\newcommand{\eval}[1]{\bigg[ #1 \bigg]}
\newcommand{\seq}[1]{\left( #1 \right)}
\renewcommand{\epsilon}{\varepsilon}
\renewcommand{\phi}{\varphi}


\renewcommand{\iff}{\Leftrightarrow}

\DeclareMathOperator{\arccot}{arccot}
\DeclareMathOperator{\arcsec}{arcsec}
\DeclareMathOperator{\arccsc}{arccsc}
\DeclareMathOperator{\si}{Si}
\DeclareMathOperator{\proj}{\vec{proj}}
\DeclareMathOperator{\scal}{scal}
\DeclareMathOperator{\sign}{sign}


%% \newcommand{\tightoverset}[2]{% for arrow vec
%%   \mathop{#2}\limits^{\vbox to -.5ex{\kern-0.75ex\hbox{$#1$}\vss}}}
\newcommand{\arrowvec}{\overrightarrow}
%\renewcommand{\vec}[1]{\arrowvec{\mathbf{#1}}}
\renewcommand{\vec}{\mathbf}
\newcommand{\veci}{{\boldsymbol{\hat{\imath}}}}
\newcommand{\vecj}{{\boldsymbol{\hat{\jmath}}}}
\newcommand{\veck}{{\boldsymbol{\hat{k}}}}
\newcommand{\vecl}{\boldsymbol{\l}}
\newcommand{\uvec}[1]{\mathbf{\hat{#1}}}
\newcommand{\utan}{\mathbf{\hat{t}}}
\newcommand{\unormal}{\mathbf{\hat{n}}}
\newcommand{\ubinormal}{\mathbf{\hat{b}}}

\newcommand{\dotp}{\bullet}
\newcommand{\cross}{\boldsymbol\times}
\newcommand{\grad}{\boldsymbol\nabla}
\newcommand{\divergence}{\grad\dotp}
\newcommand{\curl}{\grad\cross}
%\DeclareMathOperator{\divergence}{divergence}
%\DeclareMathOperator{\curl}[1]{\grad\cross #1}
\newcommand{\lto}{\mathop{\longrightarrow\,}\limits}

\renewcommand{\bar}{\overline}

\colorlet{textColor}{black} 
\colorlet{background}{white}
\colorlet{penColor}{blue!50!black} % Color of a curve in a plot
\colorlet{penColor2}{red!50!black}% Color of a curve in a plot
\colorlet{penColor3}{red!50!blue} % Color of a curve in a plot
\colorlet{penColor4}{green!50!black} % Color of a curve in a plot
\colorlet{penColor5}{orange!80!black} % Color of a curve in a plot
\colorlet{penColor6}{yellow!70!black} % Color of a curve in a plot
\colorlet{fill1}{penColor!20} % Color of fill in a plot
\colorlet{fill2}{penColor2!20} % Color of fill in a plot
\colorlet{fillp}{fill1} % Color of positive area
\colorlet{filln}{penColor2!20} % Color of negative area
\colorlet{fill3}{penColor3!20} % Fill
\colorlet{fill4}{penColor4!20} % Fill
\colorlet{fill5}{penColor5!20} % Fill
\colorlet{gridColor}{gray!50} % Color of grid in a plot

\newcommand{\surfaceColor}{violet}
\newcommand{\surfaceColorTwo}{redyellow}
\newcommand{\sliceColor}{greenyellow}




\pgfmathdeclarefunction{gauss}{2}{% gives gaussian
  \pgfmathparse{1/(#2*sqrt(2*pi))*exp(-((x-#1)^2)/(2*#2^2))}%
}


%%%%%%%%%%%%%
%% Vectors
%%%%%%%%%%%%%

%% Simple horiz vectors
\renewcommand{\vector}[1]{\left\langle #1\right\rangle}


%% %% Complex Horiz Vectors with angle brackets
%% \makeatletter
%% \renewcommand{\vector}[2][ , ]{\left\langle%
%%   \def\nextitem{\def\nextitem{#1}}%
%%   \@for \el:=#2\do{\nextitem\el}\right\rangle%
%% }
%% \makeatother

%% %% Vertical Vectors
%% \def\vector#1{\begin{bmatrix}\vecListA#1,,\end{bmatrix}}
%% \def\vecListA#1,{\if,#1,\else #1\cr \expandafter \vecListA \fi}

%%%%%%%%%%%%%
%% End of vectors
%%%%%%%%%%%%%

%\newcommand{\fullwidth}{}
%\newcommand{\normalwidth}{}



%% makes a snazzy t-chart for evaluating functions
%\newenvironment{tchart}{\rowcolors{2}{}{background!90!textColor}\array}{\endarray}

%%This is to help with formatting on future title pages.
\newenvironment{sectionOutcomes}{}{} 



%% Flowchart stuff
%\tikzstyle{startstop} = [rectangle, rounded corners, minimum width=3cm, minimum height=1cm,text centered, draw=black]
%\tikzstyle{question} = [rectangle, minimum width=3cm, minimum height=1cm, text centered, draw=black]
%\tikzstyle{decision} = [trapezium, trapezium left angle=70, trapezium right angle=110, minimum width=3cm, minimum height=1cm, text centered, draw=black]
%\tikzstyle{question} = [rectangle, rounded corners, minimum width=3cm, minimum height=1cm,text centered, draw=black]
%\tikzstyle{process} = [rectangle, minimum width=3cm, minimum height=1cm, text centered, draw=black]
%\tikzstyle{decision} = [trapezium, trapezium left angle=70, trapezium right angle=110, minimum width=3cm, minimum height=1cm, text centered, draw=black]

\outcome{Practice limits with piece-wise functions.}
   

\title[Exercises:]{Limits of piece-wise functions practice}

\begin{document}
\begin{abstract}
Here is an opportunity for you to practice finding one- and two-sided limits of piece-wise functions.
\end{abstract}
\maketitle

%%Problem 1
\begin{exercise}
Let
\[
f(x) = \begin{cases}
  \frac{x^3 - 8}{x-2}  &\text{if $x<1$,} \\
  x^3+1 &\text{if  $x>1$.}
\end{cases}
\]
Does $\displaystyle\lim_{x \to 2} f(x)$ exist?  If it does, give its value.
Otherwise write DNE.

\[
\lim_{x \to 2} f(x) = \answer{9}
\]

\begin{hint}
When $x$ is close to 2, what is the rule for $f(x)$?
\end{hint}

\end{exercise}

%%Problem 2 - Thought up by Brady
\begin{exercise}
Let
\[
g(x) = \begin{cases}
  \frac{x^2+5x}{x}  & x<0 \\
  1 & \leq x \leq 1 \\
  \frac{x}{\sqrt{x-1} +1} & x>1
\end{cases}
\]
Use $g(x)$ to evaluate the following limits if they exist.  Otherwise, write DNE.

\begin{itemize}

\begin{multicols}{2}

\item [] $\lim_{x \to -2} g(x) = \answer{3}$

\item [] $\lim_{x \to 0^-} g(x) = \answer{5}$

\item [] $\lim_{x \to 0^+} g(x) = \answer{1}$

\item [] $\lim_{x \to 0} g(x) = \answer{DNE}$

\item [] $\lim_{x \to 1^-} g(x) = \answer{1}$

\item [] $\lim_{x \to 1^+} g(x) = \answer{1}$

\item [] $\lim_{x \to 1} g(x) = \answer{1}$

\item [] $\lim_{x \to 5} g(x) = \answer{\frac{5}{3}}$

\end{multicols}

\end{itemize}

\begin{hint}

To evaluate the 2-sided limits, you will first need to evaluate the corresponding 1-sided limits. 

\end{hint}

\end{exercise}

%%Problem 3
\begin{exercise}
Let $S(x) = \frac{|x|}{x}$.  Does $\displaystyle\lim_{x \to -4} S(x)$ exist?  If it
does, give its value.  Otherwise write DNE.

\[
\lim_{x \to -4} S(x) = \answer{-1}
\] 

\begin{hint}
When $x$ is close to $x=-4$, what is $|x|$ equal to? 
\end{hint}

\end{exercise}

%% Problem 4
\begin{exercise}
Let $f(t) = \frac{t^2 - 12t +35}{|t-7|}$.  Use $f(t)$ to evaluate the following limits if they exist.  Otherwise, write DNE.

\begin{itemize}

\item [] $\lim_{x \to 7^-} f(t) = \answer{-2}$

\item [] $\lim_{x \to 7^+} f(t) = \answer{2}$

\item [] $\lim_{x \to 7} f(t) = \answer{DNE}$

\end{itemize}

\begin{hint}
Absolute value functions are piece-wise functions, so you may want to re-write $f(t)$ as an explicit piece-wise function before trying to evaluate the limits. 
\end{hint}

\end{exercise}

%% Problem 5
\begin{exercise}
Let $K(x) = \frac{x^2}{|x|}$.  Use $K(x)$ to evaluate the following limit if it exists.  Otherwise, write DNE.

\[
\lim_{x \to 0} K(x) = \answer{0}
\]

\end{exercise}

%%Problem 6
\begin{exercise}
Let
\[
f(x) =
\begin{cases} x^2+55 &\text{if $x<3$,}\\
  0 &\text{if $x=3$,} \\
  b^x &\text{if $x>3$.}
\end{cases}
\]  
What must the be the value of $b$ to make $\displaystyle\lim_{x \to 3} f(x)$ exist?

\[
b = \answer{4}
\]

\begin{hint}
  The left- and right-hand limits at $x=3$ must be equal in order for $\displaystyle\lim_{x \to 3} f(x)$ to exist.  Use this information to
  set up an equation in terms of $b$, and then solve for $b$.
\end{hint}
\end{exercise}

%%Problem 7
\begin{exercise}
Let
\[
f(x) = \begin{cases}
  |x| &\text{if $x<1$,} \\
  \frac{x^2-a^2}{x-a} &\text{if $x>1$.}
\end{cases}
\]
If $\displaystyle\lim_{x \to 1} f(x)$ exists, what is $a$?

  \[
a = \answer{0}
\]

\end{exercise}





















\end{document}
