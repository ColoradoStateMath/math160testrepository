\documentclass[handout]{ximera}
%\documentclass[10pt,handout,twocolumn,twoside,wordchoicegiven]{xercises}
%\documentclass[10pt,handout,twocolumn,twoside,wordchoicegiven]{xourse}

%\author{Steven Gubkin}
%\license{Creative Commons 3.0 By-NC}
%\usepackage{todonotes}

\newcommand{\todo}{}

\usepackage{esint} % for \oiint
\graphicspath{
{./}
{functionsOfSeveralVariables/}
{normalVectors/}
{lagrangeMultipliers/}
{vectorFields/}
{greensTheorem/}
{shapeOfThingsToCome/}
}


\usepackage{tkz-euclide}
\tikzset{>=stealth} %% cool arrow head
\tikzset{shorten <>/.style={ shorten >=#1, shorten <=#1 } } %% allows shorter vectors

\usetikzlibrary{backgrounds} %% for boxes around graphs
\usetikzlibrary{shapes,positioning}  %% Clouds and stars
\usetikzlibrary{matrix} %% for matrix
\usepgfplotslibrary{polar} %% for polar plots
\usetkzobj{all}
\usepackage[makeroom]{cancel} %% for strike outs
%\usepackage{mathtools} %% for pretty underbrace % Breaks Ximera
\usepackage{multicol}
\usepackage{pgffor} %% required for integral for loops


%% http://tex.stackexchange.com/questions/66490/drawing-a-tikz-arc-specifying-the-center
%% Draws beach ball
\tikzset{pics/carc/.style args={#1:#2:#3}{code={\draw[pic actions] (#1:#3) arc(#1:#2:#3);}}}



\usepackage{array}
\setlength{\extrarowheight}{+.1cm}   
\newdimen\digitwidth
\settowidth\digitwidth{9}
\def\divrule#1#2{
\noalign{\moveright#1\digitwidth
\vbox{\hrule width#2\digitwidth}}}





\newcommand{\RR}{\mathbb R}
\newcommand{\R}{\mathbb R}
\newcommand{\N}{\mathbb N}
\newcommand{\Z}{\mathbb Z}

%\newcommand{\sage}{\textsf{SageMath}}


%\renewcommand{\d}{\,d\!}
\renewcommand{\d}{\mathop{}\!d}
\newcommand{\dd}[2][]{\frac{\d #1}{\d #2}}
\newcommand{\pp}[2][]{\frac{\partial #1}{\partial #2}}
\renewcommand{\l}{\ell}
\newcommand{\ddx}{\frac{d}{\d x}}

\newcommand{\zeroOverZero}{\ensuremath{\boldsymbol{\tfrac{0}{0}}}}
\newcommand{\inftyOverInfty}{\ensuremath{\boldsymbol{\tfrac{\infty}{\infty}}}}
\newcommand{\zeroOverInfty}{\ensuremath{\boldsymbol{\tfrac{0}{\infty}}}}
\newcommand{\zeroTimesInfty}{\ensuremath{\small\boldsymbol{0\cdot \infty}}}
\newcommand{\inftyMinusInfty}{\ensuremath{\small\boldsymbol{\infty - \infty}}}
\newcommand{\oneToInfty}{\ensuremath{\boldsymbol{1^\infty}}}
\newcommand{\zeroToZero}{\ensuremath{\boldsymbol{0^0}}}
\newcommand{\inftyToZero}{\ensuremath{\boldsymbol{\infty^0}}}



\newcommand{\numOverZero}{\ensuremath{\boldsymbol{\tfrac{\#}{0}}}}
\newcommand{\dfn}{\textbf}
%\newcommand{\unit}{\,\mathrm}
\newcommand{\unit}{\mathop{}\!\mathrm}
\newcommand{\eval}[1]{\bigg[ #1 \bigg]}
\newcommand{\seq}[1]{\left( #1 \right)}
\renewcommand{\epsilon}{\varepsilon}
\renewcommand{\phi}{\varphi}


\renewcommand{\iff}{\Leftrightarrow}

\DeclareMathOperator{\arccot}{arccot}
\DeclareMathOperator{\arcsec}{arcsec}
\DeclareMathOperator{\arccsc}{arccsc}
\DeclareMathOperator{\si}{Si}
\DeclareMathOperator{\proj}{\vec{proj}}
\DeclareMathOperator{\scal}{scal}
\DeclareMathOperator{\sign}{sign}


%% \newcommand{\tightoverset}[2]{% for arrow vec
%%   \mathop{#2}\limits^{\vbox to -.5ex{\kern-0.75ex\hbox{$#1$}\vss}}}
\newcommand{\arrowvec}{\overrightarrow}
%\renewcommand{\vec}[1]{\arrowvec{\mathbf{#1}}}
\renewcommand{\vec}{\mathbf}
\newcommand{\veci}{{\boldsymbol{\hat{\imath}}}}
\newcommand{\vecj}{{\boldsymbol{\hat{\jmath}}}}
\newcommand{\veck}{{\boldsymbol{\hat{k}}}}
\newcommand{\vecl}{\boldsymbol{\l}}
\newcommand{\uvec}[1]{\mathbf{\hat{#1}}}
\newcommand{\utan}{\mathbf{\hat{t}}}
\newcommand{\unormal}{\mathbf{\hat{n}}}
\newcommand{\ubinormal}{\mathbf{\hat{b}}}

\newcommand{\dotp}{\bullet}
\newcommand{\cross}{\boldsymbol\times}
\newcommand{\grad}{\boldsymbol\nabla}
\newcommand{\divergence}{\grad\dotp}
\newcommand{\curl}{\grad\cross}
%\DeclareMathOperator{\divergence}{divergence}
%\DeclareMathOperator{\curl}[1]{\grad\cross #1}
\newcommand{\lto}{\mathop{\longrightarrow\,}\limits}

\renewcommand{\bar}{\overline}

\colorlet{textColor}{black} 
\colorlet{background}{white}
\colorlet{penColor}{blue!50!black} % Color of a curve in a plot
\colorlet{penColor2}{red!50!black}% Color of a curve in a plot
\colorlet{penColor3}{red!50!blue} % Color of a curve in a plot
\colorlet{penColor4}{green!50!black} % Color of a curve in a plot
\colorlet{penColor5}{orange!80!black} % Color of a curve in a plot
\colorlet{penColor6}{yellow!70!black} % Color of a curve in a plot
\colorlet{fill1}{penColor!20} % Color of fill in a plot
\colorlet{fill2}{penColor2!20} % Color of fill in a plot
\colorlet{fillp}{fill1} % Color of positive area
\colorlet{filln}{penColor2!20} % Color of negative area
\colorlet{fill3}{penColor3!20} % Fill
\colorlet{fill4}{penColor4!20} % Fill
\colorlet{fill5}{penColor5!20} % Fill
\colorlet{gridColor}{gray!50} % Color of grid in a plot

\newcommand{\surfaceColor}{violet}
\newcommand{\surfaceColorTwo}{redyellow}
\newcommand{\sliceColor}{greenyellow}




\pgfmathdeclarefunction{gauss}{2}{% gives gaussian
  \pgfmathparse{1/(#2*sqrt(2*pi))*exp(-((x-#1)^2)/(2*#2^2))}%
}


%%%%%%%%%%%%%
%% Vectors
%%%%%%%%%%%%%

%% Simple horiz vectors
\renewcommand{\vector}[1]{\left\langle #1\right\rangle}


%% %% Complex Horiz Vectors with angle brackets
%% \makeatletter
%% \renewcommand{\vector}[2][ , ]{\left\langle%
%%   \def\nextitem{\def\nextitem{#1}}%
%%   \@for \el:=#2\do{\nextitem\el}\right\rangle%
%% }
%% \makeatother

%% %% Vertical Vectors
%% \def\vector#1{\begin{bmatrix}\vecListA#1,,\end{bmatrix}}
%% \def\vecListA#1,{\if,#1,\else #1\cr \expandafter \vecListA \fi}

%%%%%%%%%%%%%
%% End of vectors
%%%%%%%%%%%%%

%\newcommand{\fullwidth}{}
%\newcommand{\normalwidth}{}



%% makes a snazzy t-chart for evaluating functions
%\newenvironment{tchart}{\rowcolors{2}{}{background!90!textColor}\array}{\endarray}

%%This is to help with formatting on future title pages.
\newenvironment{sectionOutcomes}{}{} 



%% Flowchart stuff
%\tikzstyle{startstop} = [rectangle, rounded corners, minimum width=3cm, minimum height=1cm,text centered, draw=black]
%\tikzstyle{question} = [rectangle, minimum width=3cm, minimum height=1cm, text centered, draw=black]
%\tikzstyle{decision} = [trapezium, trapezium left angle=70, trapezium right angle=110, minimum width=3cm, minimum height=1cm, text centered, draw=black]
%\tikzstyle{question} = [rectangle, rounded corners, minimum width=3cm, minimum height=1cm,text centered, draw=black]
%\tikzstyle{process} = [rectangle, minimum width=3cm, minimum height=1cm, text centered, draw=black]
%\tikzstyle{decision} = [trapezium, trapezium left angle=70, trapezium right angle=110, minimum width=3cm, minimum height=1cm, text centered, draw=black]

\outcome{Practice with finding volume.}
   

\title[Exercises:]{Volume Exercises}

\begin{document}
\begin{abstract}
  Here we'll practice finding volume using integration.
\end{abstract}
\maketitle

%%Problem 1
\begin{problem}
Consider the region bounded by $y = \frac{5x}{2}-x^2$ and
$y=\frac{x}{2}$.  What is the volume of the solid obtained by
revolving this region about the $x$-axis?

  \[
	\textrm{Volume} = \answer[given]{\frac{12 \pi}{5}}
	\]

\end{problem}

%%Problem 2
\begin{exercise}
Consider the region bounded by $y =\sqrt{x-1}$ , the $x$-axis, and the
vertical line $x=2$.  What is the volume of the solid obtained by
revolving this region about the $y$-axis?
\begin{hint}
  Draw a picture!
\end{hint}

\begin{hint}
  Solving for $x$, we have $x = y^2+1$.  Note that $y$ ranges from $0$ to $1$ as $x$ goes from $1$ to $2$.
\end{hint}

\begin{hint}
  We can decompose the solid into infinitesmal washers with width
  $\d y$, inner radius $y^2+1$ and outer radius $2$.  The volume of each
  washer is $\pi((2)^2 - (y^2+1)^2)\d y$.  Summing these volumes from
  $y=0$ to $y=1$, we obtain
  \[
  \textrm{Volume} = \int_0^1 \pi((2)^2 - (y^2+1)^2)\d y
  \]
\end{hint}

\begin{hint}
  \begin{align*}
    \int_0^1\pi((2)^2 - (y^2+1)^2)\d y &=  \int_0^1 \pi(3-2y^2-y^4)\d y\\
    &= \pi \eval{\frac{-y^5}{5}+\frac{-2y^3}{3}+3y}_0^1\\
    &=\pi(\frac{-1}{5}+\frac{-2}{3}+3)\\
    &=\frac{32\pi}{15}
  \end{align*}
\end{hint}

  \[
  \textrm{Volume} = \answer{\frac{32\pi}{15}}
  \]

\end{exercise}

%%Problem 3
\begin{exercise}
Consider the region bounded by $y =x^2$ and $y=4$.  A solid has this
region as its base, and the cross sections of the solid when cut with
planes parallel to the $(y,z)$-plane are all squares.  What is the area
of the solid?


\begin{hint}
  Draw a picture!  Note that the intersections occur at $x= \pm 2$
\end{hint}

\begin{hint}
  We can decompose the solid into square slabs with width $\d x$ and
  side lengths $4-x^2$.  The volume of each slab is $(4-x^2)\d x$.
  Summing these volumes from $x=-2$ to $x=2$, we obtain
	\[
	\textrm{Volume} = \int_{-2}^{2} (4-x^2)^2\d x
	\]
\end{hint}

\begin{hint}
  First note that this function is even, so we may use symmetry to rewrite the integral as $2\int_{0}^{2} (4-x^2)^2\d x$
  \begin{align*}
    2\int_{0}^{2} (4-x^2)^2\d x &=2 \int_{0}^{2} 16-8x^2+x^4 \d x \\
    &=2 \eval{16x-\frac{8x^3}{3}+\frac{x^5}{5}}_0^2 \\
    &=2(32-\frac{64}{3}+\frac{32}{5})\\
    &=\frac{512}{15}
  \end{align*}
\end{hint}

  \[
  \textrm{Volume} = \answer{\frac{512}{15}}
  \]

\end{exercise}

%%Problem 4
\begin{exercise}
Consider the region bounded by $y = x$ and $y=\frac{x^3}{4}$ in the
first quadrant.  What is the volume of the solid obtained by
revolving this region about the line $y=-1$?

\begin{hint}
	Draw a picture!
\end{hint}

\begin{hint}
  First we find the points of intersectios.
  \begin{align*}
    x&= \frac{x^3}{4}\\
    4x &= x^3\\
    x^3-4x &=0\\
    x(x-2)(x+2) &=0
  \end{align*}

  So the points of intersection are $x=-2$, $x=0$, $x=2$.  Since we only care about the first quadrant, our bounds are from $x=0$ to $x=2$.
\end{hint}

\begin{hint}
  By graphing the two functions, we can see that $y=x$ is always greater than $y=\frac{x^3}{4}$ on the interval $[0,2]$.
\end{hint}

\begin{hint}
  We can decompose the solid into infinitesmal washers with width $dx$, inner radius $\frac{x^3}{4}+1$ and outer radius $x+1$. The volume of each washer is $\pi((x+1)^2-(\frac{x^3}{4}+1)^2)\d x$.  Summing these volumes from $x=0$ to $x=2$, we obtain
  
  \[
  \textrm{Volume} = \int_0^2 \pi(x+1)^2-(\frac{x^3}{4}+1)^2) \d x
  \]
\end{hint}


\begin{hint}
	By expanding this polynomial, we find that this evaluates to $\frac{74\pi}{21}$
\end{hint}

  \[
  \textrm{Volume} = \answer{\frac{74\pi}{21}}
  \]

\end{exercise}

%%Problem 5
\begin{exercise}
Consider the region bounded by the lines $y = x$ , $y=\frac{x}{2}$,
$y=1$, and $y = 4$.  What is the volume of the solid obtained by
revolving this region about the line $x=-2$?

\begin{hint}
	Draw a picture!
\end{hint}

\begin{hint}
	Solving for $x$, we have $x=y$ and $x = 2y$.
\end{hint}

\begin{hint}
  We can decompose the solid into infinitesmal washers with width $\d
  y$, inner radius $y+2$ and outer radius $2y+2$. The volume of each
  washer is $\pi((2y+2)^2-(y+2)^2)\d y$.  Summing these volumes from
  $y=1$ to $y=4$, we obtain
  \[
  \textrm{Volume} = \int_1^4 \pi((2y+2)^2-(y+2)^2)\d y
  \]
\end{hint}


\begin{hint}
  By expanding this polynomial, we find that this evaluates to
  $93\pi$.
\end{hint}

  \[
  \textrm{Volume} = \answer{93\pi}
  \]


\end{exercise}


%%Problem 6
\begin{exercise}
Consider the region bounded by the lines $y = \sin(x)$, $x=0$,
$x=\pi$, and the $x$-axis.  What is the volume of the solid obtained
by revolving this region about $x$-axis?

It will be useful to recall that $\sin^2(x) = \frac{1-\cos(2x)}{2}$.

\begin{hint}
  Draw a picture!
\end{hint}

\begin{hint}
  We can decompose the solid into infinitesmal disks with width
  $\d x$ and radius $\sin(x)$. The volume of each washer is $\pi
  \sin^2(x)\d x$.  Summing these volumes from $x=0$ to $x=\pi$,
  we obtain
  \[
  \textrm{Volume} = \int_0^\pi \pi \sin^2(x)\d x
  \]
\end{hint}

\begin{hint}
  Using the half angle reduction formula, and a substitution, we obtain that this evaluates to $\frac{\pi^2}{2}$
\end{hint}

  \[
  \textrm{Volume} = \answer{\frac{\pi^2}{2}}
  \]


\end{exercise}













\end{document}
