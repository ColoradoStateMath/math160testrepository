\documentclass[handout]{ximera}
%\documentclass[10pt,handout,twocolumn,twoside,wordchoicegiven]{xercises}
%\documentclass[10pt,handout,twocolumn,twoside,wordchoicegiven]{xourse}

%\author{Steven Gubkin}
%\license{Creative Commons 3.0 By-NC}
%\usepackage{todonotes}

\newcommand{\todo}{}

\usepackage{esint} % for \oiint
\graphicspath{
{./}
{functionsOfSeveralVariables/}
{normalVectors/}
{lagrangeMultipliers/}
{vectorFields/}
{greensTheorem/}
{shapeOfThingsToCome/}
}


\usepackage{tkz-euclide}
\tikzset{>=stealth} %% cool arrow head
\tikzset{shorten <>/.style={ shorten >=#1, shorten <=#1 } } %% allows shorter vectors

\usetikzlibrary{backgrounds} %% for boxes around graphs
\usetikzlibrary{shapes,positioning}  %% Clouds and stars
\usetikzlibrary{matrix} %% for matrix
\usepgfplotslibrary{polar} %% for polar plots
\usetkzobj{all}
\usepackage[makeroom]{cancel} %% for strike outs
%\usepackage{mathtools} %% for pretty underbrace % Breaks Ximera
\usepackage{multicol}
\usepackage{pgffor} %% required for integral for loops


%% http://tex.stackexchange.com/questions/66490/drawing-a-tikz-arc-specifying-the-center
%% Draws beach ball
\tikzset{pics/carc/.style args={#1:#2:#3}{code={\draw[pic actions] (#1:#3) arc(#1:#2:#3);}}}



\usepackage{array}
\setlength{\extrarowheight}{+.1cm}   
\newdimen\digitwidth
\settowidth\digitwidth{9}
\def\divrule#1#2{
\noalign{\moveright#1\digitwidth
\vbox{\hrule width#2\digitwidth}}}





\newcommand{\RR}{\mathbb R}
\newcommand{\R}{\mathbb R}
\newcommand{\N}{\mathbb N}
\newcommand{\Z}{\mathbb Z}

%\newcommand{\sage}{\textsf{SageMath}}


%\renewcommand{\d}{\,d\!}
\renewcommand{\d}{\mathop{}\!d}
\newcommand{\dd}[2][]{\frac{\d #1}{\d #2}}
\newcommand{\pp}[2][]{\frac{\partial #1}{\partial #2}}
\renewcommand{\l}{\ell}
\newcommand{\ddx}{\frac{d}{\d x}}

\newcommand{\zeroOverZero}{\ensuremath{\boldsymbol{\tfrac{0}{0}}}}
\newcommand{\inftyOverInfty}{\ensuremath{\boldsymbol{\tfrac{\infty}{\infty}}}}
\newcommand{\zeroOverInfty}{\ensuremath{\boldsymbol{\tfrac{0}{\infty}}}}
\newcommand{\zeroTimesInfty}{\ensuremath{\small\boldsymbol{0\cdot \infty}}}
\newcommand{\inftyMinusInfty}{\ensuremath{\small\boldsymbol{\infty - \infty}}}
\newcommand{\oneToInfty}{\ensuremath{\boldsymbol{1^\infty}}}
\newcommand{\zeroToZero}{\ensuremath{\boldsymbol{0^0}}}
\newcommand{\inftyToZero}{\ensuremath{\boldsymbol{\infty^0}}}



\newcommand{\numOverZero}{\ensuremath{\boldsymbol{\tfrac{\#}{0}}}}
\newcommand{\dfn}{\textbf}
%\newcommand{\unit}{\,\mathrm}
\newcommand{\unit}{\mathop{}\!\mathrm}
\newcommand{\eval}[1]{\bigg[ #1 \bigg]}
\newcommand{\seq}[1]{\left( #1 \right)}
\renewcommand{\epsilon}{\varepsilon}
\renewcommand{\phi}{\varphi}


\renewcommand{\iff}{\Leftrightarrow}

\DeclareMathOperator{\arccot}{arccot}
\DeclareMathOperator{\arcsec}{arcsec}
\DeclareMathOperator{\arccsc}{arccsc}
\DeclareMathOperator{\si}{Si}
\DeclareMathOperator{\proj}{\vec{proj}}
\DeclareMathOperator{\scal}{scal}
\DeclareMathOperator{\sign}{sign}


%% \newcommand{\tightoverset}[2]{% for arrow vec
%%   \mathop{#2}\limits^{\vbox to -.5ex{\kern-0.75ex\hbox{$#1$}\vss}}}
\newcommand{\arrowvec}{\overrightarrow}
%\renewcommand{\vec}[1]{\arrowvec{\mathbf{#1}}}
\renewcommand{\vec}{\mathbf}
\newcommand{\veci}{{\boldsymbol{\hat{\imath}}}}
\newcommand{\vecj}{{\boldsymbol{\hat{\jmath}}}}
\newcommand{\veck}{{\boldsymbol{\hat{k}}}}
\newcommand{\vecl}{\boldsymbol{\l}}
\newcommand{\uvec}[1]{\mathbf{\hat{#1}}}
\newcommand{\utan}{\mathbf{\hat{t}}}
\newcommand{\unormal}{\mathbf{\hat{n}}}
\newcommand{\ubinormal}{\mathbf{\hat{b}}}

\newcommand{\dotp}{\bullet}
\newcommand{\cross}{\boldsymbol\times}
\newcommand{\grad}{\boldsymbol\nabla}
\newcommand{\divergence}{\grad\dotp}
\newcommand{\curl}{\grad\cross}
%\DeclareMathOperator{\divergence}{divergence}
%\DeclareMathOperator{\curl}[1]{\grad\cross #1}
\newcommand{\lto}{\mathop{\longrightarrow\,}\limits}

\renewcommand{\bar}{\overline}

\colorlet{textColor}{black} 
\colorlet{background}{white}
\colorlet{penColor}{blue!50!black} % Color of a curve in a plot
\colorlet{penColor2}{red!50!black}% Color of a curve in a plot
\colorlet{penColor3}{red!50!blue} % Color of a curve in a plot
\colorlet{penColor4}{green!50!black} % Color of a curve in a plot
\colorlet{penColor5}{orange!80!black} % Color of a curve in a plot
\colorlet{penColor6}{yellow!70!black} % Color of a curve in a plot
\colorlet{fill1}{penColor!20} % Color of fill in a plot
\colorlet{fill2}{penColor2!20} % Color of fill in a plot
\colorlet{fillp}{fill1} % Color of positive area
\colorlet{filln}{penColor2!20} % Color of negative area
\colorlet{fill3}{penColor3!20} % Fill
\colorlet{fill4}{penColor4!20} % Fill
\colorlet{fill5}{penColor5!20} % Fill
\colorlet{gridColor}{gray!50} % Color of grid in a plot

\newcommand{\surfaceColor}{violet}
\newcommand{\surfaceColorTwo}{redyellow}
\newcommand{\sliceColor}{greenyellow}




\pgfmathdeclarefunction{gauss}{2}{% gives gaussian
  \pgfmathparse{1/(#2*sqrt(2*pi))*exp(-((x-#1)^2)/(2*#2^2))}%
}


%%%%%%%%%%%%%
%% Vectors
%%%%%%%%%%%%%

%% Simple horiz vectors
\renewcommand{\vector}[1]{\left\langle #1\right\rangle}


%% %% Complex Horiz Vectors with angle brackets
%% \makeatletter
%% \renewcommand{\vector}[2][ , ]{\left\langle%
%%   \def\nextitem{\def\nextitem{#1}}%
%%   \@for \el:=#2\do{\nextitem\el}\right\rangle%
%% }
%% \makeatother

%% %% Vertical Vectors
%% \def\vector#1{\begin{bmatrix}\vecListA#1,,\end{bmatrix}}
%% \def\vecListA#1,{\if,#1,\else #1\cr \expandafter \vecListA \fi}

%%%%%%%%%%%%%
%% End of vectors
%%%%%%%%%%%%%

%\newcommand{\fullwidth}{}
%\newcommand{\normalwidth}{}



%% makes a snazzy t-chart for evaluating functions
%\newenvironment{tchart}{\rowcolors{2}{}{background!90!textColor}\array}{\endarray}

%%This is to help with formatting on future title pages.
\newenvironment{sectionOutcomes}{}{} 



%% Flowchart stuff
%\tikzstyle{startstop} = [rectangle, rounded corners, minimum width=3cm, minimum height=1cm,text centered, draw=black]
%\tikzstyle{question} = [rectangle, minimum width=3cm, minimum height=1cm, text centered, draw=black]
%\tikzstyle{decision} = [trapezium, trapezium left angle=70, trapezium right angle=110, minimum width=3cm, minimum height=1cm, text centered, draw=black]
%\tikzstyle{question} = [rectangle, rounded corners, minimum width=3cm, minimum height=1cm,text centered, draw=black]
%\tikzstyle{process} = [rectangle, minimum width=3cm, minimum height=1cm, text centered, draw=black]
%\tikzstyle{decision} = [trapezium, trapezium left angle=70, trapezium right angle=110, minimum width=3cm, minimum height=1cm, text centered, draw=black]

\outcome{Practice Limits.}
   

\title{Mixed Limit Practice - Question Bank}

\begin{document}
\begin{abstract}
This is a bank of limit problems to pick the mixed practice from. 
\end{abstract}
\maketitle

%%limit law 1 - Mixed Practice 2%%
\begin{exercise}
Suppose that $\displaystyle\lim_{u\to2}c(u)=-5$, $\displaystyle\lim_{u\to2}B(u)=-1$, and $\displaystyle\lim_{u\to2}Y(u)=-4$. Compute the limit

\[
\lim_{u\to 2 } \frac{c(u)}{B(u)-Y(u)}\begin{prompt} = \answer{-\frac{5}{3}}\end{prompt}
\]
\end{exercise}

%%Lim Law 2 - Mixed Practice 4
\begin{exercise}
Suppose that $\lim_{z\to5}G(z)=3$, $\lim_{z\to5}C(z)=-4$, and $\lim_{z\to5}c(z)=1$. Compute the limit

\[
\lim_{z\to 5 } \frac{G(z)}{C(z)-c(z)} = \answer{-\frac{3}{5}}
\]
\end{exercise}

%%Factoring 1 - MADE UP BY MARY - Mixed Practice 1%%
\begin{exercise}
\[\lim_{x \to 5} \frac{x^2-25}{x-5} = \answer{10}\]
\end{exercise}


%%Factoring 2 - MADE UP BY MARY - Mixed Practice 4%%
\begin{exercise}
\[\lim_{x \to -4} \frac{x^2+3x-4}{x^2-4x-32} = \answer{5/12}\]

\end{exercise}

%%Factoring 3 - MADE UP BY MARY%%
\begin{exercise}
\[\lim_{x \to 0} \frac{x^2}{x^2-x} = \answer{0}\]

\end{exercise}

%%Factoring 4
\begin{exercise}
\[
\lim_{v\to -5 } \frac{v^2+6 v+5}{v+5} = \answer{-4}
\]
\end{exercise}

%%Factoring 5
\begin{exercise}
\[
\lim_{z\to -4 } \frac{z^2+6 z+8}{z+4} = \answer{-2}
\]
\end{exercise}

%%Factoring 6
\begin{exercise}
\[
\lim_{z\to -2 } \frac{z+2}{-2 z^2-2 z+4} = \answer{\frac{1}{6}}
\]
\end{exercise}


%%Factoring 7 - Mixed Practice 3
\begin{exercise}
\[
\lim_{\theta\to -1 } \frac{\theta +1}{-2 \theta ^2-6 \theta -4} = \answer{-\frac{1}{2}}
\]
\end{exercise}

%%Fractions 1 - MADE UP BY MARY - Mixed Practice 1%%
\begin{exercise}
\[\lim_{x \to 2} \frac{\frac{1}{2}-\frac{1}{x}}{2-x} = \answer{-1/4}\]
\end{exercise}

%%Fractions 2 - MADE UP BY MARY - Mixed Practice 2%%
\begin{exercise}
\[\lim_{q \to 0} \frac{\frac{1}{7}-\frac{1}{3q}}{\frac{1}{3q}-\frac{1}{7}} = \answer{-1}\]
\end{exercise}

%%Fractions 3 - MADE UP BY MARY - Mixed Practice 4%%
\begin{exercise}
\[\lim_{z \to -5} \frac{3z}{\frac{1}{z+5}-\frac{2}{z+10}} = \answer{-3}\]
\end{exercise}

%%Conjugate 1 - MADE UP BY MARY%%
\begin{exercise}
\[\lim_{x \to 0} \frac{\sqrt{9+x}-3}{x} = \answer{1/6}\]

\end{exercise}

%% Conjugate 2 - Question 6 from exercises - Mixed Practice 3
\begin{exercise}
	\[\lim_{h \to 0} \frac{\sqrt{2+h} - \sqrt{2}}{h} = \answer{1/(2\sqrt{2})}\]

\end{exercise}

%% Conjugate 3 - Mixed Practice 2
\begin{exercise}
\[
\lim_{\psi\to 4 } \frac{\sqrt{\psi +5}-3}{\psi -4}\begin{prompt} = \answer{\frac{1}{6}}\end{prompt}
\]
\end{exercise}

%%Conjugate 4
\begin{exercise}
\[
\lim_{n\to -5 } \frac{\sqrt{4-n}-3}{n+5}= \answer{-\frac{1}{6}}
\]
\end{exercise}

%%Conjugate 5
\begin{exercise}
\[
\lim_{n\to 5 } \frac{\sqrt{n+5}-\sqrt{10}}{n-5} = \answer{\frac{1}{2 \sqrt{10}}}
\]
\end{exercise}


%% Conjugate 6 - Mixed Practice 4
\begin{exercise}
\[
\lim_{t\to 3 } \frac{t-3}{\sqrt{t-2}-1} = \answer{2}
\]
\end{exercise}

%%Trig 1 - MADE UP BY MARY - Mixed Practice 2%%
\begin{exercise}
\[\lim_{x \to 0} \frac{\tan(x)}{\sin(x)} = \answer{1}\]
\end{exercise}

%% trig 2 - Mixed Practice 3
\begin{exercise}
Find
\[
\lim_{x\to0}\left(\frac{\cos^{2}(x)-1}{\cos(x)-1}\right)
= \answer{2}
\]

\begin{hint}
Recall that for any two numbers, $a$ and $b$,  $a^2-b^2=(a-b)(a+b)$.
\end{hint}

\end{exercise}

%%Trig 3 - Mixed Practice 4%%

\begin{exercise}
\[\lim_{x \to \pi/3} \sec^2(x)+1 = \answer{5}\]
\end{exercise}

%% sinx/x 1 - Question 1/2 from exercises - Mixed Practice 1
\begin{exercise}
\[\lim_{x \to 0} \frac{\sin(2x)}{3x} = \answer{2/3}\]
\begin{hint}
Recall that $\displaystyle\lim_{x \to 0} \frac{\sin x}{x} = 1$.
\end{hint}
\end{exercise}

%% sinx/x 2 - Question 3 from exercises - Mixed Practice 3
\begin{exercise}
\[\lim_{x \to 0} \frac{\sin(5x)}{\sin(2x)} = \answer{5/2}\]
\end{exercise}

%% sinx/x 3 - Question 4 from exercises - Mixed Practice 2
\begin{exercise}
\[\lim_{x \to 0} \frac{\cos(x)-1}{x} = \answer{0}\]

\begin{hint}
This is a limit you could memorize.  Alternatively, if you've forgotten the value of this limit, you could multiply by $\cos(x)+1$ in the numerator and denominator and then apply the Pythagorean identity to get to the solution.
\end{hint}

\end{exercise}

%%piece-wise limits 1 - Mixed Practice 1%%
\begin{exercise}
\[
g(x) = \begin{cases}
  \frac{x^2 - 4}{x-2}  &\text{if $x<1$,} \\
  -x+1 &\text{if  $x>1$.}
\end{cases}
\]
Does $\lim_{x \to 2} g(x)$ exist?  If it does, give its value.
Otherwise, write DNE.

\[
\lim_{x \to 2} g(x) = \answer{-1}
\]

Does $\lim_{x \to 1} g(x)$ exist?  If it does, give its value.
Otherwise, write DNE.

\[
\lim_{x \to 1} g(x) = \answer{DNE}
\]
\end{exercise}

%%Piece-wise limits 2 - Mixed Practice 2
\begin{exercise}
\[
k(x) = \begin{cases}
  \frac{\sqrt{x+3}-2}{x-1}  & x<1 \\
  7x & x=1 \\
  \frac{2x^3 -x}{4x} & x > 1
\end{cases}
\]
Does $\lim_{x \to 1} k(x)$ exist?  If it does, give its value.
Otherwise, write DNE.

\[
\lim_{x \to 1} k(x) = \answer{\frac{1}{4}}
\]
\end{exercise}

%%Piece-wise limits 3 - Mixed Practice 3
\begin{exercise}
\[
m(x) = \begin{cases}
  \frac{1}{x}  & x < 0 \\
  1 & x \geq 0
\end{cases}
\]

Evaluate the following limits if they exist.  Otherwise, write DNE. 

\begin{itemize}

\item $\displaystyle\lim_{x \to 0^-} m(x) = \answer{-\infty}$

\item $\displaystyle\lim_{x \to 0^+} m(x) = \answer{1}$

\item $\displaystyle\lim_{x \to 0} m(x) = \answer{DNE}$

\end{itemize}

\end{exercise}

%%Piece-wise limits 4 - Mixed Practice 4
\begin{exercise}
\[
z(x) = \frac{3|x|+|x|^3}{2|x|^3 - |x|^2}
\]

Evaluate the following limits.  

\begin{itemize}

\item $\displaystyle\lim_{x \to \infty} z(x) = \answer{\frac{1}{2}}$

\item $\displaystyle\lim_{x \to -\infty} z(x) = \answer{\frac{1}{2}}$

\end{itemize}

\end{exercise}

%%VA 1 - Mixed Practice 1%%
\begin{exercise}
Consider 
\[
f(\psi) = \frac{-5}{\psi -4}.
\]
Compute
\begin{enumerate}
\item $\displaystyle\lim_{\psi\to 4^- } f(\psi) \begin{prompt} = \answer{\infty}\end{prompt}$
\item $\displaystyle\lim_{\psi\to 4^+ } f(\psi) \begin{prompt} = \answer{-\infty}\end{prompt}$
\item $\displaystyle\lim_{\psi\to 4 } f(\psi) \begin{prompt} = \answer{DNE}\end{prompt}$
\end{enumerate}
\end{exercise}

%%VA 2 - Mixed Practice 3%%
\begin{exercise}
Consider 
\[
g(a) = \frac{6}{a^2}.
\]
Compute
\[ \displaystyle\lim_{a \to 0 } g(a) \begin{prompt} = \answer{\infty}\end{prompt} \]

\end{exercise}

%%Inf Lim 1 - Mixed Practice 4%%
\begin{exercise}
Let 
\[
r(k) = \frac{\sqrt{k^6+5}-2 k^2}{k-k^3}.
\]
Compute
\begin{enumerate}
\item $\displaystyle\lim_{k\to \infty} r(k) \begin{prompt} = \answer{-1}\end{prompt}$
\item $\displaystyle\lim_{k\to -\infty}r(k) \begin{prompt} = \answer{1}\end{prompt}$
\end{enumerate}
\begin{hint}
Multiply by
\[
\frac{\frac{1}{k^3}}{\frac{1}{k^3}}
\]
\end{hint}
\end{exercise}


%%Inf Lim 2 - Mixed Practice 1
\begin{exercise}
Let 
\[
F(z) = \frac{-5 z^3-3}{z^3-5 z^2-2 z}.
\]
Compute
\begin{enumerate}
\item $\displaystyle\lim_{z\to \infty} F(z) \begin{prompt} = \answer{-5}\end{prompt}$
\item $\displaystyle\lim_{z\to -\infty}F(z) \begin{prompt} = \answer{-5}\end{prompt}$
\end{enumerate}
\begin{hint}
Multiply by
\[
\frac{\frac{1}{z^3}}{\frac{1}{z^3}}
\]
\end{hint}
\end{exercise}

%%Inf Lim 3 - Mixed Practice 2
\begin{exercise}
Let 
\[
y(z) = \frac{5 z^2+z^2+z+5}{2 z^3-5 z^2-3 z+3}.
\]
Compute
\begin{enumerate}
\item $\displaystyle\lim_{z\to \infty} y(z) \begin{prompt} = \answer{0}\end{prompt}$
\item $\displaystyle\lim_{z\to -\infty}y(z) \begin{prompt} = \answer{0}\end{prompt}$
\end{enumerate}
\begin{hint}
Multiply by
\[
\frac{\frac{1}{z^3}}{\frac{1}{z^3}}
\]
\end{hint}
\end{exercise}

%%Inf Lim 4
\begin{exercise}
Find
\[
\lim_{x\to-\infty}\left(\frac{2x^3-3x^2+4}{x^3+3x^2-1}\right)
= \answer{2}
\]

\end{exercise}

%% DNE 1 - Mixed Practice 2
\begin{exercise}
Find
\[
\lim_{\theta\to\infty}\left(\cos(\theta)\right)
= \answer{DNE}
\]

\end{exercise}

%% DNE 2 - Mixed Practice 4
\begin{exercise}
Find
\[
\lim_{\alpha \to 0} \sin\left(\frac{3}{\alpha}\right)
= \answer{DNE}
\]

\begin{hint}
Think very carefully how the function values of $\sin\left(\frac{3}{\alpha}\right)$ behave as $x$ approaches 0.  (You may remember a function similar to this that you explored in lab $\# 1$.) 
\end{hint}

\end{exercise}

%% Absolute Value 1 - Mixed Practice 1
\begin{exercise}
Find
\[
\lim_{x\to6}\left(\frac{\left|6-x\right|}{6-x}\right)
= \answer{DNE}
\]

\begin{hint}
Absolute value = piece-wise in disguise!
\end{hint}
\end{exercise}

%% Absolute Value 2 - Mixed Practice 3
\begin{exercise}
Find 
\[ \displaystyle\lim_{x \to -3} \frac{(x+3)^2}{|x+3|} = \answer{0} \]
\end{exercise}

%% Squeeze Theorem 1 - Mixed Practice 3
\begin{exercise}
Find
\[ \displaystyle\lim_{x \to 0} \sqrt[3]{x} \cos \left(\frac{1}{x}\right) = \answer{0}\]

\end{exercise}

%% Question 5 from exercises
%\begin{exercise}
%\[\lim_{x \to 0} \frac{\cos(2x)-1}{x\sin(3x)} = \answer{-2/3}\]
%\begin{hint}
%	Multiply numerator and denominator by $\cos(2x)+1$, expand the %numerator, use Pythagorean identity, and should be almost home free.
%\end{hint}
%\end{exercise}

%% Question 8 from exercises
%\begin{exercise}
%If I want 
%\[
%\displaystyle\lim_{x \to 0} \frac{1-\cos(ax)}{x\sin(2x)} = 4
%\]

%what should I pick for $a$ if $a>0$?
%	\[a = \answer{4}\]
%	\begin{hint}
%		Multiply numerator and denominator by $1+\cos(ax)$, expand the %numerator, and use the pythagorean identity.  This will let you find %the limit in terms of $a$.  Then solve the equation to find the value %of $a$.
%	\end{hint}
%\end{exercise}

%% Question 9 from exercises
%\begin{exercise}
%	\begin{warning}
%		This problem is pretty hard.
%	\end{warning}
%	\[\lim_{h \to 0} \frac{\sin(1+h) - 2\sin(1)+\sin(1-h)}{h} = \answer{-\sin(1)}\]
%	\begin{hint}
%		Use the angle sum formula for sine on both $\sin(1+h)$ and $\sin(1-h)$
%	\end{hint}
%\end{exercise}

\end{document}
























\end{document}
