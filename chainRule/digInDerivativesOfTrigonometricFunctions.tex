\documentclass{ximera}

%\usepackage{todonotes}

\newcommand{\todo}{}

\usepackage{esint} % for \oiint
\graphicspath{
{./}
{functionsOfSeveralVariables/}
{normalVectors/}
{lagrangeMultipliers/}
{vectorFields/}
{greensTheorem/}
{shapeOfThingsToCome/}
}


\usepackage{tkz-euclide}
\tikzset{>=stealth} %% cool arrow head
\tikzset{shorten <>/.style={ shorten >=#1, shorten <=#1 } } %% allows shorter vectors

\usetikzlibrary{backgrounds} %% for boxes around graphs
\usetikzlibrary{shapes,positioning}  %% Clouds and stars
\usetikzlibrary{matrix} %% for matrix
\usepgfplotslibrary{polar} %% for polar plots
\usetkzobj{all}
\usepackage[makeroom]{cancel} %% for strike outs
%\usepackage{mathtools} %% for pretty underbrace % Breaks Ximera
\usepackage{multicol}
\usepackage{pgffor} %% required for integral for loops


%% http://tex.stackexchange.com/questions/66490/drawing-a-tikz-arc-specifying-the-center
%% Draws beach ball
\tikzset{pics/carc/.style args={#1:#2:#3}{code={\draw[pic actions] (#1:#3) arc(#1:#2:#3);}}}



\usepackage{array}
\setlength{\extrarowheight}{+.1cm}   
\newdimen\digitwidth
\settowidth\digitwidth{9}
\def\divrule#1#2{
\noalign{\moveright#1\digitwidth
\vbox{\hrule width#2\digitwidth}}}





\newcommand{\RR}{\mathbb R}
\newcommand{\R}{\mathbb R}
\newcommand{\N}{\mathbb N}
\newcommand{\Z}{\mathbb Z}

%\newcommand{\sage}{\textsf{SageMath}}


%\renewcommand{\d}{\,d\!}
\renewcommand{\d}{\mathop{}\!d}
\newcommand{\dd}[2][]{\frac{\d #1}{\d #2}}
\newcommand{\pp}[2][]{\frac{\partial #1}{\partial #2}}
\renewcommand{\l}{\ell}
\newcommand{\ddx}{\frac{d}{\d x}}

\newcommand{\zeroOverZero}{\ensuremath{\boldsymbol{\tfrac{0}{0}}}}
\newcommand{\inftyOverInfty}{\ensuremath{\boldsymbol{\tfrac{\infty}{\infty}}}}
\newcommand{\zeroOverInfty}{\ensuremath{\boldsymbol{\tfrac{0}{\infty}}}}
\newcommand{\zeroTimesInfty}{\ensuremath{\small\boldsymbol{0\cdot \infty}}}
\newcommand{\inftyMinusInfty}{\ensuremath{\small\boldsymbol{\infty - \infty}}}
\newcommand{\oneToInfty}{\ensuremath{\boldsymbol{1^\infty}}}
\newcommand{\zeroToZero}{\ensuremath{\boldsymbol{0^0}}}
\newcommand{\inftyToZero}{\ensuremath{\boldsymbol{\infty^0}}}



\newcommand{\numOverZero}{\ensuremath{\boldsymbol{\tfrac{\#}{0}}}}
\newcommand{\dfn}{\textbf}
%\newcommand{\unit}{\,\mathrm}
\newcommand{\unit}{\mathop{}\!\mathrm}
\newcommand{\eval}[1]{\bigg[ #1 \bigg]}
\newcommand{\seq}[1]{\left( #1 \right)}
\renewcommand{\epsilon}{\varepsilon}
\renewcommand{\phi}{\varphi}


\renewcommand{\iff}{\Leftrightarrow}

\DeclareMathOperator{\arccot}{arccot}
\DeclareMathOperator{\arcsec}{arcsec}
\DeclareMathOperator{\arccsc}{arccsc}
\DeclareMathOperator{\si}{Si}
\DeclareMathOperator{\proj}{\vec{proj}}
\DeclareMathOperator{\scal}{scal}
\DeclareMathOperator{\sign}{sign}


%% \newcommand{\tightoverset}[2]{% for arrow vec
%%   \mathop{#2}\limits^{\vbox to -.5ex{\kern-0.75ex\hbox{$#1$}\vss}}}
\newcommand{\arrowvec}{\overrightarrow}
%\renewcommand{\vec}[1]{\arrowvec{\mathbf{#1}}}
\renewcommand{\vec}{\mathbf}
\newcommand{\veci}{{\boldsymbol{\hat{\imath}}}}
\newcommand{\vecj}{{\boldsymbol{\hat{\jmath}}}}
\newcommand{\veck}{{\boldsymbol{\hat{k}}}}
\newcommand{\vecl}{\boldsymbol{\l}}
\newcommand{\uvec}[1]{\mathbf{\hat{#1}}}
\newcommand{\utan}{\mathbf{\hat{t}}}
\newcommand{\unormal}{\mathbf{\hat{n}}}
\newcommand{\ubinormal}{\mathbf{\hat{b}}}

\newcommand{\dotp}{\bullet}
\newcommand{\cross}{\boldsymbol\times}
\newcommand{\grad}{\boldsymbol\nabla}
\newcommand{\divergence}{\grad\dotp}
\newcommand{\curl}{\grad\cross}
%\DeclareMathOperator{\divergence}{divergence}
%\DeclareMathOperator{\curl}[1]{\grad\cross #1}
\newcommand{\lto}{\mathop{\longrightarrow\,}\limits}

\renewcommand{\bar}{\overline}

\colorlet{textColor}{black} 
\colorlet{background}{white}
\colorlet{penColor}{blue!50!black} % Color of a curve in a plot
\colorlet{penColor2}{red!50!black}% Color of a curve in a plot
\colorlet{penColor3}{red!50!blue} % Color of a curve in a plot
\colorlet{penColor4}{green!50!black} % Color of a curve in a plot
\colorlet{penColor5}{orange!80!black} % Color of a curve in a plot
\colorlet{penColor6}{yellow!70!black} % Color of a curve in a plot
\colorlet{fill1}{penColor!20} % Color of fill in a plot
\colorlet{fill2}{penColor2!20} % Color of fill in a plot
\colorlet{fillp}{fill1} % Color of positive area
\colorlet{filln}{penColor2!20} % Color of negative area
\colorlet{fill3}{penColor3!20} % Fill
\colorlet{fill4}{penColor4!20} % Fill
\colorlet{fill5}{penColor5!20} % Fill
\colorlet{gridColor}{gray!50} % Color of grid in a plot

\newcommand{\surfaceColor}{violet}
\newcommand{\surfaceColorTwo}{redyellow}
\newcommand{\sliceColor}{greenyellow}




\pgfmathdeclarefunction{gauss}{2}{% gives gaussian
  \pgfmathparse{1/(#2*sqrt(2*pi))*exp(-((x-#1)^2)/(2*#2^2))}%
}


%%%%%%%%%%%%%
%% Vectors
%%%%%%%%%%%%%

%% Simple horiz vectors
\renewcommand{\vector}[1]{\left\langle #1\right\rangle}


%% %% Complex Horiz Vectors with angle brackets
%% \makeatletter
%% \renewcommand{\vector}[2][ , ]{\left\langle%
%%   \def\nextitem{\def\nextitem{#1}}%
%%   \@for \el:=#2\do{\nextitem\el}\right\rangle%
%% }
%% \makeatother

%% %% Vertical Vectors
%% \def\vector#1{\begin{bmatrix}\vecListA#1,,\end{bmatrix}}
%% \def\vecListA#1,{\if,#1,\else #1\cr \expandafter \vecListA \fi}

%%%%%%%%%%%%%
%% End of vectors
%%%%%%%%%%%%%

%\newcommand{\fullwidth}{}
%\newcommand{\normalwidth}{}



%% makes a snazzy t-chart for evaluating functions
%\newenvironment{tchart}{\rowcolors{2}{}{background!90!textColor}\array}{\endarray}

%%This is to help with formatting on future title pages.
\newenvironment{sectionOutcomes}{}{} 



%% Flowchart stuff
%\tikzstyle{startstop} = [rectangle, rounded corners, minimum width=3cm, minimum height=1cm,text centered, draw=black]
%\tikzstyle{question} = [rectangle, minimum width=3cm, minimum height=1cm, text centered, draw=black]
%\tikzstyle{decision} = [trapezium, trapezium left angle=70, trapezium right angle=110, minimum width=3cm, minimum height=1cm, text centered, draw=black]
%\tikzstyle{question} = [rectangle, rounded corners, minimum width=3cm, minimum height=1cm,text centered, draw=black]
%\tikzstyle{process} = [rectangle, minimum width=3cm, minimum height=1cm, text centered, draw=black]
%\tikzstyle{decision} = [trapezium, trapezium left angle=70, trapezium right angle=110, minimum width=3cm, minimum height=1cm, text centered, draw=black]

\title[Dig-In:]{Derivatives of trigonometric functions}

\outcome{Apply the product and quotient rule compute derivatives of trigonometric functions.}

\begin{document}
\begin{abstract}
  We use the product and quotient rule to unleash the derivatives of the
  trigonometric functions.
\end{abstract}
\maketitle


Up until this point of the course we have been ignoring a large class
of functions: Trigonometric functions other than $\sin(x)$ and $\cos(x)$. We know
that
\[
\ddx \sin(x) = \cos(x).
\]
and 
\[
\ddx \cos(x) = -\sin(x).
\]
Armed with this fact we will discover the derivatives of all of the
standard trigonometric functions.



%\begin{example}
%Compute:
%\[
%\eval{\ddx \cos \left( \frac{x^3}{2} \right)}_{x=\sqrt[3]{\pi}}
%\]
%\begin{explanation}
%Now that we know the derivative of cosine, we may combine this with the
%chain rule, so we have that
%\[
%\ddx \cos \left( \frac{x^3}{2} \right) = \answer[given]{\frac{3 x^2}{2}} \left(- \sin \left( \frac{x^3}{2} \right) \right)
%\]
%and so
%\[
%\eval{\ddx \cos \left( \frac{x^3}{2} \right)}_{x=\sqrt[3]{\pi}}
%\]
%\begin{align*}
%  &= \eval{\left( \frac{3}{2} x^2 \left(- \sin \left( \frac{x^3}{2}
%    \right) \right) \right)}_{x=\sqrt[3]{\pi}} \\
%  &= - \frac{3}{2}(\sqrt[3]{\pi})^2 \sin \left( \frac{\pi}{2} %\right) \\
%  &= -\frac{3}{2} \pi^{\frac{2}{3}} \cdot \answer[given]{1} \\
%  &=\answer[given]{\frac{-3 \pi^{\frac{2}{3}}}{2}}.
%\end{align*}
%\end{explanation}
%\end{example}


Next we have:

\begin{theorem}[The derivative of tangent]\index{derivative!of tangent}
\[
\ddx \tan(x) = \sec^2(x).
\]

\begin{explanation}
We'll rewrite $\tan(x)$ as $\frac{\sin(x)}{\cos(x)}$ and use the
quotient rule. Write with me:
\begin{align*}
\ddx\tan(x) &= \ddx\frac{\sin(x)}{\cos(x)}\\
&=\frac{\cos^2(x) + \answer[given]{\sin^2(x)}}{\cos^2(x)}\\
&=\frac{\answer[given]{1}}{\cos^2(x)}\\
&=\sec^2(x).
\end{align*}
\end{explanation}
\end{theorem}

\begin{example}
Compute:
\[
\ddx \left( \frac{5x \tan(x)}{x^2 - 3} \right)
\]
\begin{explanation}
Applying the quotient rule, and the product rule, and the derivative
of cosine:
\begin{align*}
  \ddx &\left( \frac{5x \tan(x)}{x^2 - 3} \right) \\
  &= \frac{(x^2 - 3) \cdot \ddx(\answer[given]{5x \tan(x)}) - 5x \tan(x) \cdot \ddx (\answer[given]{x^2 - 3})}{(x^2 - 3)^2}  \\
  &= \frac{(x^2 - 3)(5 \tan(x) + 5x \answer[given]{\sec^2(x)}) - 5x \tan(x) \cdot 2x}{(x^2 - 3)^2}  \\
  &= \frac{5(x^2-3)(\tan(x)+x \sec^2(x)) - 10x^2 \tan(x)}{(x^2-3)^2}
\end{align*}
\end{explanation}
\end{example}

Finally, we have:

\begin{theorem}[The derivative of secant]\index{derivative!of secant}
\[
\ddx \sec(x) = \sec(x)\tan(x).
\]


\begin{explanation}
We'll rewrite $\sec(x)$ as $(\cos(x))^{-1}$ and use the power rule and the chain rule. Write
\begin{align*}
\ddx \sec(x) &= \ddx(\cos (x))^{-1}\\
&=-1(\cos(x))^{-2}(\answer[given]{-\sin(x)}) \\
&= \frac{\sin(x)}{\cos^2(x)} \\
&= \frac{1}{\cos(x)} \cdot \frac{\sin(x)}{\cos(x)}  \\
&= \sec(x)\tan(x).
\end{align*}
\end{explanation}
\end{theorem}

The derivatives of the cotangent and cosecant are similar and left as
exercises.  Putting this all together, we have:

\begin{theorem}[The Derivatives of Trigonometric Functions] \hfil
\begin{itemize}
\item $\ddx \sin(x) = \cos(x)$.
\item $\ddx \cos(x) = -\sin(x)$.
\item $\ddx \tan(x) = \sec^2(x)$.
\item $\ddx \sec(x) = \sec(x)\tan(x)$.
\item $\ddx \csc(x) = -\csc(x)\cot(x)$.
\item $\ddx \cot(x) = -\csc^2(x)$.
\end{itemize}
\end{theorem}

\begin{example}
Compute:
\[
\eval{\ddx ( \csc(x) \cot(x) )}_{x=\frac{\pi}{3}}
\]
\begin{explanation}
Applying the product rule the facts above, we know that
\[
\ddx ( \csc(x) \cot(x) ) = - \csc^3(x) - \cot^2(x)\answer[given]{\csc(x)}
\]
and so
\[
\eval{\ddx ( \csc(x) \cot(x) )}_{x=\frac{\pi}{3}}
\]
\begin{align*}
  &= \eval{  - \csc^3(x) - \cot^2(x) \answer[given]{\csc(x)}}_{x=\frac{\pi}{3}}  \\
&= - \frac{8}{3 \sqrt{3}} - \frac{1}{3}\cdot \answer[given]{2/\sqrt{3}}
\end{align*}
\end{explanation}
\end{example}





\end{document}
