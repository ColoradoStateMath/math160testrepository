\documentclass{ximera}

\input{../preamble.tex}

\title{Lab 5: An Application of Integration - Arc Length}

\begin{document}

\begin{abstract}
In this lab, we will explore how we can calculate arc length using a definite integral.  This lab corresponds to section 6.3 in our physical textbook, so feel free to use the textbook as a resource while completing this lab.\\
\end{abstract}

\maketitle

\begin{sectionOutcomes}

After completing this lab, you should be able to do the following: 

\begin{itemize}
\item Explain what is meant by the `arc length' of a function.
\item Describe how to approximate the arc length of a function on a specified interval using a fixed number of line segments.
\item Give the definition of a smooth function and determine if a particular function is smooth on an interval.
\item State a formula that gives the arc length of a smooth function on a specified interval. 
\item Determine the arc length of a smooth function using the arc length formula.
\item Find the arc length of a function that is smooth with respect to y but not x.
\item Find the arc length of a function that is not smooth at its endpoints but has symmetry. 
\end{itemize}

\end{sectionOutcomes}

\end{document}
