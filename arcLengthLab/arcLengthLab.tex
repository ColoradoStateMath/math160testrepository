\documentclass[handout,nooutcomes]{ximera}
%handout
%wordchoicegiven
%space
%nooutcomes
\title{Math 160 Lab 5}
\author{The Moomaster} 
%\usepackage{todonotes}

\newcommand{\todo}{}

\usepackage{esint} % for \oiint
\graphicspath{
{./}
{functionsOfSeveralVariables/}
{normalVectors/}
{lagrangeMultipliers/}
{vectorFields/}
{greensTheorem/}
{shapeOfThingsToCome/}
}


\usepackage{tkz-euclide}
\tikzset{>=stealth} %% cool arrow head
\tikzset{shorten <>/.style={ shorten >=#1, shorten <=#1 } } %% allows shorter vectors

\usetikzlibrary{backgrounds} %% for boxes around graphs
\usetikzlibrary{shapes,positioning}  %% Clouds and stars
\usetikzlibrary{matrix} %% for matrix
\usepgfplotslibrary{polar} %% for polar plots
\usetkzobj{all}
\usepackage[makeroom]{cancel} %% for strike outs
%\usepackage{mathtools} %% for pretty underbrace % Breaks Ximera
\usepackage{multicol}
\usepackage{pgffor} %% required for integral for loops


%% http://tex.stackexchange.com/questions/66490/drawing-a-tikz-arc-specifying-the-center
%% Draws beach ball
\tikzset{pics/carc/.style args={#1:#2:#3}{code={\draw[pic actions] (#1:#3) arc(#1:#2:#3);}}}



\usepackage{array}
\setlength{\extrarowheight}{+.1cm}   
\newdimen\digitwidth
\settowidth\digitwidth{9}
\def\divrule#1#2{
\noalign{\moveright#1\digitwidth
\vbox{\hrule width#2\digitwidth}}}





\newcommand{\RR}{\mathbb R}
\newcommand{\R}{\mathbb R}
\newcommand{\N}{\mathbb N}
\newcommand{\Z}{\mathbb Z}

%\newcommand{\sage}{\textsf{SageMath}}


%\renewcommand{\d}{\,d\!}
\renewcommand{\d}{\mathop{}\!d}
\newcommand{\dd}[2][]{\frac{\d #1}{\d #2}}
\newcommand{\pp}[2][]{\frac{\partial #1}{\partial #2}}
\renewcommand{\l}{\ell}
\newcommand{\ddx}{\frac{d}{\d x}}

\newcommand{\zeroOverZero}{\ensuremath{\boldsymbol{\tfrac{0}{0}}}}
\newcommand{\inftyOverInfty}{\ensuremath{\boldsymbol{\tfrac{\infty}{\infty}}}}
\newcommand{\zeroOverInfty}{\ensuremath{\boldsymbol{\tfrac{0}{\infty}}}}
\newcommand{\zeroTimesInfty}{\ensuremath{\small\boldsymbol{0\cdot \infty}}}
\newcommand{\inftyMinusInfty}{\ensuremath{\small\boldsymbol{\infty - \infty}}}
\newcommand{\oneToInfty}{\ensuremath{\boldsymbol{1^\infty}}}
\newcommand{\zeroToZero}{\ensuremath{\boldsymbol{0^0}}}
\newcommand{\inftyToZero}{\ensuremath{\boldsymbol{\infty^0}}}



\newcommand{\numOverZero}{\ensuremath{\boldsymbol{\tfrac{\#}{0}}}}
\newcommand{\dfn}{\textbf}
%\newcommand{\unit}{\,\mathrm}
\newcommand{\unit}{\mathop{}\!\mathrm}
\newcommand{\eval}[1]{\bigg[ #1 \bigg]}
\newcommand{\seq}[1]{\left( #1 \right)}
\renewcommand{\epsilon}{\varepsilon}
\renewcommand{\phi}{\varphi}


\renewcommand{\iff}{\Leftrightarrow}

\DeclareMathOperator{\arccot}{arccot}
\DeclareMathOperator{\arcsec}{arcsec}
\DeclareMathOperator{\arccsc}{arccsc}
\DeclareMathOperator{\si}{Si}
\DeclareMathOperator{\proj}{\vec{proj}}
\DeclareMathOperator{\scal}{scal}
\DeclareMathOperator{\sign}{sign}


%% \newcommand{\tightoverset}[2]{% for arrow vec
%%   \mathop{#2}\limits^{\vbox to -.5ex{\kern-0.75ex\hbox{$#1$}\vss}}}
\newcommand{\arrowvec}{\overrightarrow}
%\renewcommand{\vec}[1]{\arrowvec{\mathbf{#1}}}
\renewcommand{\vec}{\mathbf}
\newcommand{\veci}{{\boldsymbol{\hat{\imath}}}}
\newcommand{\vecj}{{\boldsymbol{\hat{\jmath}}}}
\newcommand{\veck}{{\boldsymbol{\hat{k}}}}
\newcommand{\vecl}{\boldsymbol{\l}}
\newcommand{\uvec}[1]{\mathbf{\hat{#1}}}
\newcommand{\utan}{\mathbf{\hat{t}}}
\newcommand{\unormal}{\mathbf{\hat{n}}}
\newcommand{\ubinormal}{\mathbf{\hat{b}}}

\newcommand{\dotp}{\bullet}
\newcommand{\cross}{\boldsymbol\times}
\newcommand{\grad}{\boldsymbol\nabla}
\newcommand{\divergence}{\grad\dotp}
\newcommand{\curl}{\grad\cross}
%\DeclareMathOperator{\divergence}{divergence}
%\DeclareMathOperator{\curl}[1]{\grad\cross #1}
\newcommand{\lto}{\mathop{\longrightarrow\,}\limits}

\renewcommand{\bar}{\overline}

\colorlet{textColor}{black} 
\colorlet{background}{white}
\colorlet{penColor}{blue!50!black} % Color of a curve in a plot
\colorlet{penColor2}{red!50!black}% Color of a curve in a plot
\colorlet{penColor3}{red!50!blue} % Color of a curve in a plot
\colorlet{penColor4}{green!50!black} % Color of a curve in a plot
\colorlet{penColor5}{orange!80!black} % Color of a curve in a plot
\colorlet{penColor6}{yellow!70!black} % Color of a curve in a plot
\colorlet{fill1}{penColor!20} % Color of fill in a plot
\colorlet{fill2}{penColor2!20} % Color of fill in a plot
\colorlet{fillp}{fill1} % Color of positive area
\colorlet{filln}{penColor2!20} % Color of negative area
\colorlet{fill3}{penColor3!20} % Fill
\colorlet{fill4}{penColor4!20} % Fill
\colorlet{fill5}{penColor5!20} % Fill
\colorlet{gridColor}{gray!50} % Color of grid in a plot

\newcommand{\surfaceColor}{violet}
\newcommand{\surfaceColorTwo}{redyellow}
\newcommand{\sliceColor}{greenyellow}




\pgfmathdeclarefunction{gauss}{2}{% gives gaussian
  \pgfmathparse{1/(#2*sqrt(2*pi))*exp(-((x-#1)^2)/(2*#2^2))}%
}


%%%%%%%%%%%%%
%% Vectors
%%%%%%%%%%%%%

%% Simple horiz vectors
\renewcommand{\vector}[1]{\left\langle #1\right\rangle}


%% %% Complex Horiz Vectors with angle brackets
%% \makeatletter
%% \renewcommand{\vector}[2][ , ]{\left\langle%
%%   \def\nextitem{\def\nextitem{#1}}%
%%   \@for \el:=#2\do{\nextitem\el}\right\rangle%
%% }
%% \makeatother

%% %% Vertical Vectors
%% \def\vector#1{\begin{bmatrix}\vecListA#1,,\end{bmatrix}}
%% \def\vecListA#1,{\if,#1,\else #1\cr \expandafter \vecListA \fi}

%%%%%%%%%%%%%
%% End of vectors
%%%%%%%%%%%%%

%\newcommand{\fullwidth}{}
%\newcommand{\normalwidth}{}



%% makes a snazzy t-chart for evaluating functions
%\newenvironment{tchart}{\rowcolors{2}{}{background!90!textColor}\array}{\endarray}

%%This is to help with formatting on future title pages.
\newenvironment{sectionOutcomes}{}{} 



%% Flowchart stuff
%\tikzstyle{startstop} = [rectangle, rounded corners, minimum width=3cm, minimum height=1cm,text centered, draw=black]
%\tikzstyle{question} = [rectangle, minimum width=3cm, minimum height=1cm, text centered, draw=black]
%\tikzstyle{decision} = [trapezium, trapezium left angle=70, trapezium right angle=110, minimum width=3cm, minimum height=1cm, text centered, draw=black]
%\tikzstyle{question} = [rectangle, rounded corners, minimum width=3cm, minimum height=1cm,text centered, draw=black]
%\tikzstyle{process} = [rectangle, minimum width=3cm, minimum height=1cm, text centered, draw=black]
%\tikzstyle{decision} = [trapezium, trapezium left angle=70, trapezium right angle=110, minimum width=3cm, minimum height=1cm, text centered, draw=black]

\outcome{Explain what is meant by the `arc length' of a function.}
\outcome{Describe how to approximate the arc length of a function on a specified interval using a fixed number of line segments.}
\outcome{Give the definition of a smooth function and determine if a particular function is smooth on an interval.}
\outcome{State a formula that gives the arc length of a smooth function on a specified interval.}
\outcome{Determine the arc length of a smooth function using the arc length formula.}
\outcome{Find the arc length of a function that is smooth with respect to y but not x.}
\outcome{Find the arc length of a function that is not smooth at its endpoints but has symmetry.}

\begin{document}

\section{Calculus 1 Lab 5 \\ An Application of Integration - Arc Length}

%% Have to edit the date here each semester.
\begin{abstract}
This is Lab 5 for Math 160 - Due Friday, December 1, 2017 at 5:00PM MST.  In this lab, we will explore how we can calculate arc length using a definite integral.  This lab corresponds to section 6.3 in our physical textbook, so feel free to use the textbook as a resource while completing this lab.\\

Unless stated otherwise, input answers in \underline{exact form} on this lab.
\end{abstract}

\maketitle

%%Introduction%%

In the last few weeks of math 160, you will learn about using definite integrals to calculate the areas of 2D regions and volumes of 3D regions.  Integrals can also be used to evaluate the arc length of a function.  Consider the function $f(x) = \sin(x)$ on the interval $[0,2\pi]$ as graphed below.  \\

%%INSERT IMAGE HERE - TIKZ%%

Imagine laying a string over $f(x)$.  What length of string would you need in order for the string to exactly lay on top of $f(x)$ on the interval $[0, 2\pi]$?  This length is called the `arc length' of the function on the interval $[0, 2\pi]$, and the goal of this lab is to determine how to calculate this length exactly.  \\

At some point in your math education, you probably learned how to calculate the length of an arc on a circle.  The idea of the arc length of a curve is actually the same, except now we would like to calculate the length of \textit{any} curve, not just circular ones.  Aside from circles, you also have the tools to calculate the arc length of a linear function on a particular interval.  For example, consider the function $y = 2x$ graphed below.  Let's find this function's arc length on a few intervals.

\begin{problem}
Let's find the arc length of $f$ on the interval $[0,1]$.  Ultimately, this question is equivalent to finding the length of the line segment of $f$ that lies in $[0,1]$, and we know how to find the distance of line segments!  \\

The endpoints of this line segment are $(0,\answer{0})$ and $(1,\answer{2})$, so the $x$-distance between these endpoints is $\Delta x = \answer{1}$ and the $y$-distance between these endpoints is $\Delta y = \answer{2}$.  Therefore, using the Pythagorean Theorem, the distance of this line segment and therefore the arc length of $f(x) = 2x$ on the interval $[0,1]$ is $\sqrt{\answer{1}^2+\answer{2}^2} = \answer{\sqrt{5}}$
\end{problem}

\begin{problem}
Similarly, we can find the arc length of $f(x)=2x$ on the interval $[-1,2]$.  Using the Pythagorean Theorem again, the arc length of $f$ on $[-1,2]$ can be determined to be $\answer{\sqrt{45}}$.
\end{problem}

In a similar fashion, the Pythagorean Theorem could be used to calculate the arc length of \textit{any} linear function.  Let's use the fact that we can find the length of any line segment in order to approximate the arc length of the function $\sin(x)$ on the interval $[0, 2\pi]$.  Just like we used rectangles to approximate area, let's use line segments to approximate arc length.  For this example, let's use $n=4$ line segments.  Now, we can partition $[0, 2\pi]$ into 4 equal sub-intervals, $[0, \pi/2]$, $[\pi/2, \pi]$, $[\pi, 3\pi/2]$, and $[3\pi/2, 2\pi]$ and draw a line segment to approximate the arc length on each sub-interval, as shown below.  Finally, to approximate arc length, we must add up the length of these 4 line segments.  Let's do it! \\

%%Arc Length Approximation Picture%%

\begin{problem}
Determine the length of each of the four line segments to 2 decimal places.

$L_1 = \answer{1.86}$
$L_2 = \answer{1.86}$
$L_3 = \answer{1.86}$
$L_4 = \answer{1.86}$

So the approximate arc length of $\sin(x)$ on $[0, 2\pi]$ is $\displaystyle\sum_{k=1}^4 L_k$ = $\answer{7.45}$.
\end{problem}

Of course, this is only an approximation of the arc length.  In order to get the \textbf{exact} arc length of $f(x) = \sin(x)$ on $[0, 2\pi]$, we will need to take a limit as the number of line segments we use ($n$) tends to infinity.  Namely, because an approximation with $n$ line segments will give an approximate arc length of $\displaystyle\sum_{k=1}^n L_k$, the exact arc length of $f(x)$ on $[0, 2\pi]$ is given by $\displaystyle\lim_{n \to \infty} \displaystyle\sum_{k=1}^n L_k$.  Unfortunately, we can't evaluate this limit in its current form, so we'll need to write this in another form before we're done.  \\

Let's move away from $\sin(x)$ for a bit and try to find a formula for any function $f(x)$ on the interval $[a,b]$.  As with the sine example, we can partition the interval $[a,b]$ into $n$ equal length sub-intervals, $[a=x_0, x_1], [x_1, x_2], ..., [x_{n-1}, x_n = b]$.  On each of these sub-intervals, we will calculate the length of the corresponding line segment that approximates the arc length.

\begin{problem}
Let's say that we're working on the $k$th sub-interval, $[x_{k-1}, x_k]$.  The endpoints on the line segment on this interval are the points 
\begin{selectAll}
    \choice{Where $A(n_0)$ does not exist.}
    \choice[correct]{Where $A'(n_0)$ does not exist.}
    \choice{Where $A(n_0)$ is zero.}
    \choice[correct]{Where $A'(n_0)$ is zero.}
    \choice[correct]{At endpoints of the domain of $A(n)$}
    \choice{At endpoints of the domain of $A'(n)$}
\end{selectAll}
\end{problem}



%%Practice Problem 1 - Checking sinx Estimate%%

Now that we have derived a formula for the arc length of a curve on a particular interval, let's use this formula to determine the arc length of $y=\sin(x)$ on the interval $[0,2\pi]$.  Of course, before using this formula, we must verify that $y=\sin(x)$ is smooth on $[0,2\pi]$.

\begin{problem}

$y' = \answer{\cos(x)}$

\begin{problem}

Therefore $y$ is \wordChoice {\choice[correct]{smooth}\choice{not smooth}} on $[0,2\pi]$ because \wordChoice {\choice{$y$ is continuous} \choice[correct]{$y'$ is continuous} \choice{$y$ is not continuous} \choice{$y'$ is not continuous}} on $[0,2\pi]$.  

\begin{problem}

Finally, we can write an integral to evaluate the arc length of the curve on the specified interval and evaluate it using technology: 

Arc Length = $\displaystyle\int_{\answer{0}}^{\answer{2\pi}} \sqrt{1+{\answer{\cos(x)}^2}} \ dx \approx \answer{7.64}$ (Evaluate the arc length to 2 decimal places with technology)

\begin{feedback}[correct]
Our original approximation of this arc length with $n=4$ line segments was 7.45, which turns out to be pretty close to the exact value. 
\end{feedback}


\end{problem}
\end{problem}
\end{problem}

Let's try calculating the arc length of another function.  This time, you will be able to evaluate the resulting integral for arc length by hand using the techniques we've learned in class.

%%Practice Problem 2 - Standard WRT x%%

\begin{problem}

Find the exact arc length of $y = x^{\frac{3}{2}}$ from $x=0$ to $x=4$.  First, we determine if $y$ is smooth on the x-interval $[0,4]$.  

$y' = \answer{\frac{3}{2}\sqrt{x}}$, so $y$ is \wordChoice {\choice[correct]{smooth}\choice{not smooth}} on $[0,4]$ because \wordChoice {\choice{$y$ is continuous} \choice[correct]{$y'$ is continuous} \choice{$y$ is not continuous} \choice{$y'$ is not continuous}} on $[0,5]$.  

\begin{problem}

Arc Length = $\displaystyle\int_{\answer{0}}^{\answer{5}} \answer{\sqrt{1+\frac{9x}{4}}} \ dx = \answer{\frac{8}{27}(10^{\frac{3}{2}})-1}$ (Evaluate the arc length exactly)

\end{problem}

\end{problem}


\vspace{2.5 in}

%%Practice Problem 3 - WRT y%%

Find the arc length of $x = \frac{2}{3} (y-1)^{\frac{3}{2}}$ from $x=1$ to $x=4$. 

%%Practice Problem 4 - Integrating WRT y When $f$ isn't smooth WRT x%%

Find the arc length of $f(x) = (\frac{x}{2})^{\frac{2}{3}}$ on $[0,2]$. 

\vspace{2.5 in}

%%Practice Problem 5- Using symmetry when $f$ isn't smooth WRT x%%

\end{document}
