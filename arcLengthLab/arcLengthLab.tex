\documentclass[handout,nooutcomes]{ximera}
%handout
%wordchoicegiven
%space
%nooutcomes
\title{Math 160 Lab 5}
\author{The Moomaster} 
\input{../preamble.tex}

\outcome{Explain what is meant by the `arc length' of a function.}
\outcome{Describe how to approximate the arc length of a function on a specified interval using a fixed number of line segments.}
\outcome{Give the definition of a smooth function and determine if a particular function is smooth on an interval.}
\outcome{State a formula that gives the arc length of a smooth function on a specified interval.}
\outcome{Determine the arc length of a smooth function using the arc length formula.}
\outcome{Find the arc length of a function that is smooth with respect to y but not x.}
\outcome{Find the arc length of a function that is not smooth at its endpoints but has symmetry.}

\begin{document}

\section{Calculus 1 Lab 5 \\ An Application of Integration - Arc Length}

%% Have to edit the date here each semester.
\begin{abstract}
This is Lab 5 for Math 160 - Due Friday, December 1, 2017 at 5:00PM MST.  In this lab, we will explore how we can calculate arc length using a definite integral.  This lab corresponds to section 6.3 in our physical textbook, so feel free to use the textbook as a resource while completing this lab.\\

Unless stated otherwise, input answers in \underline{exact form} on this lab.
\end{abstract}

\maketitle

%%Introduction%%

In the last few weeks of math 160, you will learn about using definite integrals to calculate the areas of 2D regions and volumes of 3D regions.  Integrals can also be used to evaluate the arc length of a function.  Consider the function $f(x) = \sin(x)$ on the interval $[0,2\pi]$ as graphed below.  \\

%%INSERT IMAGE HERE - TIKZ%%

Imagine laying a string over $f(x)$.  What length of string would you need in order for the string to exactly lay on top of $f(x)$ on the interval $[0, 2\pi]$?  This length is called the `arc length' of the function on the interval $[0, 2\pi]$, and the goal of this lab is to determine how to calculate this length exactly.  \\

At some point in your math education, you probably learned how to calculate the length of an arc on a circle.  The idea of the arc length of a curve is actually the same, except now we would like to calculate the length of \textit{any} curve, not just circular ones.  Aside from circles, you also have the tools to calculate the arc length of a linear function on a particular interval.  For example, consider the function $y = 2x$ graphed below.  Let's find this function's arc length on a few intervals.

\begin{problem}
Let's find the arc length of $f$ on the interval $[0,1]$.  Ultimately, this question is equivalent to finding the length of the line segment of $f$ that lies in $[0,1]$, and we know how to find the distance of line segments!  \\

The endpoints of this line segment are $(0,\answer{0})$ and $(1,\answer{2})$, so the $x$-distance between these endpoints is $\Delta x = \answer{1}$ and the $y$-distance between these endpoints is $\Delta y = \answer{2}$.  Therefore, using the Pythagorean Theorem, the distance of this line segment and therefore the arc length of $y = 2x$ on the interval $[0,1]$ is $\sqrt{\answer{1}^2+\answer{2}^2} = \answer{\sqrt{10}}$
\end{problem}

\begin{problem}
Similarly, we can find the arc length of $f$ on the interval $[-1,5]$.  Using the Pythagorean Theorem again, the arc length of $f$ on $[-1,2]$ can be determined to be $\answer{6}$.
\end{problem}

In a similar fashion, the Pythagorean Theorem could be used to calculate the arc length of \textit{any} linear function.  Let's use the fact that we can find the length of any line segment in order to approximate the arc length of the function $\sin(x)$ on the interval $[0, 2\pi]$.  Just like we used rectangles to approximate area, let's use line segments to approximate arc length.  For this example, let's use $n=4$ line segments.  Now, we can partition $[0, 2\pi]$ into 4 equal sub-intervals, $[0, \pi/2]$, $[\pi/2, \pi]$, $[\pi, 3\pi/2]$, and $[3\pi/2, 2\pi]$ and draw a line segment to approximate the arc length on each sub-interval, as shown below.  Finally, to approximate arc length, we must add up the length of these 4 line segments.  Let's do it! \\

%%Arc Length Approximation Picture%%

\begin{problem}
Determine the length of each of the four line segments to 2 decimal places.

$L_1 = \answer{1.86}$
$L_2 = \answer{1.86}$
$L_3 = \answer{1.86}$
$L_4 = \answer{1.86}$

So the approximate arc length of $\sin(x)$ on $[0, 2\pi]$ is $\displaystyle\sum_1^4 L_k$ = $\answer{7.45}$.
\end{problem}

Of course, this is only an approximation of the arc length.  In order to get the \textbf{exact} arc length of $f(x) = \sin(x)$ on $[0, 2\pi]$, we will need to take a limit as the number of line segments we use ($n$) tends to infinity.  Namely, because an approximation with $n$ line segments will give an approximate arc length of $\displaystyle\sum_1^n L_k$, the exact arc length of $f(x)$ on $[0, 2\pi]$ is given by $\displaystyle\lim_{n \to \infty} \displaystyle\sum_1^n L_k$.  Unfortunately, we can't evaluate this limit in its current form, so we'll need to write this in another form before we're done.  \\

Let's move away from $\sin(x)$ for a bit and try to find a formula for any function $f(x)$ on the interval $[a,b]$.  As with the sine example, we can partition the interval $[a,b]$ into $n$ equal length sub-intervals, $[a=x_0, x_1], [x_1, x_2], ..., [x_{n-1}, x_n = b]$.  On each of these sub-intervals, we will calculate the length of the corresponding line segment that approximates the arc length.

\begin{problem}
Let's say that we're working on the $k$th sub-interval, $[x_{k-1}, x_k]$.  The endpoints on the line segment on this interval are the points 
\end{problem}

\begin{problem}
\hspace{2cm}Yikes! Ok, that's a beast, but let's not lose our heads. We're hunting critical points after all. Where do we look for critical points?
\begin{selectAll}
    \choice{Where $A(n_0)$ does not exist.}
    \choice[correct]{Where $A'(n_0)$ does not exist.}
    \choice{Where $A(n_0)$ is zero.}
    \choice[correct]{Where $A'(n_0)$ is zero.}
    \choice[correct]{At endpoints of the domain of $A(n)$}
    \choice{At endpoints of the domain of $A'(n)$}
\end{selectAll}
\end{problem}



\end{document}
