\documentclass{ximera}

%\usepackage{todonotes}

\newcommand{\todo}{}

\usepackage{esint} % for \oiint
\graphicspath{
{./}
{functionsOfSeveralVariables/}
{normalVectors/}
{lagrangeMultipliers/}
{vectorFields/}
{greensTheorem/}
{shapeOfThingsToCome/}
}


\usepackage{tkz-euclide}
\tikzset{>=stealth} %% cool arrow head
\tikzset{shorten <>/.style={ shorten >=#1, shorten <=#1 } } %% allows shorter vectors

\usetikzlibrary{backgrounds} %% for boxes around graphs
\usetikzlibrary{shapes,positioning}  %% Clouds and stars
\usetikzlibrary{matrix} %% for matrix
\usepgfplotslibrary{polar} %% for polar plots
\usetkzobj{all}
\usepackage[makeroom]{cancel} %% for strike outs
%\usepackage{mathtools} %% for pretty underbrace % Breaks Ximera
\usepackage{multicol}
\usepackage{pgffor} %% required for integral for loops


%% http://tex.stackexchange.com/questions/66490/drawing-a-tikz-arc-specifying-the-center
%% Draws beach ball
\tikzset{pics/carc/.style args={#1:#2:#3}{code={\draw[pic actions] (#1:#3) arc(#1:#2:#3);}}}



\usepackage{array}
\setlength{\extrarowheight}{+.1cm}   
\newdimen\digitwidth
\settowidth\digitwidth{9}
\def\divrule#1#2{
\noalign{\moveright#1\digitwidth
\vbox{\hrule width#2\digitwidth}}}





\newcommand{\RR}{\mathbb R}
\newcommand{\R}{\mathbb R}
\newcommand{\N}{\mathbb N}
\newcommand{\Z}{\mathbb Z}

%\newcommand{\sage}{\textsf{SageMath}}


%\renewcommand{\d}{\,d\!}
\renewcommand{\d}{\mathop{}\!d}
\newcommand{\dd}[2][]{\frac{\d #1}{\d #2}}
\newcommand{\pp}[2][]{\frac{\partial #1}{\partial #2}}
\renewcommand{\l}{\ell}
\newcommand{\ddx}{\frac{d}{\d x}}

\newcommand{\zeroOverZero}{\ensuremath{\boldsymbol{\tfrac{0}{0}}}}
\newcommand{\inftyOverInfty}{\ensuremath{\boldsymbol{\tfrac{\infty}{\infty}}}}
\newcommand{\zeroOverInfty}{\ensuremath{\boldsymbol{\tfrac{0}{\infty}}}}
\newcommand{\zeroTimesInfty}{\ensuremath{\small\boldsymbol{0\cdot \infty}}}
\newcommand{\inftyMinusInfty}{\ensuremath{\small\boldsymbol{\infty - \infty}}}
\newcommand{\oneToInfty}{\ensuremath{\boldsymbol{1^\infty}}}
\newcommand{\zeroToZero}{\ensuremath{\boldsymbol{0^0}}}
\newcommand{\inftyToZero}{\ensuremath{\boldsymbol{\infty^0}}}



\newcommand{\numOverZero}{\ensuremath{\boldsymbol{\tfrac{\#}{0}}}}
\newcommand{\dfn}{\textbf}
%\newcommand{\unit}{\,\mathrm}
\newcommand{\unit}{\mathop{}\!\mathrm}
\newcommand{\eval}[1]{\bigg[ #1 \bigg]}
\newcommand{\seq}[1]{\left( #1 \right)}
\renewcommand{\epsilon}{\varepsilon}
\renewcommand{\phi}{\varphi}


\renewcommand{\iff}{\Leftrightarrow}

\DeclareMathOperator{\arccot}{arccot}
\DeclareMathOperator{\arcsec}{arcsec}
\DeclareMathOperator{\arccsc}{arccsc}
\DeclareMathOperator{\si}{Si}
\DeclareMathOperator{\proj}{\vec{proj}}
\DeclareMathOperator{\scal}{scal}
\DeclareMathOperator{\sign}{sign}


%% \newcommand{\tightoverset}[2]{% for arrow vec
%%   \mathop{#2}\limits^{\vbox to -.5ex{\kern-0.75ex\hbox{$#1$}\vss}}}
\newcommand{\arrowvec}{\overrightarrow}
%\renewcommand{\vec}[1]{\arrowvec{\mathbf{#1}}}
\renewcommand{\vec}{\mathbf}
\newcommand{\veci}{{\boldsymbol{\hat{\imath}}}}
\newcommand{\vecj}{{\boldsymbol{\hat{\jmath}}}}
\newcommand{\veck}{{\boldsymbol{\hat{k}}}}
\newcommand{\vecl}{\boldsymbol{\l}}
\newcommand{\uvec}[1]{\mathbf{\hat{#1}}}
\newcommand{\utan}{\mathbf{\hat{t}}}
\newcommand{\unormal}{\mathbf{\hat{n}}}
\newcommand{\ubinormal}{\mathbf{\hat{b}}}

\newcommand{\dotp}{\bullet}
\newcommand{\cross}{\boldsymbol\times}
\newcommand{\grad}{\boldsymbol\nabla}
\newcommand{\divergence}{\grad\dotp}
\newcommand{\curl}{\grad\cross}
%\DeclareMathOperator{\divergence}{divergence}
%\DeclareMathOperator{\curl}[1]{\grad\cross #1}
\newcommand{\lto}{\mathop{\longrightarrow\,}\limits}

\renewcommand{\bar}{\overline}

\colorlet{textColor}{black} 
\colorlet{background}{white}
\colorlet{penColor}{blue!50!black} % Color of a curve in a plot
\colorlet{penColor2}{red!50!black}% Color of a curve in a plot
\colorlet{penColor3}{red!50!blue} % Color of a curve in a plot
\colorlet{penColor4}{green!50!black} % Color of a curve in a plot
\colorlet{penColor5}{orange!80!black} % Color of a curve in a plot
\colorlet{penColor6}{yellow!70!black} % Color of a curve in a plot
\colorlet{fill1}{penColor!20} % Color of fill in a plot
\colorlet{fill2}{penColor2!20} % Color of fill in a plot
\colorlet{fillp}{fill1} % Color of positive area
\colorlet{filln}{penColor2!20} % Color of negative area
\colorlet{fill3}{penColor3!20} % Fill
\colorlet{fill4}{penColor4!20} % Fill
\colorlet{fill5}{penColor5!20} % Fill
\colorlet{gridColor}{gray!50} % Color of grid in a plot

\newcommand{\surfaceColor}{violet}
\newcommand{\surfaceColorTwo}{redyellow}
\newcommand{\sliceColor}{greenyellow}




\pgfmathdeclarefunction{gauss}{2}{% gives gaussian
  \pgfmathparse{1/(#2*sqrt(2*pi))*exp(-((x-#1)^2)/(2*#2^2))}%
}


%%%%%%%%%%%%%
%% Vectors
%%%%%%%%%%%%%

%% Simple horiz vectors
\renewcommand{\vector}[1]{\left\langle #1\right\rangle}


%% %% Complex Horiz Vectors with angle brackets
%% \makeatletter
%% \renewcommand{\vector}[2][ , ]{\left\langle%
%%   \def\nextitem{\def\nextitem{#1}}%
%%   \@for \el:=#2\do{\nextitem\el}\right\rangle%
%% }
%% \makeatother

%% %% Vertical Vectors
%% \def\vector#1{\begin{bmatrix}\vecListA#1,,\end{bmatrix}}
%% \def\vecListA#1,{\if,#1,\else #1\cr \expandafter \vecListA \fi}

%%%%%%%%%%%%%
%% End of vectors
%%%%%%%%%%%%%

%\newcommand{\fullwidth}{}
%\newcommand{\normalwidth}{}



%% makes a snazzy t-chart for evaluating functions
%\newenvironment{tchart}{\rowcolors{2}{}{background!90!textColor}\array}{\endarray}

%%This is to help with formatting on future title pages.
\newenvironment{sectionOutcomes}{}{} 



%% Flowchart stuff
%\tikzstyle{startstop} = [rectangle, rounded corners, minimum width=3cm, minimum height=1cm,text centered, draw=black]
%\tikzstyle{question} = [rectangle, minimum width=3cm, minimum height=1cm, text centered, draw=black]
%\tikzstyle{decision} = [trapezium, trapezium left angle=70, trapezium right angle=110, minimum width=3cm, minimum height=1cm, text centered, draw=black]
%\tikzstyle{question} = [rectangle, rounded corners, minimum width=3cm, minimum height=1cm,text centered, draw=black]
%\tikzstyle{process} = [rectangle, minimum width=3cm, minimum height=1cm, text centered, draw=black]
%\tikzstyle{decision} = [trapezium, trapezium left angle=70, trapezium right angle=110, minimum width=3cm, minimum height=1cm, text centered, draw=black]

\outcome{Compute limits of families of functions.} 
\outcome{Compute average velocity.}
\outcome{Approximate instantaneous velocity.}
\outcome{Compare average and instantaneous velocity.}
\outcome{Plot difference quotients for varying approximations of the instantaneous rate of change.}

\title[Dig-In:]{Instantaneous velocity}

\begin{document}
\begin{abstract}
We use limits to compute instantaneous velocity.
\end{abstract}
\maketitle

When one computes average velocity, we look at 
\[
\frac{\text{change in position}}{\text{change in time}}.
\]
To obtain the (instantaneous) velocity, we want the change in time to
``go to'' zero. By this point we should know that ``go to'' is a
buzz-word for a \textit{limit}. The change in time is often given as
an interval whose length goes to zero.  However, intervals must always
be written
\[
[a,b] \qquad\text{where $a < b$.}
\]
Given
\[
I = [a, a+h],
\]
we see that $h$ cannot be negative, or else it violates the notation
for intervals. Hence, if we want smaller, and smaller, intervals
around a point $a$, and we want $h$ to be able to be negative, we
write
\[
I_h = 
\begin{cases}
  [a+h,a]  & \text{if $h<0$}, \\ %% note this is MORE correct than std books
  [a,a+h]  & \text{if $0<h$}.     %% in the content section, we can explain this in detail
\end{cases}
\]
\begin{question}
  Let $a = 3$ and $h = 0.1$
  \[
  I_h = \left[\answer{3},\answer{3.1} \right]
  \]
  \begin{question}
    Let $a = 3$ and $h = -0.1$
  \[
  I_h = \left[\answer{2.9},\answer{3} \right]
  \]
  \end{question}
\end{question}

Regardless of the value of $h$, the average velocity on the interval
$I_h$ is computed by
\[
\frac{\text{change in position}}{\text{change in time}} =
\frac{s(t+h)-s(t)}{h}.
\]
We will be most interested in this ratio when $h$ approaches zero.
Let's put all of this together by working an example.

\begin{example}
A group of young mathematicians recently took a road trip from
Columbus Ohio to Urbana-Champaign Illinois. The position (west of
Columbus, Ohio) of van they drove in is roughly modeled by
\[
s(t) = 36t^2 - 4.8t^3 \qquad\text{(miles West of Columbus)} %% note the model is wrong
\]
on the interval $[0,5]$, where $t$ is measured in hours. What is the
average velocity on the interval $[0,5]$?

Additionally, let
\[
I_h= \begin{cases}
  [1+h,1]  & \text{if $h<0$}, \\ %% note this is MORE correct than std books
  [1,1+h]  & \text{if $0<h$}.     %% in the content section, we can explain this in detail
\end{cases}
\]
What is the average velocity on $I_h$ when $h= 0.1$?
What is the average velocity on $I_h$ when $h= -0.1$?
\begin{explanation}
  The average velocity on the interval $[0,5]$ is
  \begin{align*}
  \frac{s(5)-s(0)}{5-0} &= \frac{36\cdot 5^2 - 4.8\cdot 5^3 - (36\cdot 0^2 - 4.8\cdot 0^3)}{5}\\
  &=\frac{\answer[given]{300}}{5}\\
  &=\answer[given]{60}\qquad\text{miles per hour.}
  \end{align*}
  On the other hand, consider the interval
  \[
  I_h = 
  \begin{cases}
    [1+h,1]  & \text{if $h<0$}, \\ %% note this is MORE correct than std books
    [1,1+h]  & \text{if $0<h$}.     %% in the content section, we can explain this in detail
  \end{cases}
  \]
  When $h = 0.1$, the average velocity is
  \[
  \frac{s(1+0.1)-s(1)}{0.1}
  \]
  \begin{align*}
    &= \frac{36\cdot (1.1)^2 - 4.8\cdot (1.1)^3 - (36\cdot 1^2 - 4.8 \cdot 1^3)}{0.1}\\
    &=\frac{\answer[given]{5.9712}}{0.1}\\
    &=\answer[given]{59.712}\qquad\text{miles per hour.}
  \end{align*}
  
  On the other hand, when $h=-0.1$, the average velocity is
  \[
  \frac{s(1-0.1)-s(1)}{-0.1}
  \]
  \begin{align*}
    &= \frac{36\cdot (0.9)^2 - 4.8\cdot (0.9)^3 - (36\cdot 1^2 - 4.8 \cdot 1^3)}{-0.1}\\
    &=\frac{\answer[given]{-5.5392}}{-0.1}\\
    &=\answer[given]{55.392}\qquad\text{miles per hour.}
  \end{align*}
\end{explanation}
\end{example}

In our previous example, we computed \textit{average velocity} on
three different intervals. If we let the size of the interval go to
zero, we get \dfn{instantaneous velocity}. Limits will allow us to
compute instantaneous velocity.  Let's use the same setting as before.

\begin{example}
  The position of van (west of Columbus, Ohio) our young
  mathematicians drove to Urbana-Champaign, Illinois is roughly
  modeled by
  \[
  s(t) = 36t^2 - 4.8t^3 \qquad \text{for $0\le t\le 5$,}
  \] 
  Find a formula for the (instantaneous) velocity of this van.
\begin{explanation}
  Again, we are working with the interval
  \[
  I_h= \begin{cases}
    [t+h,t]  & \text{if $h<0$}, \\ %% note this is MORE correct than std books
    [t,t+h]  & \text{if $0<h$}.     %% in the content section, we can explain this in detail
  \end{cases}
  \]
  To compute the average velocity, we write
  \[
    \frac{s(t+h)-s(t)}{h}
  \]
  but this time, we will let $h$ go to zero.
  Write with me
  \begin{align*}
  \lim_{h\to 0} &\frac{s(t+h)-s(t)}{h}\\
  &= \lim_{h\to 0} \frac{\answer[given]{36(t+h)^2 - 4.8(t+h)^3} - \left(36t^2 - 4.8t^3\right)}{h}
  \end{align*}
  Now expand the numerator of the fraction and combine like-terms:
  \[
  = \lim_{h\to 0}\frac{72 t h + 36h^2 - 14.4 t^2 h - 14.4th^2 - \answer[given]{4.8h^3}}{h}
  \]
  Factor an $h$ from every term in the numerator:
  \[
  = \lim_{h\to 0}\frac{h\left(\answer[given]{72t + 36h - 14.4t^2 - 14.4th - 4.8h^2}\right)}{h}
  \]
  Cancel $h$ from the numerator and denominator:
  \[
  = \lim_{h\to 0}\left(\answer[given]{72t + 36h - 14.4t^2 - 14.4th - 4.8h^2}\right)
  \]
  Plug in $h=0$:
  \[
  = \answer[given]{72t-14.4t^2}
  \]
  This gives us a formula for our instantaneous velocity, $v(t) =
  \answer[given]{72t-14.4t^2}$.  For your viewing enjoyment, check out graphs of both
  $y=s(t)$ and $y=v(t)$:
  \begin{image}
    \begin{tikzpicture}
      \begin{axis}[clip=false,
          domain=0:5,
          xmax=5,
        xmin=0,
        ymax=300,
        ymin=0,
        axis lines =middle, 
        y label style={at={(axis description cs:-0.1,0.5)},rotate=90,anchor=south},
        x label style={at={(axis description cs:0.5,-.2)}},
        xlabel={hours into road-trip}, ylabel={distance traveled in miles},
        ]
        \addplot [very thick, penColor, smooth] {36*x^2 - 4.8*x^3};
        \node at (axis cs:3,230) {\color{penColor}$s$};
      \end{axis}
    \end{tikzpicture}
  \end{image}
  \begin{image}
    \begin{tikzpicture}
      \begin{axis}[
          clip=false,
          domain=0:5,
          xmax=5,
          xmin=0,
          ymax=100,
          ymin=0,
          axis lines =middle, 
          y label style={at={(axis description cs:-0.1,0.5)},rotate=90,anchor=south},
          x label style={at={(axis description cs:0.5,-.2)}},
          xlabel=hours into road-trip, ylabel=velocity in miles per hour]
        \addplot [very thick, penColor2, smooth] {72*x-14.4*x^2};
        \node at (axis cs:3,95) {\color{penColor2}$v$};
      \end{axis}
    \end{tikzpicture}
  \end{image}
\end{explanation}
\end{example}
\end{document}
