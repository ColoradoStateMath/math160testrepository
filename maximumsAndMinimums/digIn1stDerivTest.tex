\documentclass{ximera}

\input{../preamble.tex}

\title[Dig-In]{The First Derivative Test}

\outcome{Define a critical point.}
\outcome{Find critical points.}
\outcome{Define absolute maximum and absolute minimum.}
\outcome{Find the absolute max or min of a continuous function on a closed interval.}
\outcome{Define local maximum and local minimum.}
\outcome{Compare and contrast local and absolute maxima and minima.}
\outcome{Identify situations in which an absolute maximum or minimum is guaranteed.}
\outcome{Classify critical points.}
\outcome{State the First Derivative Test.}
\outcome{Apply the First Derivative Test.}
\outcome{State the Second Derivative Test.}
\outcome{Apply the Second Derivative Test.}
\outcome{Define inflection points.}
\outcome{Find inflection points.}
  
\begin{document}
\begin{abstract}
We use the first derivative to help locate extrema.  
\end{abstract}
\maketitle

\section{The First Derivative Test}

The method of the previous section for deciding whether there is a
local maximum or minimum at a critical point by testing ``near-by''
points is not always convenient. Instead, since we have already had to
compute the derivative to find the critical points, we can use
information about the derivative to decide. Recall that
\begin{itemize}
\item If $f'(x) >0$ on an interval, then $f$ is increasing on that interval.
\item If $f'(x) <0$ on an interval, then $f$ is decreasing on that interval.
\end{itemize}

So how exactly does the derivative tell us whether there is a maximum,
minimum, or neither at a point?  That's where the \textit{first derivative test} comes into play.

\begin{theorem}[First Derivative Test]\index{first derivative test}\label{T:fdt}
Suppose that $f$ is continuous on an interval, and that $f'(a)=0$ for
some value of $a$ in that interval.
\begin{itemize}
\item If $f'(x)>0$ to the left of $a$ and $f'(x)<0$ to the right of
  $a$, then $f(a)$ is the location of a local maximum.
\item If $f'(x)<0$ to the left of $a$ and $f'(x)>0$ to the right of
  $a$, then $f(a)$ is the location of a local minimum.
\item If $f'(x)$ has the same sign to the left and right of $a$,
  then $f(a)$ is not the location of a local extremum.
\end{itemize}
\end{theorem}

\begin{example}\label{E:localextrema}
Consider the function 
\[
f(x) = \frac{x^4}{4}+\frac{x^3}{3}-x^2
\]
Find the intervals on which $f$ is increasing and decreasing and
identify the local extrema of $f$.

\begin{explanation}
Start by computing
\[
\ddx f(x) = \answer[given]{x^3+x^2-2x}.
\]
Now we need to find when this function is positive and when it is
negative. To do this, solve 
\[
f'(x) = \answer[given]{x^3+x^2-2x} =0.
\]
Factor $f'(x)$
\begin{align*}
f'(x) &= \answer[given]{x^3+x^2-2x} \\
&=x(\answer[given]{x^2+x-2})\\
&=x(x+2)\answer[given]{(x-1)}.
\end{align*}
So the critical points (when $f'(x)=0$) are when $x=-2$, $x=0$, and
$x=1$. Now we can check points \textbf{between} the critical points to find
when $f'(x)$ is increasing and decreasing:
\begin{align*}
  f'(-3)&=\answer[given]{-12},\\
  f'(.5)&=\answer[given]{-0.625},\\
  f'(-1)&=\answer[given]{2},\\
  f'(2)&=\answer[given]{8}.
\end{align*}
From this we can make a sign table:

\begin{image}
\begin{tikzpicture}
	\begin{axis}[
            trim axis left,
            scale only axis,
            domain=-3:3,
            ymax=2,
            ymin=-2,
            axis lines=none,
            height=3cm, %% Hard coded height! 
            width=\textwidth, %% width
          ]
          %\addplot [draw=none, fill=fill1, domain=(-3:-2)] {2} \closedcycle;
          %\addplot [draw=none, fill=fill2, domain=(-2:0)] {2} \closedcycle;
          %\addplot [draw=none, fill=fill1, domain=(0:1)] {2} \closedcycle;
          %\addplot [draw=none, fill=fill2, domain=(1:3)] {2} \closedcycle;
          
          \addplot [->,textColor] plot coordinates {(-3,0) (3,0)}; %% axis{0};

          \addplot [->,ultra thick,textColor,shorten <=2pt,shorten >=2pt] plot coordinates {(-3,1.5) (-2,.5)}; %% decreasing
          \addplot [->,ultra thick,textColor,shorten <=2pt,shorten >=2pt] plot coordinates {(-2,.5) (0,1.5)}; %% increasing
          \addplot [->,ultra thick,textColor,shorten <=2pt,shorten >=2pt] plot coordinates {(0,1.5) (1,.5)}; %% decreasing
          \addplot [->,ultra thick,textColor,shorten <=2pt,shorten >=2pt] plot coordinates {(1,.5) (3,1.5)}; %% increasing
          
          \addplot [dashed, textColor] plot coordinates {(-2,0) (-2,2)};
          \addplot [dashed, textColor] plot coordinates {(0,0) (0,2)};
          \addplot [dashed, textColor] plot coordinates {(1,0) (1,2)};
          
          \node at (axis cs:-2,0) [anchor=north,textColor] {\footnotesize$-2$};
          \node at (axis cs:0,0) [anchor=north,textColor] {\footnotesize$0$};
          \node at (axis cs:1,0) [anchor=north,textColor] {\footnotesize$1$};

          \node at (axis cs:-2.5,-.7) [textColor] {\footnotesize$f'(x)<0$};
          \node at (axis cs:.5,-.7) [textColor] {\footnotesize$f'(x)<0$};
          \node at (axis cs:-1,-.7) [textColor] {\footnotesize$f'(x)>0$};
          \node at (axis cs:2,-.7) [textColor] {\footnotesize$f'(x)>0$};

          %% \node at (axis cs:-2.5,-.5) [anchor=north,textColor] {\footnotesize Decreasing};
          %% \node at (axis cs:.5,-.5) [anchor=north,textColor] {\footnotesize Decreasing};
          %% \node at (axis cs:-1,-.5) [anchor=north,textColor] {\footnotesize Increasing};
          %% \node at (axis cs:2,-.5) [anchor=north,textColor] {\footnotesize Increasing};

        \end{axis}
\end{tikzpicture}
\end{image}

Hence $f$ is increasing on $(-2,0)$ and $(1,\infty)$ and $f$ is
decreasing on $(-\infty,-2)$ and $(0,1)$. Moreover, from the first
derivative test, the local maximum is at $x=0$ while the local minima
are at $x=-2$ and $x=1$, see the graphs of of $f(x) =x^4/4 + x^3/3
-x^2$ and $f'(x) = x^3 + x^2 -2x$.
\begin{image}
\begin{tikzpicture}
	\begin{axis}[
            domain=-4:4,
            ymax=5,
            ymin=-5,
            %samples=100,
            axis lines =middle, xlabel=$x$, ylabel=$y$,
            every axis y label/.style={at=(current axis.above origin),anchor=south},
            every axis x label/.style={at=(current axis.right of origin),anchor=west}
          ]
          \addplot [dashed, textColor, smooth] plot coordinates {(-2,0) (-2,-2.667)}; %% {.451};
          \addplot [dashed, textColor, smooth] plot coordinates {(1,0) (1,-.4167)}; %% axis{2.215};

          \addplot [very thick, penColor, smooth] {(x^4)/4 + (x^3)/3 -x^2};
          \addplot [very thick, penColor2, smooth] {x^3 + x^2 -2*x};

          \node at (axis cs:-1.3,-2) [anchor=west] {\color{penColor}$f$};  
          \node at (axis cs:-2.1,2) [anchor=west] {\color{penColor2}$f'$};

          \addplot[color=penColor2,fill=penColor2,only marks,mark=*] coordinates{(-2,0)};  %% closed hole
          \addplot[color=penColor2,fill=penColor2,only marks,mark=*] coordinates{(1,0)};  %% closed hole
          \addplot[color=penColor2,fill=penColor3,only marks,mark=*] coordinates{(0,0)};  %% closed hole
          \addplot[color=penColor,fill=penColor,only marks,mark=*] coordinates{(-2,.-2.667)};  %% closed hole
          \addplot[color=penColor,fill=penColor,only marks,mark=*] coordinates{(1,-.4167)};  %% closed hole
        \end{axis}
\end{tikzpicture}
\end{image}
\end{explanation}
\end{example}


Hence we have seen that if $f'$ is zero and increasing at a point,
then $f$ has a local minimum at the point. If $f'$ is zero and
decreasing at a point then $f$ has a local maximum at the
point. Thus, we see that we can gain information about $f$ by
studying how $f'$ changes. This leads us to our next section.



\end{document}
